
% Default to the notebook output style

    


% Inherit from the specified cell style.




    
\documentclass[11pt]{article}
\author{Collin Heist}
    
    
    \usepackage[T1]{fontenc}
    % Nicer default font (+ math font) than Computer Modern for most use cases
    \usepackage{mathpazo}

    % Basic figure setup, for now with no caption control since it's done
    % automatically by Pandoc (which extracts ![](path) syntax from Markdown).
    \usepackage{graphicx}
    % We will generate all images so they have a width \maxwidth. This means
    % that they will get their normal width if they fit onto the page, but
    % are scaled down if they would overflow the margins.
    \makeatletter
    \def\maxwidth{\ifdim\Gin@nat@width>\linewidth\linewidth
    \else\Gin@nat@width\fi}
    \makeatother
    \let\Oldincludegraphics\includegraphics
    % Set max figure width to be 80% of text width, for now hardcoded.
    \renewcommand{\includegraphics}[1]{\Oldincludegraphics[width=.8\maxwidth]{#1}}
    % Ensure that by default, figures have no caption (until we provide a
    % proper Figure object with a Caption API and a way to capture that
    % in the conversion process - todo).
    \usepackage{caption}
    \DeclareCaptionLabelFormat{nolabel}{}
    \captionsetup{labelformat=nolabel}

    \usepackage{adjustbox} % Used to constrain images to a maximum size 
    \usepackage{xcolor} % Allow colors to be defined
    \usepackage{enumerate} % Needed for markdown enumerations to work
    \usepackage{geometry} % Used to adjust the document margins
    \usepackage{amsmath} % Equations
    \usepackage{amssymb} % Equations
    \usepackage{textcomp} % defines textquotesingle
    % Hack from http://tex.stackexchange.com/a/47451/13684:
    \AtBeginDocument{%
        \def\PYZsq{\textquotesingle}% Upright quotes in Pygmentized code
    }
    \usepackage{upquote} % Upright quotes for verbatim code
    \usepackage{eurosym} % defines \euro
    \usepackage[mathletters]{ucs} % Extended unicode (utf-8) support
    \usepackage[utf8x]{inputenc} % Allow utf-8 characters in the tex document
    \usepackage{fancyvrb} % verbatim replacement that allows latex
    \usepackage{grffile} % extends the file name processing of package graphics 
                         % to support a larger range 
    % The hyperref package gives us a pdf with properly built
    % internal navigation ('pdf bookmarks' for the table of contents,
    % internal cross-reference links, web links for URLs, etc.)
    \usepackage{hyperref}
    \usepackage{longtable} % longtable support required by pandoc >1.10
    \usepackage{booktabs}  % table support for pandoc > 1.12.2
    \usepackage[inline]{enumitem} % IRkernel/repr support (it uses the enumerate* environment)
    \usepackage[normalem]{ulem} % ulem is needed to support strikethroughs (\sout)
                                % normalem makes italics be italics, not underlines
    \usepackage{mathrsfs}
    

    
    
    % Colors for the hyperref package
    \definecolor{urlcolor}{rgb}{0,.145,.698}
    \definecolor{linkcolor}{rgb}{.71,0.21,0.01}
    \definecolor{citecolor}{rgb}{.12,.54,.11}

    % ANSI colors
    \definecolor{ansi-black}{HTML}{3E424D}
    \definecolor{ansi-black-intense}{HTML}{282C36}
    \definecolor{ansi-red}{HTML}{E75C58}
    \definecolor{ansi-red-intense}{HTML}{B22B31}
    \definecolor{ansi-green}{HTML}{00A250}
    \definecolor{ansi-green-intense}{HTML}{007427}
    \definecolor{ansi-yellow}{HTML}{DDB62B}
    \definecolor{ansi-yellow-intense}{HTML}{B27D12}
    \definecolor{ansi-blue}{HTML}{208FFB}
    \definecolor{ansi-blue-intense}{HTML}{0065CA}
    \definecolor{ansi-magenta}{HTML}{D160C4}
    \definecolor{ansi-magenta-intense}{HTML}{A03196}
    \definecolor{ansi-cyan}{HTML}{60C6C8}
    \definecolor{ansi-cyan-intense}{HTML}{258F8F}
    \definecolor{ansi-white}{HTML}{C5C1B4}
    \definecolor{ansi-white-intense}{HTML}{A1A6B2}
    \definecolor{ansi-default-inverse-fg}{HTML}{FFFFFF}
    \definecolor{ansi-default-inverse-bg}{HTML}{000000}

    % commands and environments needed by pandoc snippets
    % extracted from the output of `pandoc -s`
    \providecommand{\tightlist}{%
      \setlength{\itemsep}{0pt}\setlength{\parskip}{0pt}}
    \DefineVerbatimEnvironment{Highlighting}{Verbatim}{commandchars=\\\{\}}
    % Add ',fontsize=\small' for more characters per line
    \newenvironment{Shaded}{}{}
    \newcommand{\KeywordTok}[1]{\textcolor[rgb]{0.00,0.44,0.13}{\textbf{{#1}}}}
    \newcommand{\DataTypeTok}[1]{\textcolor[rgb]{0.56,0.13,0.00}{{#1}}}
    \newcommand{\DecValTok}[1]{\textcolor[rgb]{0.25,0.63,0.44}{{#1}}}
    \newcommand{\BaseNTok}[1]{\textcolor[rgb]{0.25,0.63,0.44}{{#1}}}
    \newcommand{\FloatTok}[1]{\textcolor[rgb]{0.25,0.63,0.44}{{#1}}}
    \newcommand{\CharTok}[1]{\textcolor[rgb]{0.25,0.44,0.63}{{#1}}}
    \newcommand{\StringTok}[1]{\textcolor[rgb]{0.25,0.44,0.63}{{#1}}}
    \newcommand{\CommentTok}[1]{\textcolor[rgb]{0.38,0.63,0.69}{\textit{{#1}}}}
    \newcommand{\OtherTok}[1]{\textcolor[rgb]{0.00,0.44,0.13}{{#1}}}
    \newcommand{\AlertTok}[1]{\textcolor[rgb]{1.00,0.00,0.00}{\textbf{{#1}}}}
    \newcommand{\FunctionTok}[1]{\textcolor[rgb]{0.02,0.16,0.49}{{#1}}}
    \newcommand{\RegionMarkerTok}[1]{{#1}}
    \newcommand{\ErrorTok}[1]{\textcolor[rgb]{1.00,0.00,0.00}{\textbf{{#1}}}}
    \newcommand{\NormalTok}[1]{{#1}}
    
    % Additional commands for more recent versions of Pandoc
    \newcommand{\ConstantTok}[1]{\textcolor[rgb]{0.53,0.00,0.00}{{#1}}}
    \newcommand{\SpecialCharTok}[1]{\textcolor[rgb]{0.25,0.44,0.63}{{#1}}}
    \newcommand{\VerbatimStringTok}[1]{\textcolor[rgb]{0.25,0.44,0.63}{{#1}}}
    \newcommand{\SpecialStringTok}[1]{\textcolor[rgb]{0.73,0.40,0.53}{{#1}}}
    \newcommand{\ImportTok}[1]{{#1}}
    \newcommand{\DocumentationTok}[1]{\textcolor[rgb]{0.73,0.13,0.13}{\textit{{#1}}}}
    \newcommand{\AnnotationTok}[1]{\textcolor[rgb]{0.38,0.63,0.69}{\textbf{\textit{{#1}}}}}
    \newcommand{\CommentVarTok}[1]{\textcolor[rgb]{0.38,0.63,0.69}{\textbf{\textit{{#1}}}}}
    \newcommand{\VariableTok}[1]{\textcolor[rgb]{0.10,0.09,0.49}{{#1}}}
    \newcommand{\ControlFlowTok}[1]{\textcolor[rgb]{0.00,0.44,0.13}{\textbf{{#1}}}}
    \newcommand{\OperatorTok}[1]{\textcolor[rgb]{0.40,0.40,0.40}{{#1}}}
    \newcommand{\BuiltInTok}[1]{{#1}}
    \newcommand{\ExtensionTok}[1]{{#1}}
    \newcommand{\PreprocessorTok}[1]{\textcolor[rgb]{0.74,0.48,0.00}{{#1}}}
    \newcommand{\AttributeTok}[1]{\textcolor[rgb]{0.49,0.56,0.16}{{#1}}}
    \newcommand{\InformationTok}[1]{\textcolor[rgb]{0.38,0.63,0.69}{\textbf{\textit{{#1}}}}}
    \newcommand{\WarningTok}[1]{\textcolor[rgb]{0.38,0.63,0.69}{\textbf{\textit{{#1}}}}}
    
    
    % Define a nice break command that doesn't care if a line doesn't already
    % exist.
    \def\br{\hspace*{\fill} \\* }
    % Math Jax compatibility definitions
    \def\gt{>}
    \def\lt{<}
    \let\Oldtex\TeX
    \let\Oldlatex\LaTeX
    \renewcommand{\TeX}{\textrm{\Oldtex}}
    \renewcommand{\LaTeX}{\textrm{\Oldlatex}}
    % Document parameters
    % Document title
    \title{Math 428 - Exam \#1}
    
    
    
    
    

    % Pygments definitions
    
\makeatletter
\def\PY@reset{\let\PY@it=\relax \let\PY@bf=\relax%
    \let\PY@ul=\relax \let\PY@tc=\relax%
    \let\PY@bc=\relax \let\PY@ff=\relax}
\def\PY@tok#1{\csname PY@tok@#1\endcsname}
\def\PY@toks#1+{\ifx\relax#1\empty\else%
    \PY@tok{#1}\expandafter\PY@toks\fi}
\def\PY@do#1{\PY@bc{\PY@tc{\PY@ul{%
    \PY@it{\PY@bf{\PY@ff{#1}}}}}}}
\def\PY#1#2{\PY@reset\PY@toks#1+\relax+\PY@do{#2}}

\expandafter\def\csname PY@tok@w\endcsname{\def\PY@tc##1{\textcolor[rgb]{0.73,0.73,0.73}{##1}}}
\expandafter\def\csname PY@tok@c\endcsname{\let\PY@it=\textit\def\PY@tc##1{\textcolor[rgb]{0.25,0.50,0.50}{##1}}}
\expandafter\def\csname PY@tok@cp\endcsname{\def\PY@tc##1{\textcolor[rgb]{0.74,0.48,0.00}{##1}}}
\expandafter\def\csname PY@tok@k\endcsname{\let\PY@bf=\textbf\def\PY@tc##1{\textcolor[rgb]{0.00,0.50,0.00}{##1}}}
\expandafter\def\csname PY@tok@kp\endcsname{\def\PY@tc##1{\textcolor[rgb]{0.00,0.50,0.00}{##1}}}
\expandafter\def\csname PY@tok@kt\endcsname{\def\PY@tc##1{\textcolor[rgb]{0.69,0.00,0.25}{##1}}}
\expandafter\def\csname PY@tok@o\endcsname{\def\PY@tc##1{\textcolor[rgb]{0.40,0.40,0.40}{##1}}}
\expandafter\def\csname PY@tok@ow\endcsname{\let\PY@bf=\textbf\def\PY@tc##1{\textcolor[rgb]{0.67,0.13,1.00}{##1}}}
\expandafter\def\csname PY@tok@nb\endcsname{\def\PY@tc##1{\textcolor[rgb]{0.00,0.50,0.00}{##1}}}
\expandafter\def\csname PY@tok@nf\endcsname{\def\PY@tc##1{\textcolor[rgb]{0.00,0.00,1.00}{##1}}}
\expandafter\def\csname PY@tok@nc\endcsname{\let\PY@bf=\textbf\def\PY@tc##1{\textcolor[rgb]{0.00,0.00,1.00}{##1}}}
\expandafter\def\csname PY@tok@nn\endcsname{\let\PY@bf=\textbf\def\PY@tc##1{\textcolor[rgb]{0.00,0.00,1.00}{##1}}}
\expandafter\def\csname PY@tok@ne\endcsname{\let\PY@bf=\textbf\def\PY@tc##1{\textcolor[rgb]{0.82,0.25,0.23}{##1}}}
\expandafter\def\csname PY@tok@nv\endcsname{\def\PY@tc##1{\textcolor[rgb]{0.10,0.09,0.49}{##1}}}
\expandafter\def\csname PY@tok@no\endcsname{\def\PY@tc##1{\textcolor[rgb]{0.53,0.00,0.00}{##1}}}
\expandafter\def\csname PY@tok@nl\endcsname{\def\PY@tc##1{\textcolor[rgb]{0.63,0.63,0.00}{##1}}}
\expandafter\def\csname PY@tok@ni\endcsname{\let\PY@bf=\textbf\def\PY@tc##1{\textcolor[rgb]{0.60,0.60,0.60}{##1}}}
\expandafter\def\csname PY@tok@na\endcsname{\def\PY@tc##1{\textcolor[rgb]{0.49,0.56,0.16}{##1}}}
\expandafter\def\csname PY@tok@nt\endcsname{\let\PY@bf=\textbf\def\PY@tc##1{\textcolor[rgb]{0.00,0.50,0.00}{##1}}}
\expandafter\def\csname PY@tok@nd\endcsname{\def\PY@tc##1{\textcolor[rgb]{0.67,0.13,1.00}{##1}}}
\expandafter\def\csname PY@tok@s\endcsname{\def\PY@tc##1{\textcolor[rgb]{0.73,0.13,0.13}{##1}}}
\expandafter\def\csname PY@tok@sd\endcsname{\let\PY@it=\textit\def\PY@tc##1{\textcolor[rgb]{0.73,0.13,0.13}{##1}}}
\expandafter\def\csname PY@tok@si\endcsname{\let\PY@bf=\textbf\def\PY@tc##1{\textcolor[rgb]{0.73,0.40,0.53}{##1}}}
\expandafter\def\csname PY@tok@se\endcsname{\let\PY@bf=\textbf\def\PY@tc##1{\textcolor[rgb]{0.73,0.40,0.13}{##1}}}
\expandafter\def\csname PY@tok@sr\endcsname{\def\PY@tc##1{\textcolor[rgb]{0.73,0.40,0.53}{##1}}}
\expandafter\def\csname PY@tok@ss\endcsname{\def\PY@tc##1{\textcolor[rgb]{0.10,0.09,0.49}{##1}}}
\expandafter\def\csname PY@tok@sx\endcsname{\def\PY@tc##1{\textcolor[rgb]{0.00,0.50,0.00}{##1}}}
\expandafter\def\csname PY@tok@m\endcsname{\def\PY@tc##1{\textcolor[rgb]{0.40,0.40,0.40}{##1}}}
\expandafter\def\csname PY@tok@gh\endcsname{\let\PY@bf=\textbf\def\PY@tc##1{\textcolor[rgb]{0.00,0.00,0.50}{##1}}}
\expandafter\def\csname PY@tok@gu\endcsname{\let\PY@bf=\textbf\def\PY@tc##1{\textcolor[rgb]{0.50,0.00,0.50}{##1}}}
\expandafter\def\csname PY@tok@gd\endcsname{\def\PY@tc##1{\textcolor[rgb]{0.63,0.00,0.00}{##1}}}
\expandafter\def\csname PY@tok@gi\endcsname{\def\PY@tc##1{\textcolor[rgb]{0.00,0.63,0.00}{##1}}}
\expandafter\def\csname PY@tok@gr\endcsname{\def\PY@tc##1{\textcolor[rgb]{1.00,0.00,0.00}{##1}}}
\expandafter\def\csname PY@tok@ge\endcsname{\let\PY@it=\textit}
\expandafter\def\csname PY@tok@gs\endcsname{\let\PY@bf=\textbf}
\expandafter\def\csname PY@tok@gp\endcsname{\let\PY@bf=\textbf\def\PY@tc##1{\textcolor[rgb]{0.00,0.00,0.50}{##1}}}
\expandafter\def\csname PY@tok@go\endcsname{\def\PY@tc##1{\textcolor[rgb]{0.53,0.53,0.53}{##1}}}
\expandafter\def\csname PY@tok@gt\endcsname{\def\PY@tc##1{\textcolor[rgb]{0.00,0.27,0.87}{##1}}}
\expandafter\def\csname PY@tok@err\endcsname{\def\PY@bc##1{\setlength{\fboxsep}{0pt}\fcolorbox[rgb]{1.00,0.00,0.00}{1,1,1}{\strut ##1}}}
\expandafter\def\csname PY@tok@kc\endcsname{\let\PY@bf=\textbf\def\PY@tc##1{\textcolor[rgb]{0.00,0.50,0.00}{##1}}}
\expandafter\def\csname PY@tok@kd\endcsname{\let\PY@bf=\textbf\def\PY@tc##1{\textcolor[rgb]{0.00,0.50,0.00}{##1}}}
\expandafter\def\csname PY@tok@kn\endcsname{\let\PY@bf=\textbf\def\PY@tc##1{\textcolor[rgb]{0.00,0.50,0.00}{##1}}}
\expandafter\def\csname PY@tok@kr\endcsname{\let\PY@bf=\textbf\def\PY@tc##1{\textcolor[rgb]{0.00,0.50,0.00}{##1}}}
\expandafter\def\csname PY@tok@bp\endcsname{\def\PY@tc##1{\textcolor[rgb]{0.00,0.50,0.00}{##1}}}
\expandafter\def\csname PY@tok@fm\endcsname{\def\PY@tc##1{\textcolor[rgb]{0.00,0.00,1.00}{##1}}}
\expandafter\def\csname PY@tok@vc\endcsname{\def\PY@tc##1{\textcolor[rgb]{0.10,0.09,0.49}{##1}}}
\expandafter\def\csname PY@tok@vg\endcsname{\def\PY@tc##1{\textcolor[rgb]{0.10,0.09,0.49}{##1}}}
\expandafter\def\csname PY@tok@vi\endcsname{\def\PY@tc##1{\textcolor[rgb]{0.10,0.09,0.49}{##1}}}
\expandafter\def\csname PY@tok@vm\endcsname{\def\PY@tc##1{\textcolor[rgb]{0.10,0.09,0.49}{##1}}}
\expandafter\def\csname PY@tok@sa\endcsname{\def\PY@tc##1{\textcolor[rgb]{0.73,0.13,0.13}{##1}}}
\expandafter\def\csname PY@tok@sb\endcsname{\def\PY@tc##1{\textcolor[rgb]{0.73,0.13,0.13}{##1}}}
\expandafter\def\csname PY@tok@sc\endcsname{\def\PY@tc##1{\textcolor[rgb]{0.73,0.13,0.13}{##1}}}
\expandafter\def\csname PY@tok@dl\endcsname{\def\PY@tc##1{\textcolor[rgb]{0.73,0.13,0.13}{##1}}}
\expandafter\def\csname PY@tok@s2\endcsname{\def\PY@tc##1{\textcolor[rgb]{0.73,0.13,0.13}{##1}}}
\expandafter\def\csname PY@tok@sh\endcsname{\def\PY@tc##1{\textcolor[rgb]{0.73,0.13,0.13}{##1}}}
\expandafter\def\csname PY@tok@s1\endcsname{\def\PY@tc##1{\textcolor[rgb]{0.73,0.13,0.13}{##1}}}
\expandafter\def\csname PY@tok@mb\endcsname{\def\PY@tc##1{\textcolor[rgb]{0.40,0.40,0.40}{##1}}}
\expandafter\def\csname PY@tok@mf\endcsname{\def\PY@tc##1{\textcolor[rgb]{0.40,0.40,0.40}{##1}}}
\expandafter\def\csname PY@tok@mh\endcsname{\def\PY@tc##1{\textcolor[rgb]{0.40,0.40,0.40}{##1}}}
\expandafter\def\csname PY@tok@mi\endcsname{\def\PY@tc##1{\textcolor[rgb]{0.40,0.40,0.40}{##1}}}
\expandafter\def\csname PY@tok@il\endcsname{\def\PY@tc##1{\textcolor[rgb]{0.40,0.40,0.40}{##1}}}
\expandafter\def\csname PY@tok@mo\endcsname{\def\PY@tc##1{\textcolor[rgb]{0.40,0.40,0.40}{##1}}}
\expandafter\def\csname PY@tok@ch\endcsname{\let\PY@it=\textit\def\PY@tc##1{\textcolor[rgb]{0.25,0.50,0.50}{##1}}}
\expandafter\def\csname PY@tok@cm\endcsname{\let\PY@it=\textit\def\PY@tc##1{\textcolor[rgb]{0.25,0.50,0.50}{##1}}}
\expandafter\def\csname PY@tok@cpf\endcsname{\let\PY@it=\textit\def\PY@tc##1{\textcolor[rgb]{0.25,0.50,0.50}{##1}}}
\expandafter\def\csname PY@tok@c1\endcsname{\let\PY@it=\textit\def\PY@tc##1{\textcolor[rgb]{0.25,0.50,0.50}{##1}}}
\expandafter\def\csname PY@tok@cs\endcsname{\let\PY@it=\textit\def\PY@tc##1{\textcolor[rgb]{0.25,0.50,0.50}{##1}}}

\def\PYZbs{\char`\\}
\def\PYZus{\char`\_}
\def\PYZob{\char`\{}
\def\PYZcb{\char`\}}
\def\PYZca{\char`\^}
\def\PYZam{\char`\&}
\def\PYZlt{\char`\<}
\def\PYZgt{\char`\>}
\def\PYZsh{\char`\#}
\def\PYZpc{\char`\%}
\def\PYZdl{\char`\$}
\def\PYZhy{\char`\-}
\def\PYZsq{\char`\'}
\def\PYZdq{\char`\"}
\def\PYZti{\char`\~}
% for compatibility with earlier versions
\def\PYZat{@}
\def\PYZlb{[}
\def\PYZrb{]}
\makeatother


    % Exact colors from NB
    \definecolor{incolor}{rgb}{0.0, 0.0, 0.5}
    \definecolor{outcolor}{rgb}{0.545, 0.0, 0.0}



    
    % Prevent overflowing lines due to hard-to-break entities
    \sloppy 
    % Setup hyperref package
    \hypersetup{
      breaklinks=true,  % so long urls are correctly broken across lines
      colorlinks=true,
      urlcolor=urlcolor,
      linkcolor=linkcolor,
      citecolor=citecolor,
      }
    % Slightly bigger margins than the latex defaults
    
    \geometry{verbose,tmargin=1in,bmargin=1in,lmargin=1in,rmargin=1in}
    
    

    \begin{document}
    
    
    \maketitle
    
    

    
    \hypertarget{math-428-midterm-exam}{%
\subsection{Math 428, Midterm Exam}\label{math-428-midterm-exam}}

\hypertarget{problem-2}{%
\subsubsection{Problem \#2}\label{problem-2}}

    \begin{Verbatim}[commandchars=\\\{\}]
{\color{incolor}In [{\color{incolor}123}]:} \PY{k+kn}{import} \PY{n+nn}{numpy} \PY{k}{as} \PY{n+nn}{np}
          \PY{n}{np}\PY{o}{.}\PY{n}{set\PYZus{}printoptions}\PY{p}{(}\PY{n}{precision} \PY{o}{=} \PY{l+m+mi}{25}\PY{p}{)}
          \PY{k+kn}{import} \PY{n+nn}{matplotlib}\PY{n+nn}{.}\PY{n+nn}{pyplot} \PY{k}{as} \PY{n+nn}{plt}
          \PY{k+kn}{import} \PY{n+nn}{warnings}
          \PY{k+kn}{import} \PY{n+nn}{cmath}
          \PY{n}{warnings}\PY{o}{.}\PY{n}{filterwarnings}\PY{p}{(}\PY{l+s+s2}{\PYZdq{}}\PY{l+s+s2}{ignore}\PY{l+s+s2}{\PYZdq{}}\PY{p}{)}
          \PY{n}{colors} \PY{o}{=} \PY{p}{[}\PY{l+s+s1}{\PYZsq{}}\PY{l+s+s1}{red}\PY{l+s+s1}{\PYZsq{}}\PY{p}{,} \PY{l+s+s1}{\PYZsq{}}\PY{l+s+s1}{blue}\PY{l+s+s1}{\PYZsq{}}\PY{p}{,} \PY{l+s+s1}{\PYZsq{}}\PY{l+s+s1}{green}\PY{l+s+s1}{\PYZsq{}}\PY{p}{,} \PY{l+s+s1}{\PYZsq{}}\PY{l+s+s1}{black}\PY{l+s+s1}{\PYZsq{}}\PY{p}{,} \PY{l+s+s1}{\PYZsq{}}\PY{l+s+s1}{orange}\PY{l+s+s1}{\PYZsq{}}\PY{p}{,} \PY{l+s+s1}{\PYZsq{}}\PY{l+s+s1}{purple}\PY{l+s+s1}{\PYZsq{}}\PY{p}{]}
          \PY{n}{step\PYZus{}size} \PY{o}{=} \PY{l+m+mf}{1e\PYZhy{}2}
          
          \PY{c+c1}{\PYZsh{} Generic Function to create arbitrary plots with arbitrary subplots and functions}
          \PY{k}{def} \PY{n+nf}{create\PYZus{}plot}\PY{p}{(}\PY{n}{x}\PY{p}{,} \PY{n}{y}\PY{p}{,} \PY{n}{xLabel}\PY{o}{=}\PY{p}{[}\PY{l+s+s2}{\PYZdq{}}\PY{l+s+s2}{X\PYZhy{}Values}\PY{l+s+s2}{\PYZdq{}}\PY{p}{]}\PY{p}{,} \PY{n}{yLabel}\PY{o}{=}\PY{p}{[}\PY{l+s+s2}{\PYZdq{}}\PY{l+s+s2}{Y\PYZhy{}Values}\PY{l+s+s2}{\PYZdq{}}\PY{p}{]}\PY{p}{,}
                          \PY{n}{title}\PY{o}{=}\PY{p}{[}\PY{l+s+s2}{\PYZdq{}}\PY{l+s+s2}{Plot}\PY{l+s+s2}{\PYZdq{}}\PY{p}{]}\PY{p}{,} \PY{n}{num\PYZus{}rows}\PY{o}{=}\PY{l+m+mi}{1}\PY{p}{,} \PY{n}{size}\PY{o}{=}\PY{p}{(}\PY{l+m+mi}{18}\PY{p}{,} \PY{l+m+mi}{14}\PY{p}{)}\PY{p}{,} \PY{n}{y\PYZus{}lim}\PY{o}{=}\PY{p}{[}\PY{k+kc}{None}\PY{p}{,} \PY{k+kc}{None}\PY{p}{]}\PY{p}{,} \PY{n}{roots}\PY{o}{=}\PY{k+kc}{None}\PY{p}{)}\PY{p}{:}
              \PY{n}{plt}\PY{o}{.}\PY{n}{figure}\PY{p}{(}\PY{n}{figsize}\PY{o}{=}\PY{n}{size}\PY{p}{,} \PY{n}{dpi}\PY{o}{=}\PY{l+m+mi}{300}\PY{p}{)}
              \PY{k}{for} \PY{n}{c}\PY{p}{,} \PY{p}{(}\PY{n}{x\PYZus{}vals}\PY{p}{,} \PY{n}{y\PYZus{}vals}\PY{p}{,} \PY{n}{x\PYZus{}labels}\PY{p}{,} \PY{n}{y\PYZus{}labels}\PY{p}{,} \PY{n}{titles}\PY{p}{)} \PY{o+ow}{in} \PY{n+nb}{enumerate}\PY{p}{(}
                  \PY{n+nb}{zip}\PY{p}{(}\PY{n}{x}\PY{p}{,} \PY{n}{y}\PY{p}{,} \PY{n}{xLabel}\PY{p}{,} \PY{n}{yLabel}\PY{p}{,} \PY{n}{title}\PY{p}{)}\PY{p}{)}\PY{p}{:}
                  \PY{k}{for} \PY{n}{c2}\PY{p}{,} \PY{p}{(}\PY{n}{y\PYZus{}v}\PY{p}{,} \PY{n}{t}\PY{p}{)} \PY{o+ow}{in} \PY{n+nb}{enumerate}\PY{p}{(}\PY{n+nb}{zip}\PY{p}{(}\PY{n}{y\PYZus{}vals}\PY{p}{,} \PY{n}{titles}\PY{p}{)}\PY{p}{)}\PY{p}{:}
                      \PY{n}{plt}\PY{o}{.}\PY{n}{subplot}\PY{p}{(}\PY{n}{num\PYZus{}rows}\PY{p}{,} \PY{l+m+mi}{1}\PY{p}{,} \PY{n}{c} \PY{o}{+} \PY{l+m+mi}{1}\PY{p}{)}
                      \PY{n}{plt}\PY{o}{.}\PY{n}{ylim}\PY{p}{(}\PY{n}{y\PYZus{}lim}\PY{p}{[}\PY{l+m+mi}{0}\PY{p}{]}\PY{p}{,} \PY{n}{y\PYZus{}lim}\PY{p}{[}\PY{l+m+mi}{1}\PY{p}{]}\PY{p}{)} \PY{c+c1}{\PYZsh{} Apply new scaling if necessary}
                      \PY{n}{plt}\PY{o}{.}\PY{n}{autoscale}\PY{p}{(}\PY{k+kc}{True}\PY{p}{,} \PY{l+s+s1}{\PYZsq{}}\PY{l+s+s1}{y}\PY{l+s+s1}{\PYZsq{}}\PY{p}{)} \PY{c+c1}{\PYZsh{} Reset old scaling}
                      \PY{c+c1}{\PYZsh{} Add a plot to the subplot, use transparency so they can both be seen}
                      \PY{n}{plt}\PY{o}{.}\PY{n}{plot}\PY{p}{(}\PY{n}{x\PYZus{}vals}\PY{p}{,} \PY{n}{y\PYZus{}v}\PY{p}{,} \PY{n}{label}\PY{o}{=}\PY{n}{t}\PY{p}{,} \PY{n}{color}\PY{o}{=}\PY{n}{colors}\PY{p}{[}\PY{n}{c2}\PY{p}{]}\PY{p}{,} \PY{n}{alpha}\PY{o}{=}\PY{l+m+mf}{0.70}\PY{p}{)}
                      \PY{k}{if} \PY{n}{roots}\PY{p}{:}
                          \PY{n}{tmp} \PY{o}{=} \PY{p}{[}\PY{n}{root} \PY{k}{for} \PY{n}{root} \PY{o+ow}{in} \PY{n}{roots} \PY{k}{if} \PY{n}{root}\PY{o}{.}\PY{n}{imag} \PY{o}{==} \PY{l+m+mi}{0}\PY{p}{]}
                          \PY{k}{for} \PY{n}{root} \PY{o+ow}{in} \PY{n}{tmp}\PY{p}{:}
                              \PY{n}{plt}\PY{o}{.}\PY{n}{axvline}\PY{p}{(}\PY{n}{x}\PY{o}{=}\PY{n}{root}\PY{p}{)}
                      \PY{n}{plt}\PY{o}{.}\PY{n}{ylabel}\PY{p}{(}\PY{n}{y\PYZus{}labels}\PY{p}{)}
                      \PY{n}{plt}\PY{o}{.}\PY{n}{xlabel}\PY{p}{(}\PY{n}{x\PYZus{}labels}\PY{p}{)}
                      \PY{n}{plt}\PY{o}{.}\PY{n}{grid}\PY{p}{(}\PY{k+kc}{True}\PY{p}{)}
                      \PY{n}{plt}\PY{o}{.}\PY{n}{legend}\PY{p}{(}\PY{n}{loc}\PY{o}{=}\PY{l+s+s1}{\PYZsq{}}\PY{l+s+s1}{upper right}\PY{l+s+s1}{\PYZsq{}}\PY{p}{)}
              
              \PY{n}{plt}\PY{o}{.}\PY{n}{show}\PY{p}{(}\PY{p}{)}
\end{Verbatim}

    \hypertarget{function-defintion-of-gx}{%
\paragraph{\texorpdfstring{Function defintion of
\(G(x)\)}{Function defintion of G(x)}}\label{function-defintion-of-gx}}

    \begin{Verbatim}[commandchars=\\\{\}]
{\color{incolor}In [{\color{incolor}25}]:} \PY{n}{x\PYZus{}vals} \PY{o}{=} \PY{n}{np}\PY{o}{.}\PY{n}{arange}\PY{p}{(}\PY{l+m+mi}{0}\PY{p}{,} \PY{l+m+mi}{10000} \PY{o}{+} \PY{n}{step\PYZus{}size}\PY{p}{,} \PY{n}{step\PYZus{}size}\PY{p}{)}
         \PY{k}{def} \PY{n+nf}{g\PYZus{}x}\PY{p}{(}\PY{n}{x}\PY{p}{)}\PY{p}{:}
             \PY{n}{a}   \PY{o}{=} \PY{o}{\PYZhy{}}\PY{l+m+mf}{1.230e0}
             \PY{n}{b}   \PY{o}{=} \PY{l+m+mf}{3.686e\PYZhy{}4}
             \PY{n}{c\PYZus{}1} \PY{o}{=} \PY{l+m+mf}{3.658e4}
             \PY{n}{c\PYZus{}2} \PY{o}{=} \PY{l+m+mf}{2.387e3}
             \PY{n}{d}   \PY{o}{=} \PY{l+m+mf}{3.742e3}
             \PY{n}{e}   \PY{o}{=} \PY{l+m+mf}{2.232e4}
             \PY{n}{f}   \PY{o}{=} \PY{l+m+mf}{9.303e2}
             
             \PY{k}{return} \PY{p}{(}\PY{n}{e}\PY{o}{/}\PY{p}{(}\PY{n}{x}\PY{o}{\PYZhy{}}\PY{p}{(}\PY{n}{x}\PY{o}{\PYZhy{}}\PY{n}{c\PYZus{}1}\PY{p}{)} \PY{o}{/} \PY{n}{d}\PY{o}{*}\PY{n}{x}\PY{o}{*}\PY{p}{(}\PY{n}{a}\PY{o}{+}\PY{n}{b}\PY{o}{*}\PY{n}{x}\PY{p}{)}\PY{p}{)}\PY{p}{)} \PY{o}{+} \PY{p}{(}\PY{n}{f}\PY{o}{/}\PY{p}{(}\PY{n}{x}\PY{o}{\PYZhy{}}\PY{p}{(}\PY{n}{x}\PY{o}{\PYZhy{}}\PY{n}{c\PYZus{}2}\PY{p}{)}\PY{o}{/}\PY{n}{d}\PY{o}{*}\PY{n}{x}\PY{o}{*}\PY{p}{(}\PY{n}{a}\PY{o}{+}\PY{n}{b}\PY{o}{*}\PY{n}{x}\PY{p}{)}\PY{p}{)}\PY{o}{\PYZhy{}}\PY{l+m+mi}{1}\PY{p}{)}
             
             \PY{k}{return} \PY{n}{pt\PYZus{}1} \PY{o}{+} \PY{n}{pt\PYZus{}2} \PY{o}{\PYZhy{}}\PY{l+m+mi}{1}
\end{Verbatim}

    \hypertarget{question-1}{%
\subsubsection{Question \#1}\label{question-1}}

The following plot has vertical lines where the asymptotes are. Although
they \emph{visually} cross the x-axis, they are not actually roots and
should thus be ignored. I found the roots of this function using
WolframAlpha's Mathematica package, and simply filtered out all the
values that were not in the range \(0 < x < 10000\).

    \begin{Verbatim}[commandchars=\\\{\}]
{\color{incolor}In [{\color{incolor}26}]:} \PY{n}{roots} \PY{o}{=} \PY{p}{[}\PY{l+m+mf}{283.225}\PY{p}{,} \PY{l+m+mf}{5262.66}\PY{p}{,} \PY{l+m+mf}{5323.8}\PY{p}{]} \PY{c+c1}{\PYZsh{} Found using WolframAlpha\PYZsq{}s Mathematica}
         \PY{n}{create\PYZus{}plot}\PY{p}{(}\PY{p}{[}\PY{n}{x\PYZus{}vals}\PY{p}{]}\PY{p}{,} \PY{p}{[}\PY{p}{(}\PY{n}{g\PYZus{}x}\PY{p}{(}\PY{n}{x\PYZus{}vals}\PY{p}{)}\PY{p}{,} \PY{p}{)}\PY{p}{]}\PY{p}{,} \PY{p}{[}\PY{l+s+s2}{\PYZdq{}}\PY{l+s+s2}{x}\PY{l+s+s2}{\PYZdq{}}\PY{p}{]}\PY{p}{,}
                     \PY{p}{[}\PY{l+s+s2}{\PYZdq{}}\PY{l+s+s2}{\PYZdl{}G(x)\PYZdl{}}\PY{l+s+s2}{\PYZdq{}}\PY{p}{]}\PY{p}{,} \PY{p}{[}\PY{p}{(}\PY{l+s+s2}{\PYZdq{}}\PY{l+s+s2}{Plot of \PYZdl{}G(x)\PYZdl{}}\PY{l+s+s2}{\PYZdq{}}\PY{p}{,} \PY{p}{)}\PY{p}{]}\PY{p}{,} \PY{l+m+mi}{1}\PY{p}{,}
                     \PY{n}{y\PYZus{}lim}\PY{o}{=}\PY{p}{[}\PY{o}{\PYZhy{}}\PY{l+m+mi}{50}\PY{p}{,} \PY{l+m+mi}{50}\PY{p}{]}\PY{p}{,} \PY{n}{roots}\PY{o}{=}\PY{n}{roots}\PY{p}{)}
\end{Verbatim}

    \begin{center}
    \adjustimage{max size={0.9\linewidth}{0.9\paperheight}}{output_5_0.png}
    \end{center}
    { \hspace*{\fill} \\}
    
    \hypertarget{question-2}{%
\subsubsection{Question \#2}\label{question-2}}

I'll be using the modified secant method for this problem. My reasoning
for this was more a process of elimination than anything else. In order
to implement the Newton-Rhapson method, the derivative of the function
must be computed, and rather than relying on either auxillary software
or my hand-calculations to compute that (as it is fairly complicated), I
decided to not consider the Newton-Rhapson method. Left with either the
secant or modified secant method, I chose the modified secant method
because it should produce a much better approximation of the derivative
of the function, and is relatively easy to implement (and, as a bonus, I
don't need to provide two initial guesses).

    \hypertarget{question-3}{%
\subsubsection{Question \#3}\label{question-3}}

\hypertarget{implementation-of-the-modified-secant-method-of-root-finding}{%
\paragraph{Implementation of the modified secant method of root
finding}\label{implementation-of-the-modified-secant-method-of-root-finding}}

    \begin{Verbatim}[commandchars=\\\{\}]
{\color{incolor}In [{\color{incolor}66}]:} \PY{k}{def} \PY{n+nf}{modified\PYZus{}secant}\PY{p}{(}\PY{n}{func}\PY{p}{,} \PY{n}{x\PYZus{}initial}\PY{p}{,} \PY{n}{delta}\PY{o}{=}\PY{l+m+mf}{0.01}\PY{p}{,} \PY{n}{max\PYZus{}rel\PYZus{}error}\PY{o}{=}\PY{l+m+mf}{0.00001}\PY{p}{,} \PY{n}{max\PYZus{}iters}\PY{o}{=}\PY{l+m+mi}{1000}\PY{p}{)}\PY{p}{:}
             \PY{n}{x\PYZus{}vals}  \PY{o}{=} \PY{p}{[}\PY{n}{x\PYZus{}initial}\PY{p}{]} \PY{c+c1}{\PYZsh{} List of the approximations of the root}
             \PY{n}{rel\PYZus{}err} \PY{o}{=} \PY{p}{[}\PY{l+m+mi}{1}\PY{p}{]} \PY{c+c1}{\PYZsh{} The relative error of the approximation}
             \PY{n}{iters}   \PY{o}{=} \PY{l+m+mi}{0} \PY{c+c1}{\PYZsh{} Number of iterations}
             
             \PY{c+c1}{\PYZsh{} While the program hasn\PYZsq{}t looped too much, and the relative error isn\PYZsq{}t small enough}
             \PY{k}{while} \PY{n}{iters} \PY{o}{\PYZlt{}} \PY{n}{max\PYZus{}iters} \PY{o+ow}{and} \PY{n}{rel\PYZus{}err}\PY{p}{[}\PY{o}{\PYZhy{}}\PY{l+m+mi}{1}\PY{p}{]} \PY{o}{\PYZgt{}}\PY{o}{=} \PY{n}{max\PYZus{}rel\PYZus{}error}\PY{p}{:}
                 \PY{n}{iters}  \PY{o}{=} \PY{n}{iters} \PY{o}{+} \PY{l+m+mi}{1}
                 
                 \PY{c+c1}{\PYZsh{} Compute the next guess of the root}
                 \PY{n}{approx\PYZus{}top} \PY{o}{=} \PY{n}{delta} \PY{o}{*} \PY{n}{x\PYZus{}vals}\PY{p}{[}\PY{o}{\PYZhy{}}\PY{l+m+mi}{1}\PY{p}{]} \PY{o}{*} \PY{n}{func}\PY{p}{(}\PY{n}{x\PYZus{}vals}\PY{p}{[}\PY{o}{\PYZhy{}}\PY{l+m+mi}{1}\PY{p}{]}\PY{p}{)}
                 \PY{n}{approx\PYZus{}bot} \PY{o}{=} \PY{n}{func}\PY{p}{(}\PY{n}{x\PYZus{}vals}\PY{p}{[}\PY{o}{\PYZhy{}}\PY{l+m+mi}{1}\PY{p}{]} \PY{o}{+} \PY{n}{delta} \PY{o}{*} \PY{n}{x\PYZus{}vals}\PY{p}{[}\PY{o}{\PYZhy{}}\PY{l+m+mi}{1}\PY{p}{]}\PY{p}{)} \PY{o}{\PYZhy{}} \PY{n}{func}\PY{p}{(}\PY{n}{x\PYZus{}vals}\PY{p}{[}\PY{o}{\PYZhy{}}\PY{l+m+mi}{1}\PY{p}{]}\PY{p}{)}
                 \PY{n}{next\PYZus{}x} \PY{o}{=} \PY{n}{x\PYZus{}vals}\PY{p}{[}\PY{o}{\PYZhy{}}\PY{l+m+mi}{1}\PY{p}{]} \PY{o}{\PYZhy{}} \PY{n}{approx\PYZus{}top} \PY{o}{/} \PY{n}{approx\PYZus{}bot}
                 \PY{n}{x\PYZus{}vals}\PY{o}{.}\PY{n}{append}\PY{p}{(}\PY{n}{next\PYZus{}x}\PY{p}{)}
                 
                  \PY{c+c1}{\PYZsh{} Compute the relative error}
                 \PY{n}{rel\PYZus{}err}\PY{o}{.}\PY{n}{append}\PY{p}{(}\PY{n+nb}{abs}\PY{p}{(}\PY{p}{(}\PY{n}{x\PYZus{}vals}\PY{p}{[}\PY{o}{\PYZhy{}}\PY{l+m+mi}{1}\PY{p}{]} \PY{o}{\PYZhy{}} \PY{n}{x\PYZus{}vals}\PY{p}{[}\PY{o}{\PYZhy{}}\PY{l+m+mi}{2}\PY{p}{]}\PY{p}{)} \PY{o}{/} \PY{n}{x\PYZus{}vals}\PY{p}{[}\PY{o}{\PYZhy{}}\PY{l+m+mi}{1}\PY{p}{]}\PY{p}{)}\PY{p}{)}
                 
             \PY{k}{return} \PY{p}{(}\PY{n}{x\PYZus{}vals}\PY{p}{,} \PY{n}{rel\PYZus{}err}\PY{p}{,} \PY{n}{iters}\PY{p}{)}
\end{Verbatim}

    \hypertarget{using-the-modified-secant-method-to-find-each-of-the-roots-of-gx}{%
\paragraph{\texorpdfstring{Using the modified secant method to find each
of the roots of
\(G(x)\)}{Using the modified secant method to find each of the roots of G(x)}}\label{using-the-modified-secant-method-to-find-each-of-the-roots-of-gx}}

    \begin{Verbatim}[commandchars=\\\{\}]
{\color{incolor}In [{\color{incolor}94}]:} \PY{c+c1}{\PYZsh{} Find the roots of the g\PYZus{}x function with the modified secant method written above}
         \PY{n}{max\PYZus{}error} \PY{o}{=} \PY{l+m+mf}{0.00001}
         \PY{n}{root\PYZus{}1}\PY{p}{,} \PY{n}{err\PYZus{}1}\PY{p}{,} \PY{n}{iters\PYZus{}1} \PY{o}{=} \PY{n}{modified\PYZus{}secant}\PY{p}{(}\PY{n}{g\PYZus{}x}\PY{p}{,} \PY{l+m+mi}{200}\PY{p}{,} \PY{l+m+mf}{0.001}\PY{p}{,} \PY{n}{max\PYZus{}error}\PY{p}{)}
         \PY{n}{root\PYZus{}2}\PY{p}{,} \PY{n}{err\PYZus{}2}\PY{p}{,} \PY{n}{iters\PYZus{}2} \PY{o}{=} \PY{n}{modified\PYZus{}secant}\PY{p}{(}\PY{n}{g\PYZus{}x}\PY{p}{,} \PY{l+m+mi}{5000}\PY{p}{,} \PY{l+m+mf}{0.001}\PY{p}{,} \PY{n}{max\PYZus{}error}\PY{p}{)}
         \PY{n}{root\PYZus{}3}\PY{p}{,} \PY{n}{err\PYZus{}3}\PY{p}{,} \PY{n}{iters\PYZus{}3} \PY{o}{=} \PY{n}{modified\PYZus{}secant}\PY{p}{(}\PY{n}{g\PYZus{}x}\PY{p}{,} \PY{l+m+mi}{6000}\PY{p}{,} \PY{l+m+mf}{0.001}\PY{p}{,} \PY{n}{max\PYZus{}error}\PY{p}{)}
         \PY{n+nb}{print} \PY{p}{(}\PY{l+s+s2}{\PYZdq{}}\PY{l+s+s2}{Roots:}\PY{l+s+se}{\PYZbs{}t}\PY{l+s+se}{\PYZbs{}t}\PY{l+s+se}{\PYZbs{}t}\PY{l+s+s2}{Relative Error:}\PY{l+s+se}{\PYZbs{}t}\PY{l+s+se}{\PYZbs{}t}\PY{l+s+se}{\PYZbs{}t}\PY{l+s+s2}{RE Less than 0.00001?}\PY{l+s+se}{\PYZbs{}n}\PY{l+s+s2}{\PYZdq{}}\PY{p}{,}
                \PY{n}{root\PYZus{}1}\PY{p}{[}\PY{o}{\PYZhy{}}\PY{l+m+mi}{1}\PY{p}{]}\PY{p}{,} \PY{l+s+s2}{\PYZdq{}}\PY{l+s+se}{\PYZbs{}t}\PY{l+s+s2}{\PYZdq{}}\PY{p}{,} \PY{n}{err\PYZus{}1}\PY{p}{[}\PY{o}{\PYZhy{}}\PY{l+m+mi}{1}\PY{p}{]}\PY{p}{,} \PY{l+s+s2}{\PYZdq{}}\PY{l+s+se}{\PYZbs{}t}\PY{l+s+s2}{\PYZdq{}}\PY{p}{,} \PY{p}{(}\PY{n}{err\PYZus{}1}\PY{p}{[}\PY{o}{\PYZhy{}}\PY{l+m+mi}{1}\PY{p}{]} \PY{o}{\PYZlt{}} \PY{n}{max\PYZus{}error}\PY{p}{)}\PY{p}{,} \PY{l+s+s2}{\PYZdq{}}\PY{l+s+se}{\PYZbs{}n}\PY{l+s+s2}{\PYZdq{}}\PY{p}{,}
                \PY{n}{root\PYZus{}2}\PY{p}{[}\PY{o}{\PYZhy{}}\PY{l+m+mi}{1}\PY{p}{]}\PY{p}{,} \PY{l+s+s2}{\PYZdq{}}\PY{l+s+se}{\PYZbs{}t}\PY{l+s+s2}{\PYZdq{}}\PY{p}{,} \PY{n}{err\PYZus{}2}\PY{p}{[}\PY{o}{\PYZhy{}}\PY{l+m+mi}{1}\PY{p}{]}\PY{p}{,} \PY{l+s+s2}{\PYZdq{}}\PY{l+s+se}{\PYZbs{}t}\PY{l+s+se}{\PYZbs{}t}\PY{l+s+s2}{\PYZdq{}}\PY{p}{,} \PY{p}{(}\PY{n}{err\PYZus{}1}\PY{p}{[}\PY{o}{\PYZhy{}}\PY{l+m+mi}{1}\PY{p}{]} \PY{o}{\PYZlt{}} \PY{n}{max\PYZus{}error}\PY{p}{)}\PY{p}{,} \PY{l+s+s2}{\PYZdq{}}\PY{l+s+se}{\PYZbs{}n}\PY{l+s+s2}{\PYZdq{}}\PY{p}{,}
                \PY{n}{root\PYZus{}3}\PY{p}{[}\PY{o}{\PYZhy{}}\PY{l+m+mi}{1}\PY{p}{]}\PY{p}{,} \PY{l+s+s2}{\PYZdq{}}\PY{l+s+se}{\PYZbs{}t}\PY{l+s+s2}{\PYZdq{}}\PY{p}{,} \PY{n}{err\PYZus{}3}\PY{p}{[}\PY{o}{\PYZhy{}}\PY{l+m+mi}{1}\PY{p}{]}\PY{p}{,} \PY{l+s+s2}{\PYZdq{}}\PY{l+s+se}{\PYZbs{}t}\PY{l+s+s2}{\PYZdq{}}\PY{p}{,} \PY{p}{(}\PY{n}{err\PYZus{}1}\PY{p}{[}\PY{o}{\PYZhy{}}\PY{l+m+mi}{1}\PY{p}{]} \PY{o}{\PYZlt{}} \PY{n}{max\PYZus{}error}\PY{p}{)}\PY{p}{)}
\end{Verbatim}

    \begin{Verbatim}[commandchars=\\\{\}]
Roots:			Relative Error:		RE Less than 0.00001?
 283.22494920685557 	1.8102662618770673e-07	   True 
 5262.659211321554 	 7.538805337406939e-06	    True 
 5323.800497434899 	 1.3857952864025386e-06	   True

    \end{Verbatim}

    Clearly, my modified secant method of root finding was able to find all
the real roots of the function in the given range, and for each root the
relative error is less than 0.00001 (as shown in the third column).
\#\#\# Question \#4 \#\#\#\# How poor initial guesses can fail to find
the smallest root of \(G(x)\)

    \begin{Verbatim}[commandchars=\\\{\}]
{\color{incolor}In [{\color{incolor}131}]:} \PY{n}{initials} \PY{o}{=} \PY{p}{[}\PY{p}{]} \PY{c+c1}{\PYZsh{} List of initial guesses}
          \PY{n}{roots} \PY{o}{=} \PY{p}{[}\PY{p}{]} \PY{c+c1}{\PYZsh{} List of the resulting root estimation using the modified secant method}
          \PY{c+c1}{\PYZsh{} Calculate the estimated root for guesses between 0 and 3000 in increments of 50}
          \PY{k}{for} \PY{n}{initial} \PY{o+ow}{in} \PY{n}{np}\PY{o}{.}\PY{n}{arange}\PY{p}{(}\PY{l+m+mi}{0}\PY{p}{,} \PY{l+m+mi}{3000}\PY{p}{,} \PY{l+m+mi}{10}\PY{p}{)}\PY{p}{:}
              \PY{n}{initials}\PY{o}{.}\PY{n}{append}\PY{p}{(}\PY{n}{initial}\PY{p}{)}
              \PY{n}{roots}\PY{o}{.}\PY{n}{append}\PY{p}{(}\PY{n}{modified\PYZus{}secant}\PY{p}{(}\PY{n}{g\PYZus{}x}\PY{p}{,} \PY{n}{initial}\PY{p}{,} \PY{l+m+mf}{0.001}\PY{p}{,} \PY{n}{max\PYZus{}error}\PY{p}{)}\PY{p}{[}\PY{l+m+mi}{0}\PY{p}{]}\PY{p}{[}\PY{o}{\PYZhy{}}\PY{l+m+mi}{1}\PY{p}{]}\PY{p}{)}
              \PY{k}{if} \PY{o+ow}{not} \PY{n}{np}\PY{o}{.}\PY{n}{isfinite}\PY{p}{(}\PY{n}{roots}\PY{p}{[}\PY{o}{\PYZhy{}}\PY{l+m+mi}{1}\PY{p}{]}\PY{p}{)}\PY{p}{:} \PY{c+c1}{\PYZsh{} Remove all errored values}
                  \PY{k}{del} \PY{n}{initials}\PY{p}{[}\PY{o}{\PYZhy{}}\PY{l+m+mi}{1}\PY{p}{]}
                  \PY{k}{del} \PY{n}{roots}\PY{p}{[}\PY{o}{\PYZhy{}}\PY{l+m+mi}{1}\PY{p}{]}
            
          \PY{n}{create\PYZus{}plot}\PY{p}{(}\PY{p}{[}\PY{n}{initials}\PY{p}{]}\PY{p}{,} \PY{p}{[}\PY{p}{(}\PY{n}{roots}\PY{p}{,} \PY{p}{)}\PY{p}{]}\PY{p}{,} \PY{p}{[}\PY{l+s+s2}{\PYZdq{}}\PY{l+s+s2}{Initial Guess}\PY{l+s+s2}{\PYZdq{}}\PY{p}{]}\PY{p}{,} \PY{p}{[}\PY{l+s+s2}{\PYZdq{}}\PY{l+s+s2}{Estimated Root of \PYZdl{}G(x)\PYZdl{}}\PY{l+s+s2}{\PYZdq{}}\PY{p}{]}\PY{p}{,}
                      \PY{p}{[}\PY{p}{(}\PY{l+s+s2}{\PYZdq{}}\PY{l+s+s2}{Root Estimation of \PYZdl{}G(x)\PYZdl{}}\PY{l+s+s2}{\PYZdq{}}\PY{p}{,} \PY{p}{)}\PY{p}{]}\PY{p}{,} \PY{n}{num\PYZus{}rows}\PY{o}{=}\PY{l+m+mi}{1}\PY{p}{,} \PY{n}{size}\PY{o}{=}\PY{p}{(}\PY{l+m+mi}{9}\PY{p}{,} \PY{l+m+mi}{6}\PY{p}{)}\PY{p}{)}
\end{Verbatim}

    \begin{center}
    \adjustimage{max size={0.9\linewidth}{0.9\paperheight}}{output_12_0.png}
    \end{center}
    { \hspace*{\fill} \\}
    
    The above code and attached graph show how my modified secant method
performs in finding the root of \(G(x)\) for different initial guesses.
I've plotted guesses between 0 and 3000 in increments of 10. Clearly,
for guesses below (roughly) 500, the algorithm performs as expected, and
the lowest root of 283.225 is found. But, if the initial guess is above
500, the algorithm finds the root at -1928.05, which is beyond the
bounds of the problem. This happens for many ranges of guesses (as shown
above), but clearly the method behaves erratically at values far from
the first root, but not near enough to the second and third root. This
happens because of the function \(G(x)\), and how the modified secant
method approximates each sequential term (\(x_i\)). Unlike bounded
methods (like bisection) the modified secant method simply uses the
approximate derivative of the function to estimate the next point /
guess of the root. As can be seen in the graph of \(G(x)\) above, for
values past the root, this can lead to the method extrapolating the
curve out beyond 0 to the negative roots, thereby skipping the first
root we actually care about. Hence, having a poor initial guess for this
first root can easily lead to divergence from the root we're trying to
find, even though it's technically closer than the negative root.


    % Add a bibliography block to the postdoc
    
    
    
    \end{document}
