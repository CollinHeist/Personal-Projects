
% Default to the notebook output style

    


% Inherit from the specified cell style.




    
\documentclass[11pt]{article}
\author{Collin Heist}
    
    
    \usepackage[T1]{fontenc}
    % Nicer default font (+ math font) than Computer Modern for most use cases
    \usepackage{mathpazo}

    % Basic figure setup, for now with no caption control since it's done
    % automatically by Pandoc (which extracts ![](path) syntax from Markdown).
    \usepackage{graphicx}
    % We will generate all images so they have a width \maxwidth. This means
    % that they will get their normal width if they fit onto the page, but
    % are scaled down if they would overflow the margins.
    \makeatletter
    \def\maxwidth{\ifdim\Gin@nat@width>\linewidth\linewidth
    \else\Gin@nat@width\fi}
    \makeatother
    \let\Oldincludegraphics\includegraphics
    % Set max figure width to be 80% of text width, for now hardcoded.
    \renewcommand{\includegraphics}[1]{\Oldincludegraphics[width=.8\maxwidth]{#1}}
    % Ensure that by default, figures have no caption (until we provide a
    % proper Figure object with a Caption API and a way to capture that
    % in the conversion process - todo).
    \usepackage{caption}
    \DeclareCaptionLabelFormat{nolabel}{}
    \captionsetup{labelformat=nolabel}

    \usepackage{adjustbox} % Used to constrain images to a maximum size 
    \usepackage{xcolor} % Allow colors to be defined
    \usepackage{enumerate} % Needed for markdown enumerations to work
    \usepackage{geometry} % Used to adjust the document margins
    \usepackage{amsmath} % Equations
    \usepackage{amssymb} % Equations
    \usepackage{textcomp} % defines textquotesingle
    % Hack from http://tex.stackexchange.com/a/47451/13684:
    \AtBeginDocument{%
        \def\PYZsq{\textquotesingle}% Upright quotes in Pygmentized code
    }
    \usepackage{upquote} % Upright quotes for verbatim code
    \usepackage{eurosym} % defines \euro
    \usepackage[mathletters]{ucs} % Extended unicode (utf-8) support
    \usepackage[utf8x]{inputenc} % Allow utf-8 characters in the tex document
    \usepackage{fancyvrb} % verbatim replacement that allows latex
    \usepackage{grffile} % extends the file name processing of package graphics 
                         % to support a larger range 
    % The hyperref package gives us a pdf with properly built
    % internal navigation ('pdf bookmarks' for the table of contents,
    % internal cross-reference links, web links for URLs, etc.)
    \usepackage{hyperref}
    \usepackage{longtable} % longtable support required by pandoc >1.10
    \usepackage{booktabs}  % table support for pandoc > 1.12.2
    \usepackage[inline]{enumitem} % IRkernel/repr support (it uses the enumerate* environment)
    \usepackage[normalem]{ulem} % ulem is needed to support strikethroughs (\sout)
                                % normalem makes italics be italics, not underlines
    \usepackage{mathrsfs}
    

    
    
    % Colors for the hyperref package
    \definecolor{urlcolor}{rgb}{0,.145,.698}
    \definecolor{linkcolor}{rgb}{.71,0.21,0.01}
    \definecolor{citecolor}{rgb}{.12,.54,.11}

    % ANSI colors
    \definecolor{ansi-black}{HTML}{3E424D}
    \definecolor{ansi-black-intense}{HTML}{282C36}
    \definecolor{ansi-red}{HTML}{E75C58}
    \definecolor{ansi-red-intense}{HTML}{B22B31}
    \definecolor{ansi-green}{HTML}{00A250}
    \definecolor{ansi-green-intense}{HTML}{007427}
    \definecolor{ansi-yellow}{HTML}{DDB62B}
    \definecolor{ansi-yellow-intense}{HTML}{B27D12}
    \definecolor{ansi-blue}{HTML}{208FFB}
    \definecolor{ansi-blue-intense}{HTML}{0065CA}
    \definecolor{ansi-magenta}{HTML}{D160C4}
    \definecolor{ansi-magenta-intense}{HTML}{A03196}
    \definecolor{ansi-cyan}{HTML}{60C6C8}
    \definecolor{ansi-cyan-intense}{HTML}{258F8F}
    \definecolor{ansi-white}{HTML}{C5C1B4}
    \definecolor{ansi-white-intense}{HTML}{A1A6B2}
    \definecolor{ansi-default-inverse-fg}{HTML}{FFFFFF}
    \definecolor{ansi-default-inverse-bg}{HTML}{000000}

    % commands and environments needed by pandoc snippets
    % extracted from the output of `pandoc -s`
    \providecommand{\tightlist}{%
      \setlength{\itemsep}{0pt}\setlength{\parskip}{0pt}}
    \DefineVerbatimEnvironment{Highlighting}{Verbatim}{commandchars=\\\{\}}
    % Add ',fontsize=\small' for more characters per line
    \newenvironment{Shaded}{}{}
    \newcommand{\KeywordTok}[1]{\textcolor[rgb]{0.00,0.44,0.13}{\textbf{{#1}}}}
    \newcommand{\DataTypeTok}[1]{\textcolor[rgb]{0.56,0.13,0.00}{{#1}}}
    \newcommand{\DecValTok}[1]{\textcolor[rgb]{0.25,0.63,0.44}{{#1}}}
    \newcommand{\BaseNTok}[1]{\textcolor[rgb]{0.25,0.63,0.44}{{#1}}}
    \newcommand{\FloatTok}[1]{\textcolor[rgb]{0.25,0.63,0.44}{{#1}}}
    \newcommand{\CharTok}[1]{\textcolor[rgb]{0.25,0.44,0.63}{{#1}}}
    \newcommand{\StringTok}[1]{\textcolor[rgb]{0.25,0.44,0.63}{{#1}}}
    \newcommand{\CommentTok}[1]{\textcolor[rgb]{0.38,0.63,0.69}{\textit{{#1}}}}
    \newcommand{\OtherTok}[1]{\textcolor[rgb]{0.00,0.44,0.13}{{#1}}}
    \newcommand{\AlertTok}[1]{\textcolor[rgb]{1.00,0.00,0.00}{\textbf{{#1}}}}
    \newcommand{\FunctionTok}[1]{\textcolor[rgb]{0.02,0.16,0.49}{{#1}}}
    \newcommand{\RegionMarkerTok}[1]{{#1}}
    \newcommand{\ErrorTok}[1]{\textcolor[rgb]{1.00,0.00,0.00}{\textbf{{#1}}}}
    \newcommand{\NormalTok}[1]{{#1}}
    
    % Additional commands for more recent versions of Pandoc
    \newcommand{\ConstantTok}[1]{\textcolor[rgb]{0.53,0.00,0.00}{{#1}}}
    \newcommand{\SpecialCharTok}[1]{\textcolor[rgb]{0.25,0.44,0.63}{{#1}}}
    \newcommand{\VerbatimStringTok}[1]{\textcolor[rgb]{0.25,0.44,0.63}{{#1}}}
    \newcommand{\SpecialStringTok}[1]{\textcolor[rgb]{0.73,0.40,0.53}{{#1}}}
    \newcommand{\ImportTok}[1]{{#1}}
    \newcommand{\DocumentationTok}[1]{\textcolor[rgb]{0.73,0.13,0.13}{\textit{{#1}}}}
    \newcommand{\AnnotationTok}[1]{\textcolor[rgb]{0.38,0.63,0.69}{\textbf{\textit{{#1}}}}}
    \newcommand{\CommentVarTok}[1]{\textcolor[rgb]{0.38,0.63,0.69}{\textbf{\textit{{#1}}}}}
    \newcommand{\VariableTok}[1]{\textcolor[rgb]{0.10,0.09,0.49}{{#1}}}
    \newcommand{\ControlFlowTok}[1]{\textcolor[rgb]{0.00,0.44,0.13}{\textbf{{#1}}}}
    \newcommand{\OperatorTok}[1]{\textcolor[rgb]{0.40,0.40,0.40}{{#1}}}
    \newcommand{\BuiltInTok}[1]{{#1}}
    \newcommand{\ExtensionTok}[1]{{#1}}
    \newcommand{\PreprocessorTok}[1]{\textcolor[rgb]{0.74,0.48,0.00}{{#1}}}
    \newcommand{\AttributeTok}[1]{\textcolor[rgb]{0.49,0.56,0.16}{{#1}}}
    \newcommand{\InformationTok}[1]{\textcolor[rgb]{0.38,0.63,0.69}{\textbf{\textit{{#1}}}}}
    \newcommand{\WarningTok}[1]{\textcolor[rgb]{0.38,0.63,0.69}{\textbf{\textit{{#1}}}}}
    
    
    % Define a nice break command that doesn't care if a line doesn't already
    % exist.
    \def\br{\hspace*{\fill} \\* }
    % Math Jax compatibility definitions
    \def\gt{>}
    \def\lt{<}
    \let\Oldtex\TeX
    \let\Oldlatex\LaTeX
    \renewcommand{\TeX}{\textrm{\Oldtex}}
    \renewcommand{\LaTeX}{\textrm{\Oldlatex}}
    % Document parameters
    % Document title
    \title{Math 428 - Final Exam}
    
    
    
    
    

    % Pygments definitions
    
\makeatletter
\def\PY@reset{\let\PY@it=\relax \let\PY@bf=\relax%
    \let\PY@ul=\relax \let\PY@tc=\relax%
    \let\PY@bc=\relax \let\PY@ff=\relax}
\def\PY@tok#1{\csname PY@tok@#1\endcsname}
\def\PY@toks#1+{\ifx\relax#1\empty\else%
    \PY@tok{#1}\expandafter\PY@toks\fi}
\def\PY@do#1{\PY@bc{\PY@tc{\PY@ul{%
    \PY@it{\PY@bf{\PY@ff{#1}}}}}}}
\def\PY#1#2{\PY@reset\PY@toks#1+\relax+\PY@do{#2}}

\expandafter\def\csname PY@tok@w\endcsname{\def\PY@tc##1{\textcolor[rgb]{0.73,0.73,0.73}{##1}}}
\expandafter\def\csname PY@tok@c\endcsname{\let\PY@it=\textit\def\PY@tc##1{\textcolor[rgb]{0.25,0.50,0.50}{##1}}}
\expandafter\def\csname PY@tok@cp\endcsname{\def\PY@tc##1{\textcolor[rgb]{0.74,0.48,0.00}{##1}}}
\expandafter\def\csname PY@tok@k\endcsname{\let\PY@bf=\textbf\def\PY@tc##1{\textcolor[rgb]{0.00,0.50,0.00}{##1}}}
\expandafter\def\csname PY@tok@kp\endcsname{\def\PY@tc##1{\textcolor[rgb]{0.00,0.50,0.00}{##1}}}
\expandafter\def\csname PY@tok@kt\endcsname{\def\PY@tc##1{\textcolor[rgb]{0.69,0.00,0.25}{##1}}}
\expandafter\def\csname PY@tok@o\endcsname{\def\PY@tc##1{\textcolor[rgb]{0.40,0.40,0.40}{##1}}}
\expandafter\def\csname PY@tok@ow\endcsname{\let\PY@bf=\textbf\def\PY@tc##1{\textcolor[rgb]{0.67,0.13,1.00}{##1}}}
\expandafter\def\csname PY@tok@nb\endcsname{\def\PY@tc##1{\textcolor[rgb]{0.00,0.50,0.00}{##1}}}
\expandafter\def\csname PY@tok@nf\endcsname{\def\PY@tc##1{\textcolor[rgb]{0.00,0.00,1.00}{##1}}}
\expandafter\def\csname PY@tok@nc\endcsname{\let\PY@bf=\textbf\def\PY@tc##1{\textcolor[rgb]{0.00,0.00,1.00}{##1}}}
\expandafter\def\csname PY@tok@nn\endcsname{\let\PY@bf=\textbf\def\PY@tc##1{\textcolor[rgb]{0.00,0.00,1.00}{##1}}}
\expandafter\def\csname PY@tok@ne\endcsname{\let\PY@bf=\textbf\def\PY@tc##1{\textcolor[rgb]{0.82,0.25,0.23}{##1}}}
\expandafter\def\csname PY@tok@nv\endcsname{\def\PY@tc##1{\textcolor[rgb]{0.10,0.09,0.49}{##1}}}
\expandafter\def\csname PY@tok@no\endcsname{\def\PY@tc##1{\textcolor[rgb]{0.53,0.00,0.00}{##1}}}
\expandafter\def\csname PY@tok@nl\endcsname{\def\PY@tc##1{\textcolor[rgb]{0.63,0.63,0.00}{##1}}}
\expandafter\def\csname PY@tok@ni\endcsname{\let\PY@bf=\textbf\def\PY@tc##1{\textcolor[rgb]{0.60,0.60,0.60}{##1}}}
\expandafter\def\csname PY@tok@na\endcsname{\def\PY@tc##1{\textcolor[rgb]{0.49,0.56,0.16}{##1}}}
\expandafter\def\csname PY@tok@nt\endcsname{\let\PY@bf=\textbf\def\PY@tc##1{\textcolor[rgb]{0.00,0.50,0.00}{##1}}}
\expandafter\def\csname PY@tok@nd\endcsname{\def\PY@tc##1{\textcolor[rgb]{0.67,0.13,1.00}{##1}}}
\expandafter\def\csname PY@tok@s\endcsname{\def\PY@tc##1{\textcolor[rgb]{0.73,0.13,0.13}{##1}}}
\expandafter\def\csname PY@tok@sd\endcsname{\let\PY@it=\textit\def\PY@tc##1{\textcolor[rgb]{0.73,0.13,0.13}{##1}}}
\expandafter\def\csname PY@tok@si\endcsname{\let\PY@bf=\textbf\def\PY@tc##1{\textcolor[rgb]{0.73,0.40,0.53}{##1}}}
\expandafter\def\csname PY@tok@se\endcsname{\let\PY@bf=\textbf\def\PY@tc##1{\textcolor[rgb]{0.73,0.40,0.13}{##1}}}
\expandafter\def\csname PY@tok@sr\endcsname{\def\PY@tc##1{\textcolor[rgb]{0.73,0.40,0.53}{##1}}}
\expandafter\def\csname PY@tok@ss\endcsname{\def\PY@tc##1{\textcolor[rgb]{0.10,0.09,0.49}{##1}}}
\expandafter\def\csname PY@tok@sx\endcsname{\def\PY@tc##1{\textcolor[rgb]{0.00,0.50,0.00}{##1}}}
\expandafter\def\csname PY@tok@m\endcsname{\def\PY@tc##1{\textcolor[rgb]{0.40,0.40,0.40}{##1}}}
\expandafter\def\csname PY@tok@gh\endcsname{\let\PY@bf=\textbf\def\PY@tc##1{\textcolor[rgb]{0.00,0.00,0.50}{##1}}}
\expandafter\def\csname PY@tok@gu\endcsname{\let\PY@bf=\textbf\def\PY@tc##1{\textcolor[rgb]{0.50,0.00,0.50}{##1}}}
\expandafter\def\csname PY@tok@gd\endcsname{\def\PY@tc##1{\textcolor[rgb]{0.63,0.00,0.00}{##1}}}
\expandafter\def\csname PY@tok@gi\endcsname{\def\PY@tc##1{\textcolor[rgb]{0.00,0.63,0.00}{##1}}}
\expandafter\def\csname PY@tok@gr\endcsname{\def\PY@tc##1{\textcolor[rgb]{1.00,0.00,0.00}{##1}}}
\expandafter\def\csname PY@tok@ge\endcsname{\let\PY@it=\textit}
\expandafter\def\csname PY@tok@gs\endcsname{\let\PY@bf=\textbf}
\expandafter\def\csname PY@tok@gp\endcsname{\let\PY@bf=\textbf\def\PY@tc##1{\textcolor[rgb]{0.00,0.00,0.50}{##1}}}
\expandafter\def\csname PY@tok@go\endcsname{\def\PY@tc##1{\textcolor[rgb]{0.53,0.53,0.53}{##1}}}
\expandafter\def\csname PY@tok@gt\endcsname{\def\PY@tc##1{\textcolor[rgb]{0.00,0.27,0.87}{##1}}}
\expandafter\def\csname PY@tok@err\endcsname{\def\PY@bc##1{\setlength{\fboxsep}{0pt}\fcolorbox[rgb]{1.00,0.00,0.00}{1,1,1}{\strut ##1}}}
\expandafter\def\csname PY@tok@kc\endcsname{\let\PY@bf=\textbf\def\PY@tc##1{\textcolor[rgb]{0.00,0.50,0.00}{##1}}}
\expandafter\def\csname PY@tok@kd\endcsname{\let\PY@bf=\textbf\def\PY@tc##1{\textcolor[rgb]{0.00,0.50,0.00}{##1}}}
\expandafter\def\csname PY@tok@kn\endcsname{\let\PY@bf=\textbf\def\PY@tc##1{\textcolor[rgb]{0.00,0.50,0.00}{##1}}}
\expandafter\def\csname PY@tok@kr\endcsname{\let\PY@bf=\textbf\def\PY@tc##1{\textcolor[rgb]{0.00,0.50,0.00}{##1}}}
\expandafter\def\csname PY@tok@bp\endcsname{\def\PY@tc##1{\textcolor[rgb]{0.00,0.50,0.00}{##1}}}
\expandafter\def\csname PY@tok@fm\endcsname{\def\PY@tc##1{\textcolor[rgb]{0.00,0.00,1.00}{##1}}}
\expandafter\def\csname PY@tok@vc\endcsname{\def\PY@tc##1{\textcolor[rgb]{0.10,0.09,0.49}{##1}}}
\expandafter\def\csname PY@tok@vg\endcsname{\def\PY@tc##1{\textcolor[rgb]{0.10,0.09,0.49}{##1}}}
\expandafter\def\csname PY@tok@vi\endcsname{\def\PY@tc##1{\textcolor[rgb]{0.10,0.09,0.49}{##1}}}
\expandafter\def\csname PY@tok@vm\endcsname{\def\PY@tc##1{\textcolor[rgb]{0.10,0.09,0.49}{##1}}}
\expandafter\def\csname PY@tok@sa\endcsname{\def\PY@tc##1{\textcolor[rgb]{0.73,0.13,0.13}{##1}}}
\expandafter\def\csname PY@tok@sb\endcsname{\def\PY@tc##1{\textcolor[rgb]{0.73,0.13,0.13}{##1}}}
\expandafter\def\csname PY@tok@sc\endcsname{\def\PY@tc##1{\textcolor[rgb]{0.73,0.13,0.13}{##1}}}
\expandafter\def\csname PY@tok@dl\endcsname{\def\PY@tc##1{\textcolor[rgb]{0.73,0.13,0.13}{##1}}}
\expandafter\def\csname PY@tok@s2\endcsname{\def\PY@tc##1{\textcolor[rgb]{0.73,0.13,0.13}{##1}}}
\expandafter\def\csname PY@tok@sh\endcsname{\def\PY@tc##1{\textcolor[rgb]{0.73,0.13,0.13}{##1}}}
\expandafter\def\csname PY@tok@s1\endcsname{\def\PY@tc##1{\textcolor[rgb]{0.73,0.13,0.13}{##1}}}
\expandafter\def\csname PY@tok@mb\endcsname{\def\PY@tc##1{\textcolor[rgb]{0.40,0.40,0.40}{##1}}}
\expandafter\def\csname PY@tok@mf\endcsname{\def\PY@tc##1{\textcolor[rgb]{0.40,0.40,0.40}{##1}}}
\expandafter\def\csname PY@tok@mh\endcsname{\def\PY@tc##1{\textcolor[rgb]{0.40,0.40,0.40}{##1}}}
\expandafter\def\csname PY@tok@mi\endcsname{\def\PY@tc##1{\textcolor[rgb]{0.40,0.40,0.40}{##1}}}
\expandafter\def\csname PY@tok@il\endcsname{\def\PY@tc##1{\textcolor[rgb]{0.40,0.40,0.40}{##1}}}
\expandafter\def\csname PY@tok@mo\endcsname{\def\PY@tc##1{\textcolor[rgb]{0.40,0.40,0.40}{##1}}}
\expandafter\def\csname PY@tok@ch\endcsname{\let\PY@it=\textit\def\PY@tc##1{\textcolor[rgb]{0.25,0.50,0.50}{##1}}}
\expandafter\def\csname PY@tok@cm\endcsname{\let\PY@it=\textit\def\PY@tc##1{\textcolor[rgb]{0.25,0.50,0.50}{##1}}}
\expandafter\def\csname PY@tok@cpf\endcsname{\let\PY@it=\textit\def\PY@tc##1{\textcolor[rgb]{0.25,0.50,0.50}{##1}}}
\expandafter\def\csname PY@tok@c1\endcsname{\let\PY@it=\textit\def\PY@tc##1{\textcolor[rgb]{0.25,0.50,0.50}{##1}}}
\expandafter\def\csname PY@tok@cs\endcsname{\let\PY@it=\textit\def\PY@tc##1{\textcolor[rgb]{0.25,0.50,0.50}{##1}}}

\def\PYZbs{\char`\\}
\def\PYZus{\char`\_}
\def\PYZob{\char`\{}
\def\PYZcb{\char`\}}
\def\PYZca{\char`\^}
\def\PYZam{\char`\&}
\def\PYZlt{\char`\<}
\def\PYZgt{\char`\>}
\def\PYZsh{\char`\#}
\def\PYZpc{\char`\%}
\def\PYZdl{\char`\$}
\def\PYZhy{\char`\-}
\def\PYZsq{\char`\'}
\def\PYZdq{\char`\"}
\def\PYZti{\char`\~}
% for compatibility with earlier versions
\def\PYZat{@}
\def\PYZlb{[}
\def\PYZrb{]}
\makeatother


    % Exact colors from NB
    \definecolor{incolor}{rgb}{0.0, 0.0, 0.5}
    \definecolor{outcolor}{rgb}{0.545, 0.0, 0.0}



    
    % Prevent overflowing lines due to hard-to-break entities
    \sloppy 
    % Setup hyperref package
    \hypersetup{
      breaklinks=true,  % so long urls are correctly broken across lines
      colorlinks=true,
      urlcolor=urlcolor,
      linkcolor=linkcolor,
      citecolor=citecolor,
      }
    % Slightly bigger margins than the latex defaults
    
    \geometry{verbose,tmargin=1in,bmargin=1in,lmargin=1in,rmargin=1in}
    
    

    \begin{document}
    
    
    \maketitle
    
    

    
    \hypertarget{math-428---final-exam}{%
\section{Math 428 - Final Exam}\label{math-428---final-exam}}

\hypertarget{problem-1}{%
\subsection{Problem 1}\label{problem-1}}

\hypertarget{part-a}{%
\subsubsection{Part A}\label{part-a}}

Evaluate \[\int_{0}^{4}x(1-x)e^{-x}dx\] with \(n\) applications of the
trapezoidal rule. Use \(n=5, 10, 20\).

\hypertarget{graphing-function-setup}{%
\paragraph{Graphing function \& setup}\label{graphing-function-setup}}

    \begin{Verbatim}[commandchars=\\\{\}]
{\color{incolor}In [{\color{incolor}1}]:} \PY{k+kn}{import} \PY{n+nn}{numpy} \PY{k}{as} \PY{n+nn}{np}
        \PY{k+kn}{import} \PY{n+nn}{matplotlib}\PY{n+nn}{.}\PY{n+nn}{pyplot} \PY{k}{as} \PY{n+nn}{plt}
        \PY{k+kn}{from} \PY{n+nn}{matplotlib} \PY{k}{import} \PY{n}{patches}
        \PY{k+kn}{import} \PY{n+nn}{warnings}
        \PY{n}{warnings}\PY{o}{.}\PY{n}{filterwarnings}\PY{p}{(}\PY{l+s+s2}{\PYZdq{}}\PY{l+s+s2}{ignore}\PY{l+s+s2}{\PYZdq{}}\PY{p}{)}
        \PY{n}{colors} \PY{o}{=} \PY{p}{[}\PY{l+s+s2}{\PYZdq{}}\PY{l+s+s2}{red}\PY{l+s+s2}{\PYZdq{}}\PY{p}{,} \PY{l+s+s2}{\PYZdq{}}\PY{l+s+s2}{blue}\PY{l+s+s2}{\PYZdq{}}\PY{p}{,} \PY{l+s+s2}{\PYZdq{}}\PY{l+s+s2}{green}\PY{l+s+s2}{\PYZdq{}}\PY{p}{,} \PY{l+s+s2}{\PYZdq{}}\PY{l+s+s2}{gray}\PY{l+s+s2}{\PYZdq{}}\PY{p}{,} \PY{l+s+s2}{\PYZdq{}}\PY{l+s+s2}{purple}\PY{l+s+s2}{\PYZdq{}}\PY{p}{,} \PY{l+s+s2}{\PYZdq{}}\PY{l+s+s2}{orange}\PY{l+s+s2}{\PYZdq{}}\PY{p}{]}
        \PY{c+c1}{\PYZsh{} Generic Function to create a plot}
        \PY{k}{def} \PY{n+nf}{create\PYZus{}plot}\PY{p}{(}\PY{n}{x}\PY{p}{,} \PY{n}{y}\PY{p}{,} \PY{n}{xLabel}\PY{o}{=}\PY{p}{[}\PY{l+s+s2}{\PYZdq{}}\PY{l+s+s2}{X\PYZhy{}Values}\PY{l+s+s2}{\PYZdq{}}\PY{p}{]}\PY{p}{,} \PY{n}{yLabel}\PY{o}{=}\PY{p}{[}\PY{l+s+s2}{\PYZdq{}}\PY{l+s+s2}{Y\PYZhy{}Values}\PY{l+s+s2}{\PYZdq{}}\PY{p}{]}\PY{p}{,}
                        \PY{n}{title}\PY{o}{=}\PY{p}{[}\PY{l+s+s2}{\PYZdq{}}\PY{l+s+s2}{Plot}\PY{l+s+s2}{\PYZdq{}}\PY{p}{]}\PY{p}{,} \PY{n}{num\PYZus{}rows}\PY{o}{=}\PY{l+m+mi}{1}\PY{p}{,} \PY{n}{size}\PY{o}{=}\PY{p}{(}\PY{l+m+mi}{16}\PY{p}{,} \PY{l+m+mi}{12}\PY{p}{)}\PY{p}{)}\PY{p}{:}
            \PY{n}{plt}\PY{o}{.}\PY{n}{figure}\PY{p}{(}\PY{n}{figsize}\PY{o}{=}\PY{n}{size}\PY{p}{,} \PY{n}{dpi}\PY{o}{=}\PY{l+m+mi}{300}\PY{p}{)}
            \PY{k}{for} \PY{n}{c}\PY{p}{,} \PY{p}{(}\PY{n}{x\PYZus{}vals}\PY{p}{,} \PY{n}{y\PYZus{}vals}\PY{p}{,} \PY{n}{x\PYZus{}labels}\PY{p}{,} \PY{n}{y\PYZus{}labels}\PY{p}{,} \PY{n}{titles}\PY{p}{)} \PY{o+ow}{in} \PY{n+nb}{enumerate}\PY{p}{(}
                \PY{n+nb}{zip}\PY{p}{(}\PY{n}{x}\PY{p}{,} \PY{n}{y}\PY{p}{,} \PY{n}{xLabel}\PY{p}{,} \PY{n}{yLabel}\PY{p}{,} \PY{n}{title}\PY{p}{)}\PY{p}{)}\PY{p}{:}
                \PY{k}{for} \PY{n}{c2}\PY{p}{,} \PY{p}{(}\PY{n}{y\PYZus{}v}\PY{p}{,} \PY{n}{t}\PY{p}{)} \PY{o+ow}{in} \PY{n+nb}{enumerate}\PY{p}{(}\PY{n+nb}{zip}\PY{p}{(}\PY{n}{y\PYZus{}vals}\PY{p}{,} \PY{n}{titles}\PY{p}{)}\PY{p}{)}\PY{p}{:}
                    \PY{n}{plt}\PY{o}{.}\PY{n}{subplot}\PY{p}{(}\PY{n}{num\PYZus{}rows}\PY{p}{,} \PY{l+m+mi}{1}\PY{p}{,} \PY{n}{c} \PY{o}{+} \PY{l+m+mi}{1}\PY{p}{)}
                    \PY{c+c1}{\PYZsh{} Add a plot to the subplot, use transparency so they can both be seen}
                    \PY{n}{plt}\PY{o}{.}\PY{n}{plot}\PY{p}{(}\PY{n}{x\PYZus{}vals}\PY{p}{,} \PY{n}{y\PYZus{}v}\PY{p}{,} \PY{n}{label}\PY{o}{=}\PY{n}{t}\PY{p}{,} \PY{n}{color}\PY{o}{=}\PY{n}{colors}\PY{p}{[}\PY{n}{c2}\PY{o}{+}\PY{n}{c}\PY{p}{]}\PY{p}{,} \PY{n}{alpha}\PY{o}{=}\PY{l+m+mf}{0.70}\PY{p}{)}
                    \PY{n}{plt}\PY{o}{.}\PY{n}{ylabel}\PY{p}{(}\PY{n}{y\PYZus{}labels}\PY{p}{)}
                    \PY{n}{plt}\PY{o}{.}\PY{n}{xlabel}\PY{p}{(}\PY{n}{x\PYZus{}labels}\PY{p}{)}
                    \PY{n}{plt}\PY{o}{.}\PY{n}{grid}\PY{p}{(}\PY{k+kc}{True}\PY{p}{)}
                    \PY{n}{plt}\PY{o}{.}\PY{n}{legend}\PY{p}{(}\PY{n}{loc}\PY{o}{=}\PY{l+s+s1}{\PYZsq{}}\PY{l+s+s1}{lower right}\PY{l+s+s1}{\PYZsq{}}\PY{p}{)}
            
            \PY{n}{plt}\PY{o}{.}\PY{n}{show}\PY{p}{(}\PY{p}{)}
\end{Verbatim}

    \hypertarget{trapezoidal-approximation-implementation}{%
\paragraph{Trapezoidal approximation
implementation}\label{trapezoidal-approximation-implementation}}

    \begin{Verbatim}[commandchars=\\\{\}]
{\color{incolor}In [{\color{incolor}2}]:} \PY{k}{def} \PY{n+nf}{trapezoid\PYZus{}rule}\PY{p}{(}\PY{n}{f\PYZus{}x}\PY{p}{,} \PY{n}{x\PYZus{}start}\PY{p}{,} \PY{n}{x\PYZus{}end}\PY{p}{,} \PY{n}{num\PYZus{}divs}\PY{p}{)}\PY{p}{:}
            \PY{n}{integral} \PY{o}{=} \PY{n}{f\PYZus{}x}\PY{p}{(}\PY{n}{x\PYZus{}start}\PY{p}{)} \PY{o}{+} \PY{n}{f\PYZus{}x}\PY{p}{(}\PY{n}{x\PYZus{}end}\PY{p}{)}
            \PY{n}{dx} \PY{o}{=} \PY{p}{(}\PY{n}{x\PYZus{}end} \PY{o}{\PYZhy{}} \PY{n}{x\PYZus{}start}\PY{p}{)} \PY{o}{/} \PY{n}{num\PYZus{}divs}
            \PY{k}{for} \PY{n}{i} \PY{o+ow}{in} \PY{n}{np}\PY{o}{.}\PY{n}{arange}\PY{p}{(}\PY{l+m+mi}{1}\PY{p}{,} \PY{n}{num\PYZus{}divs}\PY{p}{)}\PY{p}{:}
                \PY{n}{x\PYZus{}val} \PY{o}{=} \PY{n}{x\PYZus{}start} \PY{o}{+} \PY{n}{i} \PY{o}{*} \PY{n}{dx}
                \PY{n}{integral} \PY{o}{+}\PY{o}{=} \PY{l+m+mi}{2} \PY{o}{*} \PY{n}{f\PYZus{}x}\PY{p}{(}\PY{n}{x\PYZus{}val}\PY{p}{)}
                
            \PY{k}{return} \PY{p}{(}\PY{n}{integral} \PY{o}{*} \PY{p}{(}\PY{n}{x\PYZus{}end} \PY{o}{\PYZhy{}} \PY{n}{x\PYZus{}start}\PY{p}{)} \PY{o}{/} \PY{p}{(}\PY{l+m+mi}{2} \PY{o}{*} \PY{n}{num\PYZus{}divs}\PY{p}{)}\PY{p}{)}
\end{Verbatim}

    \hypertarget{function-defintion-of-the-integral}{%
\paragraph{Function defintion of the
integral}\label{function-defintion-of-the-integral}}

    \begin{Verbatim}[commandchars=\\\{\}]
{\color{incolor}In [{\color{incolor}3}]:} \PY{k}{def} \PY{n+nf}{f\PYZus{}x}\PY{p}{(}\PY{n}{x}\PY{p}{)}\PY{p}{:}
            \PY{k}{return} \PY{n}{x} \PY{o}{*} \PY{p}{(}\PY{l+m+mi}{1} \PY{o}{\PYZhy{}} \PY{n}{x}\PY{p}{)} \PY{o}{*} \PY{n}{np}\PY{o}{.}\PY{n}{exp}\PY{p}{(}\PY{o}{\PYZhy{}}\PY{n}{x}\PY{p}{)}
\end{Verbatim}

    \hypertarget{evaluate-with-the-three-different-number-of-steps}{%
\paragraph{Evaluate with the three different number of
steps}\label{evaluate-with-the-three-different-number-of-steps}}

    \begin{Verbatim}[commandchars=\\\{\}]
{\color{incolor}In [{\color{incolor}4}]:} \PY{n}{n5}  \PY{o}{=} \PY{n}{trapezoid\PYZus{}rule}\PY{p}{(}\PY{n}{f\PYZus{}x}\PY{p}{,} \PY{l+m+mi}{0}\PY{p}{,} \PY{l+m+mi}{4}\PY{p}{,} \PY{l+m+mi}{5}\PY{p}{)}
        \PY{n}{n10} \PY{o}{=} \PY{n}{trapezoid\PYZus{}rule}\PY{p}{(}\PY{n}{f\PYZus{}x}\PY{p}{,} \PY{l+m+mi}{0}\PY{p}{,} \PY{l+m+mi}{4}\PY{p}{,} \PY{l+m+mi}{10}\PY{p}{)}
        \PY{n}{n20} \PY{o}{=} \PY{n}{trapezoid\PYZus{}rule}\PY{p}{(}\PY{n}{f\PYZus{}x}\PY{p}{,} \PY{l+m+mi}{0}\PY{p}{,} \PY{l+m+mi}{4}\PY{p}{,} \PY{l+m+mi}{20}\PY{p}{)}
        
        \PY{n+nb}{print} \PY{p}{(}\PY{l+s+s2}{\PYZdq{}}\PY{l+s+s2}{n = 5:  }\PY{l+s+si}{\PYZpc{}.4f}\PY{l+s+se}{\PYZbs{}n}\PY{l+s+s2}{n = 10: }\PY{l+s+si}{\PYZpc{}.4f}\PY{l+s+se}{\PYZbs{}n}\PY{l+s+s2}{n = 20: }\PY{l+s+si}{\PYZpc{}.4f}\PY{l+s+s2}{\PYZdq{}} \PY{o}{\PYZpc{}} \PY{p}{(}\PY{n}{n5}\PY{p}{,} \PY{n}{n10}\PY{p}{,} \PY{n}{n20}\PY{p}{)}\PY{p}{)}
\end{Verbatim}

    \begin{Verbatim}[commandchars=\\\{\}]
n = 5:  -0.6589
n = 10: -0.6272
n = 20: -0.6184

    \end{Verbatim}

    \hypertarget{part-b}{%
\subsubsection{Part B}\label{part-b}}

\[\int_{0}^{4}x(1-x)e^{-x}dx=\int_{0}^{4}xe^{-x}-x^2e^{-x}dx\]
Performing integration by parts on the second part of the integrand:
\[u_1=x^2, dv_1=e^{-x}dx\] \[du_1=2xdx, v_1=-e^{-x}\]
\[\int_{0}^{4}xe^{-x}-x^2e^{-x}dx=\int_{0}^{4}xe^{-x}dx-((-x^2e^{-x})|_0^4-\int_{0}^{4}-2xe^{-x}dx)\]
\[\int_{0}^{4}xe^{-x}dx-((-x^2e^{-x})|_0^4-\int_{0}^{4}-2xe^{-x}dx)=\int_{0}^{4}xe^{-x}dx+(x^2e^{-x})|_0^4-\int_{0}^{4}2xe^{-x}dx\]
\[\int_{0}^{4}xe^{-x}dx+(x^2e^{-x})|_0^4-\int_{0}^{4}2xe^{-x}dx=(x^2e^{-x})|_0^4-\int_{0}^{4}xe^{-x}dx\]
Performing a \emph{second} integration by parts:
\[u_2=x, dv_2=e^{-x}dx\] \[du_2=dx, v_2=-e^{-x}\]
\[(x^2e^{-x})|_0^4-\int_{0}^{4}xe^{-x}dx=(x^2e^{-x})|_0^4-((-xe^{-x})|_0^4-\int_{0}^{4}-e^{-x}dx)\]
\[(x^2e^{-x})|_0^4-((-xe^{-x})|_0^4-\int_{0}^{4}-e^{-x}dx)=(x^2e^{-x})|_0^4+(xe^{-x})|_0^4+(e^{-x})|_0^4\]
Evaluating the bounds of each of these expressions:
\[(x^2e^{-x})|_0^4+(xe^{-x})|_0^4+(e^{-x})|_0^4=4^2e^{-4}+4e^{-4}+e^{-4}-e^0\]
\[4^2e^{-4}+4e^{-4}+e^{-4}-e^0=21e^{-4}-1\] This can be directly
computed as\ldots{} \[\int_{0}^{4}x(1-x)e^{-x}dx\approx -0.61537158...\]
This value can be compared with the trapezoidal approximation of the
integral with \(n=20\) to obtain the approximate truncation error.

\hypertarget{computation-of-the-local-truncation-error}{%
\paragraph{Computation of the local truncation
error}\label{computation-of-the-local-truncation-error}}

    \begin{Verbatim}[commandchars=\\\{\}]
{\color{incolor}In [{\color{incolor}5}]:} \PY{n+nb}{print} \PY{p}{(}\PY{l+s+s2}{\PYZdq{}}\PY{l+s+s2}{The estimated local truncation error is }\PY{l+s+si}{\PYZpc{}.4f}\PY{l+s+s2}{\PYZdq{}} \PY{o}{\PYZpc{}}
               \PY{p}{(}\PY{l+m+mi}{21}\PY{o}{*}\PY{n}{np}\PY{o}{.}\PY{n}{exp}\PY{p}{(}\PY{o}{\PYZhy{}}\PY{l+m+mi}{4}\PY{p}{)}\PY{o}{\PYZhy{}}\PY{l+m+mi}{1}\PY{o}{\PYZhy{}}\PY{n}{n20}\PY{p}{)}\PY{p}{)}
\end{Verbatim}

    \begin{Verbatim}[commandchars=\\\{\}]
The estimated local truncation error is 0.0030

    \end{Verbatim}

    \hypertarget{part-c}{%
\subsubsection{Part C}\label{part-c}}

\hypertarget{finding-a-more-accurate-approximation-of-the-integral-with-richardson-extrapolation}{%
\paragraph{Finding a more accurate approximation of the integral with
Richardson
Extrapolation}\label{finding-a-more-accurate-approximation-of-the-integral-with-richardson-extrapolation}}

Richardson extrapolation is defined as the following:
\[I\approx \frac{4}{3}I(h_2)-\frac{1}{3}I(h_1)\] This can be used with
the results of the integration from \textbf{Part A} specifically with
\(n=10\), and \(n=20\).
\[I\approx \frac{4}{3}(-0.6184)-\frac{1}{3}(-0.6272)\]
\[I\approx -0.8245+0.2061\approx -0.6155\] This value is (clearly) more
accurate than the either of the original approximations. With a local
truncation error of\ldots{}

    \begin{Verbatim}[commandchars=\\\{\}]
{\color{incolor}In [{\color{incolor}6}]:} \PY{n}{rich\PYZus{}extrap} \PY{o}{=} \PY{l+m+mi}{4}\PY{o}{/}\PY{l+m+mi}{3} \PY{o}{*} \PY{n}{n20} \PY{o}{\PYZhy{}} \PY{l+m+mi}{1}\PY{o}{/}\PY{l+m+mi}{3} \PY{o}{*} \PY{n}{n10}
        \PY{n+nb}{print} \PY{p}{(}\PY{l+s+s2}{\PYZdq{}}\PY{l+s+s2}{The new estimated local truncation error is }\PY{l+s+si}{\PYZpc{}.4f}\PY{l+s+s2}{\PYZdq{}} \PY{o}{\PYZpc{}}
               \PY{p}{(}\PY{l+m+mi}{21}\PY{o}{*}\PY{n}{np}\PY{o}{.}\PY{n}{exp}\PY{p}{(}\PY{o}{\PYZhy{}}\PY{l+m+mi}{4}\PY{p}{)}\PY{o}{\PYZhy{}}\PY{l+m+mi}{1}\PY{o}{\PYZhy{}}\PY{n}{rich\PYZus{}extrap}\PY{p}{)}\PY{p}{)}
\end{Verbatim}

    \begin{Verbatim}[commandchars=\\\{\}]
The estimated local truncation error is 0.0001

    \end{Verbatim}

    \hypertarget{problem-2}{%
\subsection{Problem 2}\label{problem-2}}

\hypertarget{part-a}{%
\subsubsection{Part A}\label{part-a}}

Solve the following ODE between the bounds of \(0\leq x\leq 6\) with
\(y(0)=1\). \[\frac{dy}{dx}=x^2-y^2\]

\hypertarget{function-implementation-of-the-euler-method-of-approximating-odes}{%
\paragraph{Function implementation of the Euler Method of approximating
ODE's}\label{function-implementation-of-the-euler-method-of-approximating-odes}}

    \begin{Verbatim}[commandchars=\\\{\}]
{\color{incolor}In [{\color{incolor}7}]:} \PY{k}{def} \PY{n+nf}{euler\PYZus{}ode}\PY{p}{(}\PY{n}{dy\PYZus{}dx}\PY{p}{,} \PY{n}{initial}\PY{o}{=}\PY{p}{[}\PY{l+m+mi}{0}\PY{p}{,} \PY{l+m+mi}{0}\PY{p}{]}\PY{p}{,} \PY{n}{x\PYZus{}bounds}\PY{o}{=}\PY{p}{[}\PY{l+m+mi}{0}\PY{p}{,} \PY{l+m+mi}{0}\PY{p}{]}\PY{p}{,} \PY{n}{h}\PY{o}{=}\PY{l+m+mf}{0.1}\PY{p}{)}\PY{p}{:}
            \PY{n}{x}\PY{p}{,} \PY{n}{y} \PY{o}{=} \PY{p}{[}\PY{n}{initial}\PY{p}{[}\PY{l+m+mi}{0}\PY{p}{]}\PY{p}{]}\PY{p}{,} \PY{p}{[}\PY{n}{initial}\PY{p}{[}\PY{l+m+mi}{1}\PY{p}{]}\PY{p}{]}
            \PY{k}{for} \PY{n}{i} \PY{o+ow}{in} \PY{n+nb}{range}\PY{p}{(}\PY{l+m+mi}{1}\PY{p}{,} \PY{n+nb}{int}\PY{p}{(}\PY{p}{(}\PY{n}{x\PYZus{}bounds}\PY{p}{[}\PY{l+m+mi}{1}\PY{p}{]} \PY{o}{\PYZhy{}} \PY{n}{x\PYZus{}bounds}\PY{p}{[}\PY{l+m+mi}{0}\PY{p}{]}\PY{p}{)} \PY{o}{/} \PY{n}{h}\PY{p}{)} \PY{o}{+} \PY{l+m+mi}{1}\PY{p}{)}\PY{p}{:}
                \PY{n}{x\PYZus{}n}\PY{p}{,} \PY{n}{y\PYZus{}n} \PY{o}{=} \PY{n}{x}\PY{p}{[}\PY{o}{\PYZhy{}}\PY{l+m+mi}{1}\PY{p}{]}\PY{p}{,} \PY{n}{y}\PY{p}{[}\PY{o}{\PYZhy{}}\PY{l+m+mi}{1}\PY{p}{]}
                \PY{n}{k1} \PY{o}{=} \PY{n}{h} \PY{o}{*} \PY{n}{dy\PYZus{}dx}\PY{p}{(}\PY{n}{x\PYZus{}n}\PY{p}{,} \PY{n}{y\PYZus{}n}\PY{p}{)}
                \PY{n}{x}\PY{o}{.}\PY{n}{append}\PY{p}{(}\PY{n}{x\PYZus{}bounds}\PY{p}{[}\PY{l+m+mi}{0}\PY{p}{]} \PY{o}{+} \PY{n}{i} \PY{o}{*} \PY{n}{h}\PY{p}{)}
                \PY{n}{y}\PY{o}{.}\PY{n}{append}\PY{p}{(}\PY{n}{y\PYZus{}n} \PY{o}{+} \PY{n}{k1}\PY{p}{)}
                
            \PY{k}{return} \PY{n}{x}\PY{p}{,} \PY{n}{y}
\end{Verbatim}

    \hypertarget{function-implementation-of-the-ordinary-differential-equation}{%
\paragraph{Function implementation of the ordinary differential
equation}\label{function-implementation-of-the-ordinary-differential-equation}}

    \begin{Verbatim}[commandchars=\\\{\}]
{\color{incolor}In [{\color{incolor}8}]:} \PY{k}{def} \PY{n+nf}{dy\PYZus{}dx}\PY{p}{(}\PY{n}{x}\PY{p}{,} \PY{n}{y}\PY{p}{)}\PY{p}{:}
            \PY{k}{return} \PY{n}{x} \PY{o}{*} \PY{n}{x} \PY{o}{\PYZhy{}} \PY{n}{y} \PY{o}{*} \PY{n}{y}
\end{Verbatim}

    \hypertarget{compute-the-euler-approximation-for-the-step-sizes-of-h0.4-and-h0.2-plot-the-results}{%
\paragraph{\texorpdfstring{Compute the Euler approximation for the step
sizes of \(h=0.4\) and \(h=0.2\), plot the
results}{Compute the Euler approximation for the step sizes of h=0.4 and h=0.2, plot the results}}\label{compute-the-euler-approximation-for-the-step-sizes-of-h0.4-and-h0.2-plot-the-results}}

    \begin{Verbatim}[commandchars=\\\{\}]
{\color{incolor}In [{\color{incolor}9}]:} \PY{n}{e\PYZus{}04\PYZus{}x}\PY{p}{,} \PY{n}{e\PYZus{}04\PYZus{}y} \PY{o}{=} \PY{n}{euler\PYZus{}ode}\PY{p}{(}\PY{n}{dy\PYZus{}dx}\PY{p}{,} \PY{n}{initial}\PY{o}{=}\PY{p}{[}\PY{l+m+mi}{0}\PY{p}{,} \PY{l+m+mi}{1}\PY{p}{]}\PY{p}{,} \PY{n}{x\PYZus{}bounds}\PY{o}{=}\PY{p}{[}\PY{l+m+mi}{0}\PY{p}{,} \PY{l+m+mi}{6}\PY{p}{]}\PY{p}{,} \PY{n}{h}\PY{o}{=}\PY{l+m+mf}{0.4}\PY{p}{)}
        \PY{n}{e\PYZus{}02\PYZus{}x}\PY{p}{,} \PY{n}{e\PYZus{}02\PYZus{}y} \PY{o}{=} \PY{n}{euler\PYZus{}ode}\PY{p}{(}\PY{n}{dy\PYZus{}dx}\PY{p}{,} \PY{n}{initial}\PY{o}{=}\PY{p}{[}\PY{l+m+mi}{0}\PY{p}{,} \PY{l+m+mi}{1}\PY{p}{]}\PY{p}{,} \PY{n}{x\PYZus{}bounds}\PY{o}{=}\PY{p}{[}\PY{l+m+mi}{0}\PY{p}{,} \PY{l+m+mi}{6}\PY{p}{]}\PY{p}{,} \PY{n}{h}\PY{o}{=}\PY{l+m+mf}{0.2}\PY{p}{)}
        
        \PY{n}{create\PYZus{}plot}\PY{p}{(}\PY{p}{[}\PY{n}{e\PYZus{}04\PYZus{}x}\PY{p}{,} \PY{n}{e\PYZus{}02\PYZus{}x}\PY{p}{]}\PY{p}{,} \PY{p}{[}\PY{p}{(}\PY{n}{e\PYZus{}04\PYZus{}y}\PY{p}{,} \PY{p}{)}\PY{p}{,} \PY{p}{(}\PY{n}{e\PYZus{}02\PYZus{}y}\PY{p}{,} \PY{p}{)}\PY{p}{]}\PY{p}{,} \PY{p}{[}\PY{l+s+s2}{\PYZdq{}}\PY{l+s+s2}{\PYZdl{}x\PYZdl{}}\PY{l+s+s2}{\PYZdq{}}\PY{p}{,} \PY{l+s+s2}{\PYZdq{}}\PY{l+s+s2}{\PYZdl{}x\PYZdl{}}\PY{l+s+s2}{\PYZdq{}}\PY{p}{]}\PY{p}{,}
                    \PY{p}{[}\PY{l+s+s2}{\PYZdq{}}\PY{l+s+s2}{\PYZdl{}y(x)\PYZdl{}}\PY{l+s+s2}{\PYZdq{}}\PY{p}{,} \PY{l+s+s2}{\PYZdq{}}\PY{l+s+s2}{\PYZdl{}y(x)\PYZdl{}}\PY{l+s+s2}{\PYZdq{}}\PY{p}{]}\PY{p}{,} \PY{p}{[}\PY{p}{(}\PY{l+s+s2}{\PYZdq{}}\PY{l+s+s2}{Euler Approximation (\PYZdl{}h=0.4\PYZdl{})}\PY{l+s+s2}{\PYZdq{}}\PY{p}{,} \PY{p}{)}\PY{p}{,}
                                           \PY{p}{(}\PY{l+s+s2}{\PYZdq{}}\PY{l+s+s2}{Euler Approximation (\PYZdl{}h=0.2\PYZdl{})}\PY{l+s+s2}{\PYZdq{}}\PY{p}{,} \PY{p}{)}\PY{p}{]}\PY{p}{,} \PY{l+m+mi}{2}\PY{p}{)}
\end{Verbatim}

    \begin{center}
    \adjustimage{max size={0.9\linewidth}{0.9\paperheight}}{output_17_0.png}
    \end{center}
    { \hspace*{\fill} \\}
    
    \hypertarget{part-b}{%
\subsubsection{Part B}\label{part-b}}

As the first plot clearly demonstrates, the step size of
\(h\approx 0.4\) is less numerically stable than the other step size.
This can be explicitly proved by looking at the maximum step size:

\[h_{max}=\frac{-2}{\frac{\partial f}{\partial y}}\]

\[h_{max}=\frac{-2}{-2y}\]

\[h_{max}=\frac{1}{y}\]

Since we are using a fixed step size of \(0.4\), the y value at which
this function becomes numerically unstable is computed as:

\[0.4=\frac{1}{y}\]

\[y=\frac{1}{0.4}=2.5\]

This is in line with the observed euler approximation from \textbf{Part
A}. At a y value above 2.5 (for this step size) the result is
increasingly numerically unstable. The reason the effect of this
instability is only prevalent at larger values of y is because the
instability is more pronounced for larger values of y.

\hypertarget{part-c}{%
\subsubsection{Part C}\label{part-c}}

\hypertarget{function-to-visualize-the-slope-field-of-the-given-differential-equation}{%
\paragraph{Function to visualize the slope field of the given
differential
equation}\label{function-to-visualize-the-slope-field-of-the-given-differential-equation}}

    \begin{Verbatim}[commandchars=\\\{\}]
{\color{incolor}In [{\color{incolor}10}]:} \PY{k}{def} \PY{n+nf}{slope\PYZus{}field}\PY{p}{(}\PY{n}{dy\PYZus{}dx}\PY{p}{,} \PY{n}{x\PYZus{}bounds}\PY{o}{=}\PY{p}{[}\PY{o}{\PYZhy{}}\PY{l+m+mi}{10}\PY{p}{,}\PY{l+m+mi}{10}\PY{p}{]}\PY{p}{,} \PY{n}{y\PYZus{}bounds}\PY{o}{=}\PY{p}{[}\PY{o}{\PYZhy{}}\PY{l+m+mi}{10}\PY{p}{,} \PY{l+m+mi}{10}\PY{p}{]}\PY{p}{,} \PY{n}{spacing}\PY{o}{=}\PY{l+m+mi}{1}\PY{p}{,} \PY{n}{sol}\PY{o}{=}\PY{k+kc}{None}\PY{p}{)}\PY{p}{:}
             \PY{n}{x} \PY{o}{=} \PY{n}{np}\PY{o}{.}\PY{n}{linspace}\PY{p}{(}\PY{n}{x\PYZus{}bounds}\PY{p}{[}\PY{l+m+mi}{0}\PY{p}{]}\PY{p}{,} \PY{n}{x\PYZus{}bounds}\PY{p}{[}\PY{l+m+mi}{1}\PY{p}{]}\PY{p}{,} \PY{p}{(}\PY{n}{x\PYZus{}bounds}\PY{p}{[}\PY{l+m+mi}{1}\PY{p}{]}\PY{o}{\PYZhy{}}\PY{n}{x\PYZus{}bounds}\PY{p}{[}\PY{l+m+mi}{0}\PY{p}{]}\PY{p}{)}\PY{o}{/}\PY{n}{spacing}\PY{p}{)}
             \PY{n}{y} \PY{o}{=} \PY{n}{np}\PY{o}{.}\PY{n}{linspace}\PY{p}{(}\PY{n}{y\PYZus{}bounds}\PY{p}{[}\PY{l+m+mi}{0}\PY{p}{]}\PY{p}{,} \PY{n}{y\PYZus{}bounds}\PY{p}{[}\PY{l+m+mi}{1}\PY{p}{]}\PY{p}{,}\PY{p}{(}\PY{n}{y\PYZus{}bounds}\PY{p}{[}\PY{l+m+mi}{1}\PY{p}{]}\PY{o}{\PYZhy{}}\PY{n}{y\PYZus{}bounds}\PY{p}{[}\PY{l+m+mi}{0}\PY{p}{]}\PY{p}{)}\PY{o}{/}\PY{n}{spacing}\PY{p}{)}
             
             \PY{n}{plt}\PY{o}{.}\PY{n}{figure}\PY{p}{(}\PY{n}{figsize}\PY{o}{=}\PY{p}{(}\PY{l+m+mi}{15}\PY{p}{,} \PY{l+m+mi}{8}\PY{p}{)}\PY{p}{,} \PY{n}{dpi}\PY{o}{=}\PY{l+m+mi}{250}\PY{p}{)}
             \PY{c+c1}{\PYZsh{} Generate the slope field}
             \PY{k}{for} \PY{n}{i} \PY{o+ow}{in} \PY{n}{x}\PY{p}{:}
                 \PY{k}{for} \PY{n}{j} \PY{o+ow}{in} \PY{n}{y}\PY{p}{:}
                     \PY{n}{slope} \PY{o}{=} \PY{n}{dy\PYZus{}dx}\PY{p}{(}\PY{n}{i}\PY{p}{,} \PY{n}{j}\PY{p}{)}
                     \PY{n}{dx} \PY{o}{=} \PY{n}{np}\PY{o}{.}\PY{n}{sqrt}\PY{p}{(}\PY{n}{spacing}\PY{o}{/}\PY{l+m+mi}{100}\PY{o}{/}\PY{p}{(}\PY{l+m+mi}{1}\PY{o}{+}\PY{p}{(}\PY{n}{slope} \PY{o}{*} \PY{n}{slope}\PY{p}{)}\PY{p}{)}\PY{p}{)}
                     \PY{n}{dy} \PY{o}{=} \PY{n}{dx} \PY{o}{*} \PY{n}{slope}
                     \PY{n}{sub\PYZus{}x} \PY{o}{=} \PY{n}{np}\PY{o}{.}\PY{n}{linspace}\PY{p}{(}\PY{n}{i}\PY{o}{\PYZhy{}}\PY{n}{dx}\PY{p}{,} \PY{n}{i}\PY{o}{+}\PY{n}{dx}\PY{p}{,} \PY{l+m+mi}{2}\PY{p}{)}
                     \PY{n}{sub\PYZus{}y} \PY{o}{=} \PY{n}{np}\PY{o}{.}\PY{n}{linspace}\PY{p}{(}\PY{n}{j}\PY{o}{\PYZhy{}}\PY{n}{dy}\PY{p}{,} \PY{n}{j}\PY{o}{+}\PY{n}{dy}\PY{p}{,} \PY{l+m+mi}{2}\PY{p}{)}
                     \PY{n}{plt}\PY{o}{.}\PY{n}{plot}\PY{p}{(}\PY{n}{sub\PYZus{}x}\PY{p}{,} \PY{n}{sub\PYZus{}y}\PY{p}{,} \PY{n}{solid\PYZus{}capstyle}\PY{o}{=}\PY{l+s+s1}{\PYZsq{}}\PY{l+s+s1}{projecting}\PY{l+s+s1}{\PYZsq{}}\PY{p}{,} \PY{n}{solid\PYZus{}joinstyle}\PY{o}{=}\PY{l+s+s1}{\PYZsq{}}\PY{l+s+s1}{bevel}\PY{l+s+s1}{\PYZsq{}}\PY{p}{)}
             
             \PY{c+c1}{\PYZsh{} Draw the solution to the provided differential equation}
             \PY{k}{if} \PY{n}{sol}\PY{p}{:}
                 \PY{n}{new\PYZus{}x} \PY{o}{=} \PY{n}{np}\PY{o}{.}\PY{n}{linspace}\PY{p}{(}\PY{n}{x\PYZus{}bounds}\PY{p}{[}\PY{l+m+mi}{0}\PY{p}{]}\PY{p}{,} \PY{n}{x\PYZus{}bounds}\PY{p}{[}\PY{l+m+mi}{1}\PY{p}{]}\PY{p}{,} \PY{p}{(}\PY{n}{x\PYZus{}bounds}\PY{p}{[}\PY{l+m+mi}{1}\PY{p}{]}\PY{o}{\PYZhy{}}\PY{n}{x\PYZus{}bounds}\PY{p}{[}\PY{l+m+mi}{0}\PY{p}{]}\PY{p}{)}\PY{o}{/}\PY{n}{spacing}\PY{o}{*}\PY{l+m+mi}{10}\PY{p}{)}
                 \PY{n}{plt}\PY{o}{.}\PY{n}{plot}\PY{p}{(}\PY{n}{new\PYZus{}x}\PY{p}{,} \PY{n}{sol}\PY{p}{(}\PY{n}{new\PYZus{}x}\PY{p}{)}\PY{p}{)}
                     
             \PY{n}{plt}\PY{o}{.}\PY{n}{grid}\PY{p}{(}\PY{k+kc}{True}\PY{p}{)}
             \PY{n}{plt}\PY{o}{.}\PY{n}{show}\PY{p}{(}\PY{p}{)}
\end{Verbatim}

    \hypertarget{draw-the-slope-field}{%
\paragraph{Draw the slope field}\label{draw-the-slope-field}}

    \begin{Verbatim}[commandchars=\\\{\}]
{\color{incolor}In [{\color{incolor}11}]:} \PY{n}{slope\PYZus{}field}\PY{p}{(}\PY{n}{dy\PYZus{}dx}\PY{p}{,} \PY{n}{x\PYZus{}bounds}\PY{o}{=}\PY{p}{[}\PY{o}{\PYZhy{}}\PY{l+m+mf}{0.2}\PY{p}{,} \PY{l+m+mf}{6.4}\PY{p}{]}\PY{p}{,} \PY{n}{y\PYZus{}bounds}\PY{o}{=}\PY{p}{[}\PY{l+m+mi}{0}\PY{p}{,} \PY{l+m+mf}{7.2}\PY{p}{]}\PY{p}{,} \PY{n}{spacing}\PY{o}{=}\PY{l+m+mf}{0.2}\PY{p}{)}
\end{Verbatim}

    \begin{center}
    \adjustimage{max size={0.9\linewidth}{0.9\paperheight}}{output_21_0.png}
    \end{center}
    { \hspace*{\fill} \\}
    
    Visually, the previously obtained solutions from \textbf{Part A} fit
very well in the above slope field. Should a line be mentally drawn
starting at point (0, 1) and following the plotted slopes, then the
solution is very clearly correct. It can also be seen that at larger y
values, the `line' where the slopes converge to goes from relatively
in-line slopes (diagonally starting at (1, 1) and intersecting (3,
\textasciitilde{}2.8)) to very sharp slopes. This is in line with the
observed numerical instability at these values.

\hypertarget{problem-3}{%
\subsection{Problem 3}\label{problem-3}}

\[\frac{d}{dx}(Ky(x)\frac{dy}{dx})+N=0\] Where y is the water depth (in
meters), x is the horizontal distance (in meters), N is the infiltration
rate (in meters per day), and K is the hydraulic conductivity of the
aquifer (meters per day). We'll be supposing \(K=1\), \(N=0.002\) we're
provided that \(y(0)=20\) and \(y(1500)=10\)

\hypertarget{part-a}{%
\subsubsection{Part A}\label{part-a}}

The given differential equation can be simplified as:
\[\frac{d}{dx}(y(x)\frac{dy(x)}{dx})=\frac{-N}{k}\]

\[y(x)\frac{d^2y(x)}{dx^2}+\frac{dy(x)}{dx}\frac{dy(x)}{dx}=\frac{-N}{k}\]

\[y(x)\frac{d^2y(x)}{dx^2}+(\frac{dy(x)}{dx})^2=\frac{-N}{k}\] Rewrite
the differential equation into the following form:
\[z(x)=\frac{dy(x)}{dx}, \frac{dz(x)}{dx}=\frac{d^2y(x)}{dx^2}\] Thus
the system can be represented by these two first-order ODEs:
\[z(x)=\frac{dy(x)}{dx}\]

\[\frac{dz(x)}{dx}=\frac{\frac{-N}{K}-z(x)^2}{y(x)}\]

\hypertarget{part-b}{%
\subsubsection{Part B}\label{part-b}}

\[y(0)=20, y'(0)=\alpha\]

\hypertarget{implementation-of-the-runge-kutta-4th-order-method}{%
\paragraph{Implementation of the Runge-Kutta 4th order
method}\label{implementation-of-the-runge-kutta-4th-order-method}}

    \begin{Verbatim}[commandchars=\\\{\}]
{\color{incolor}In [{\color{incolor}12}]:} \PY{c+c1}{\PYZsh{} Fourth\PYZhy{}order Runge\PYZhy{}Kutta method for two simultaneous ODEs}
         \PY{k}{def} \PY{n+nf}{runge\PYZus{}kutta4}\PY{p}{(}\PY{n}{dy\PYZus{}dx}\PY{p}{,} \PY{n}{dz\PYZus{}dx}\PY{p}{,} \PY{n}{initial}\PY{o}{=}\PY{p}{[}\PY{l+m+mi}{0}\PY{p}{,} \PY{l+m+mi}{0}\PY{p}{]}\PY{p}{,} \PY{n}{x\PYZus{}bounds}\PY{o}{=}\PY{p}{[}\PY{l+m+mi}{0}\PY{p}{,} \PY{l+m+mi}{0}\PY{p}{]}\PY{p}{,} \PY{n}{h}\PY{o}{=}\PY{l+m+mf}{0.1}\PY{p}{)}\PY{p}{:}
             \PY{n}{x}\PY{p}{,} \PY{n}{y} \PY{o}{=} \PY{p}{[}\PY{n}{initial}\PY{p}{[}\PY{l+m+mi}{0}\PY{p}{]}\PY{p}{]}\PY{p}{,} \PY{p}{[}\PY{n}{initial}\PY{p}{[}\PY{l+m+mi}{1}\PY{p}{]}\PY{p}{]}
             \PY{k}{for} \PY{n}{i} \PY{o+ow}{in} \PY{n+nb}{range}\PY{p}{(}\PY{l+m+mi}{1}\PY{p}{,} \PY{n+nb}{int}\PY{p}{(}\PY{p}{(}\PY{n}{x\PYZus{}bounds}\PY{p}{[}\PY{l+m+mi}{1}\PY{p}{]} \PY{o}{\PYZhy{}} \PY{n}{x\PYZus{}bounds}\PY{p}{[}\PY{l+m+mi}{0}\PY{p}{]}\PY{p}{)} \PY{o}{/} \PY{n}{h}\PY{p}{)} \PY{o}{+} \PY{l+m+mi}{1}\PY{p}{)}\PY{p}{:}
                 \PY{n}{x\PYZus{}n}\PY{p}{,} \PY{n}{y\PYZus{}n} \PY{o}{=} \PY{n}{x}\PY{p}{[}\PY{o}{\PYZhy{}}\PY{l+m+mi}{1}\PY{p}{]}\PY{p}{,} \PY{n}{y}\PY{p}{[}\PY{o}{\PYZhy{}}\PY{l+m+mi}{1}\PY{p}{]} \PY{c+c1}{\PYZsh{} t, y, z}
                 \PY{n}{k1} \PY{o}{=} \PY{n}{h} \PY{o}{*} \PY{n}{dy\PYZus{}dx}\PY{p}{(}\PY{n}{x\PYZus{}n}\PY{p}{,} \PY{n}{y\PYZus{}n}\PY{p}{)}
                 \PY{n}{m1} \PY{o}{=} \PY{n}{h} \PY{o}{*} \PY{n}{dz\PYZus{}dx}\PY{p}{(}\PY{n}{x\PYZus{}n}\PY{p}{,} \PY{n}{y\PYZus{}n}\PY{p}{)}
                 \PY{n}{k2} \PY{o}{=} \PY{n}{h} \PY{o}{*} \PY{n}{dy\PYZus{}dx}\PY{p}{(}\PY{n}{x\PYZus{}n} \PY{o}{+} \PY{n}{k1} \PY{o}{/} \PY{l+m+mi}{2}\PY{p}{,} \PY{n}{y\PYZus{}n} \PY{o}{+} \PY{n}{m1} \PY{o}{/} \PY{l+m+mi}{2}\PY{p}{)}
                 \PY{n}{m2} \PY{o}{=} \PY{n}{h} \PY{o}{*} \PY{n}{dz\PYZus{}dx}\PY{p}{(}\PY{n}{x\PYZus{}n} \PY{o}{+} \PY{n}{k1} \PY{o}{/} \PY{l+m+mi}{2}\PY{p}{,} \PY{n}{y\PYZus{}n} \PY{o}{+} \PY{n}{m1} \PY{o}{/} \PY{l+m+mi}{2}\PY{p}{)}
                 \PY{n}{k3} \PY{o}{=} \PY{n}{h} \PY{o}{*} \PY{n}{dy\PYZus{}dx}\PY{p}{(}\PY{n}{x\PYZus{}n} \PY{o}{+} \PY{n}{k2} \PY{o}{/} \PY{l+m+mi}{2}\PY{p}{,} \PY{n}{y\PYZus{}n} \PY{o}{+} \PY{n}{m2} \PY{o}{/} \PY{l+m+mi}{2}\PY{p}{)}
                 \PY{n}{m3} \PY{o}{=} \PY{n}{h} \PY{o}{*} \PY{n}{dz\PYZus{}dx}\PY{p}{(}\PY{n}{x\PYZus{}n} \PY{o}{+} \PY{n}{k2} \PY{o}{/} \PY{l+m+mi}{2}\PY{p}{,} \PY{n}{y\PYZus{}n} \PY{o}{+} \PY{n}{m2} \PY{o}{/} \PY{l+m+mi}{2}\PY{p}{)}
                 \PY{n}{k4} \PY{o}{=} \PY{n}{h} \PY{o}{*} \PY{n}{dy\PYZus{}dx}\PY{p}{(}\PY{n}{x\PYZus{}n} \PY{o}{+} \PY{n}{k3}\PY{p}{,} \PY{n}{y\PYZus{}n} \PY{o}{+} \PY{n}{m3}\PY{p}{)}
                 \PY{n}{m4} \PY{o}{=} \PY{n}{h} \PY{o}{*} \PY{n}{dz\PYZus{}dx}\PY{p}{(}\PY{n}{x\PYZus{}n} \PY{o}{+} \PY{n}{k3}\PY{p}{,} \PY{n}{y\PYZus{}n} \PY{o}{+} \PY{n}{m3}\PY{p}{)}
                 \PY{n}{x}\PY{o}{.}\PY{n}{append}\PY{p}{(}\PY{n}{x\PYZus{}n} \PY{o}{+} \PY{p}{(}\PY{n}{k1} \PY{o}{+} \PY{l+m+mi}{2} \PY{o}{*} \PY{n}{k2} \PY{o}{+} \PY{l+m+mi}{2} \PY{o}{*} \PY{n}{k3} \PY{o}{+} \PY{n}{k4}\PY{p}{)} \PY{o}{/} \PY{l+m+mi}{6}\PY{p}{)}
                 \PY{n}{y}\PY{o}{.}\PY{n}{append}\PY{p}{(}\PY{n}{y\PYZus{}n} \PY{o}{+} \PY{p}{(}\PY{n}{m1} \PY{o}{+} \PY{l+m+mi}{2} \PY{o}{*} \PY{n}{m2} \PY{o}{+} \PY{l+m+mi}{2} \PY{o}{*} \PY{n}{m3} \PY{o}{+} \PY{n}{m4}\PY{p}{)} \PY{o}{/} \PY{l+m+mi}{6}\PY{p}{)}
                 
             \PY{k}{return} \PY{n}{x}\PY{p}{,} \PY{n}{y}
\end{Verbatim}

    \hypertarget{function-implementation-of-the-two-first-order-odes}{%
\paragraph{Function implementation of the two first-order
ODE's}\label{function-implementation-of-the-two-first-order-odes}}

    \begin{Verbatim}[commandchars=\\\{\}]
{\color{incolor}In [{\color{incolor}13}]:} \PY{k}{def} \PY{n+nf}{dy\PYZus{}dx}\PY{p}{(}\PY{n}{x}\PY{p}{,} \PY{n}{y}\PY{p}{)}\PY{p}{:} \PY{c+c1}{\PYZsh{} u(x)}
             \PY{k}{return} \PY{n}{y}
         
         \PY{k}{def} \PY{n+nf}{d2y\PYZus{}dx2}\PY{p}{(}\PY{n}{x}\PY{p}{,} \PY{n}{y}\PY{p}{)}\PY{p}{:} \PY{c+c1}{\PYZsh{} u\PYZsq{}(x)}
             \PY{n}{N}\PY{p}{,} \PY{n}{K} \PY{o}{=} \PY{l+m+mf}{0.0002}\PY{p}{,} \PY{l+m+mi}{1}
             \PY{k}{if} \PY{n}{y} \PY{o}{\PYZgt{}} \PY{l+m+mf}{10e30}\PY{p}{:} \PY{c+c1}{\PYZsh{} Avoid extremely large values of y}
                 \PY{k}{return} \PY{l+m+mf}{10e30}
             \PY{k}{return} \PY{p}{(}\PY{p}{(}\PY{o}{\PYZhy{}}\PY{n}{N}\PY{o}{/}\PY{n}{K} \PY{o}{\PYZhy{}} \PY{n}{y} \PY{o}{*}\PY{o}{*} \PY{l+m+mi}{2}\PY{p}{)} \PY{o}{/} \PY{n}{x}\PY{p}{)}
\end{Verbatim}

    The above functions can be used to numerically solve the two first order
ODEs given two initial conditions, one of which would be 20 and one of
which is \(\alpha\).

\hypertarget{part-c}{%
\subsubsection{Part C}\label{part-c}}

In order to find the correct alpha value for these systems of
differential equations, I will use a binary search algorithm that
assumes the changes on the Runge-Kutta approximation of \(y(x)\) is
linearly related to the input \(\alpha\) value. This is essential for
the algorithm to work, because I will use this relationship to
incrementally adjust the \(\alpha\). If \(y(1500)\) for the current
alpha being tested is higher than 10 (the known result), then I will
decrease it by half and try again. If the \(y(1500)\) is \emph{below} 10
then it is instead increased by half. I will repeat this until 10 is
matched with three decimal places.

\hypertarget{implementation-of-the-binary-search-algorithm}{%
\paragraph{Implementation of the binary search
algorithm}\label{implementation-of-the-binary-search-algorithm}}

    \begin{Verbatim}[commandchars=\\\{\}]
{\color{incolor}In [{\color{incolor}14}]:} \PY{c+c1}{\PYZsh{} Implementation of a binary search algorithm for finding the correct alpha}
         \PY{c+c1}{\PYZsh{} The two first order ODEs are used in an RK4 method to iteratively find the}
         \PY{c+c1}{\PYZsh{} proper alpha value among the provided range. Trying to match y\PYZus{}real with the}
         \PY{c+c1}{\PYZsh{} given tolerance. This method assumes the changes to the RK estimation of }
         \PY{c+c1}{\PYZsh{} y(x) is linearly proportional to the changes on alpha}
         \PY{k}{def} \PY{n+nf}{alpha\PYZus{}binary\PYZus{}search}\PY{p}{(}\PY{n}{dy\PYZus{}dx}\PY{p}{,} \PY{n}{dz\PYZus{}dx}\PY{p}{,} \PY{n}{y\PYZus{}real}\PY{p}{,} \PY{n}{initial}\PY{o}{=}\PY{l+m+mi}{0}\PY{p}{,} \PY{n}{x\PYZus{}range}\PY{o}{=}\PY{p}{[}\PY{l+m+mi}{0}\PY{p}{,} \PY{l+m+mi}{10}\PY{p}{]}\PY{p}{,}
                                 \PY{n}{x\PYZus{}step\PYZus{}size}\PY{o}{=}\PY{l+m+mi}{1}\PY{p}{,} \PY{n}{alpha\PYZus{}range}\PY{o}{=}\PY{p}{[}\PY{l+m+mi}{0}\PY{p}{,} \PY{l+m+mi}{1}\PY{p}{]}\PY{p}{,} \PY{n}{tol}\PY{o}{=}\PY{l+m+mf}{0.001}\PY{p}{,} \PY{n}{max\PYZus{}iters}\PY{o}{=}\PY{l+m+mi}{1000}\PY{p}{)}\PY{p}{:}
             \PY{n}{diff}\PY{p}{,} \PY{n}{iters}\PY{p}{,} \PY{n}{y\PYZus{}est} \PY{o}{=} \PY{l+m+mi}{1}\PY{p}{,} \PY{l+m+mi}{0}\PY{p}{,} \PY{p}{[}\PY{o}{\PYZhy{}}\PY{n+nb}{float}\PY{p}{(}\PY{l+s+s2}{\PYZdq{}}\PY{l+s+s2}{inf}\PY{l+s+s2}{\PYZdq{}}\PY{p}{)}\PY{p}{]}
             \PY{n}{alpha} \PY{o}{=} \PY{p}{(}\PY{n}{alpha\PYZus{}range}\PY{p}{[}\PY{l+m+mi}{1}\PY{p}{]} \PY{o}{\PYZhy{}} \PY{n}{alpha\PYZus{}range}\PY{p}{[}\PY{l+m+mi}{0}\PY{p}{]}\PY{p}{)} \PY{o}{/} \PY{l+m+mi}{2}
             \PY{n}{width} \PY{o}{=} \PY{n}{alpha}
             \PY{c+c1}{\PYZsh{} Search while the result is off by enough, max iters hasn\PYZsq{}t occurred}
             \PY{c+c1}{\PYZsh{} and the estimation is below the real value}
             \PY{k}{while} \PY{k+kc}{True}\PY{p}{:}
                 \PY{n}{y0\PYZus{}dy0} \PY{o}{=} \PY{p}{[}\PY{n}{initial}\PY{p}{,} \PY{n}{alpha}\PY{p}{]} \PY{c+c1}{\PYZsh{} Array for y(0)=initial and y\PYZsq{}(0)=alpha}
                 \PY{n}{y\PYZus{}est}\PY{p}{,} \PY{n}{\PYZus{}} \PY{o}{=} \PY{n}{runge\PYZus{}kutta4}\PY{p}{(}\PY{n}{dy\PYZus{}dx}\PY{p}{,} \PY{n}{dz\PYZus{}dx}\PY{p}{,} \PY{n}{y0\PYZus{}dy0}\PY{p}{,} \PY{n}{x\PYZus{}range}\PY{p}{,} \PY{n}{x\PYZus{}step\PYZus{}size}\PY{p}{)}
                 \PY{n}{diff} \PY{o}{=} \PY{n+nb}{abs}\PY{p}{(}\PY{n}{y\PYZus{}real} \PY{o}{\PYZhy{}} \PY{n}{y\PYZus{}est}\PY{p}{[}\PY{o}{\PYZhy{}}\PY{l+m+mi}{1}\PY{p}{]}\PY{p}{)} \PY{c+c1}{\PYZsh{} How far off the estimation is}
                 \PY{n+nb}{print} \PY{p}{(}\PY{l+s+s2}{\PYZdq{}}\PY{l+s+s2}{Alpha: }\PY{l+s+si}{\PYZpc{}9.6f}\PY{l+s+s2}{,}\PY{l+s+se}{\PYZbs{}t}\PY{l+s+s2}{y(1500): }\PY{l+s+si}{\PYZpc{}.5f}\PY{l+s+s2}{\PYZdq{}} \PY{o}{\PYZpc{}} \PY{p}{(}\PY{n}{alpha}\PY{p}{,} \PY{n}{y\PYZus{}est}\PY{p}{[}\PY{o}{\PYZhy{}}\PY{l+m+mi}{1}\PY{p}{]}\PY{p}{)}\PY{p}{)}
                 \PY{c+c1}{\PYZsh{} Check if the new estimate fits the exit conditions}
                 \PY{k}{if} \PY{p}{(}\PY{n}{diff} \PY{o}{\PYZlt{}} \PY{n}{tol} \PY{o+ow}{or} \PY{n}{iters} \PY{o}{\PYZgt{}} \PY{n}{max\PYZus{}iters}\PY{p}{)} \PY{o+ow}{and} \PY{p}{(}\PY{n}{y\PYZus{}est}\PY{p}{[}\PY{o}{\PYZhy{}}\PY{l+m+mi}{1}\PY{p}{]} \PY{o}{\PYZgt{}} \PY{n}{y\PYZus{}real}\PY{p}{)}\PY{p}{:}
                     \PY{k}{break}
                 
                 \PY{c+c1}{\PYZsh{} Adjust the alpha according to the estimate\PYZsq{}s relative difference}
                 \PY{k}{if} \PY{n}{y\PYZus{}est}\PY{p}{[}\PY{o}{\PYZhy{}}\PY{l+m+mi}{1}\PY{p}{]} \PY{o}{\PYZgt{}} \PY{n}{y\PYZus{}real}\PY{p}{:}
                     \PY{n}{alpha} \PY{o}{=} \PY{n}{alpha} \PY{o}{\PYZhy{}} \PY{n}{width} \PY{o}{/} \PY{l+m+mf}{2.0}
                 \PY{k}{elif} \PY{n}{y\PYZus{}est}\PY{p}{[}\PY{o}{\PYZhy{}}\PY{l+m+mi}{1}\PY{p}{]} \PY{o}{\PYZlt{}} \PY{n}{y\PYZus{}real}\PY{p}{:}
                     \PY{n}{alpha} \PY{o}{=} \PY{n}{alpha} \PY{o}{+} \PY{n}{width} \PY{o}{/} \PY{l+m+mf}{2.0}
                     
                 \PY{c+c1}{\PYZsh{} Half the width each time, increment the iterations}
                 \PY{n}{width} \PY{o}{/}\PY{o}{=} \PY{l+m+mf}{2.0}
                 \PY{n}{iters} \PY{o}{+}\PY{o}{=} \PY{l+m+mi}{1}
                 
             \PY{k}{return} \PY{n}{y\PYZus{}est}\PY{p}{,} \PY{n}{alpha}\PY{p}{,} \PY{n}{iters}
\end{Verbatim}

    \hypertarget{use-the-binary-search-algorithm-to-find-the-alpha-value-that-matches-the-given-condition}{%
\paragraph{Use the binary search algorithm to find the alpha value that
matches the given
condition}\label{use-the-binary-search-algorithm-to-find-the-alpha-value-that-matches-the-given-condition}}

I will be using a step size of 75 for the RK4 method so that numeric
stability is preserved. Ideally a smaller step size would be used to
ensure a more accurate result, but the problem requirement states a
limited requirement for accuracy.

    \begin{Verbatim}[commandchars=\\\{\}]
{\color{incolor}In [{\color{incolor}15}]:} \PY{n}{\PYZus{}}\PY{p}{,} \PY{n}{alpha}\PY{p}{,} \PY{n}{i} \PY{o}{=} \PY{n}{alpha\PYZus{}binary\PYZus{}search}\PY{p}{(}\PY{n}{dy\PYZus{}dx}\PY{p}{,} \PY{n}{d2y\PYZus{}dx2}\PY{p}{,} \PY{n}{y\PYZus{}real}\PY{o}{=}\PY{l+m+mi}{10}\PY{p}{,} \PY{n}{initial}\PY{o}{=}\PY{l+m+mi}{20}\PY{p}{,}
                                           \PY{n}{x\PYZus{}range}\PY{o}{=}\PY{p}{[}\PY{l+m+mi}{0}\PY{p}{,} \PY{l+m+mi}{1500}\PY{p}{]}\PY{p}{,} \PY{n}{x\PYZus{}step\PYZus{}size}\PY{o}{=}\PY{l+m+mi}{75}\PY{p}{,}
                                           \PY{n}{alpha\PYZus{}range}\PY{o}{=}\PY{p}{[}\PY{l+m+mi}{0}\PY{p}{,} \PY{l+m+mi}{100}\PY{p}{]}\PY{p}{,} \PY{n}{tol}\PY{o}{=}\PY{l+m+mf}{0.001}\PY{p}{)}
         \PY{n+nb}{print} \PY{p}{(}\PY{l+s+s2}{\PYZdq{}}\PY{l+s+se}{\PYZbs{}n}\PY{l+s+s2}{Found an alpha of }\PY{l+s+si}{\PYZpc{}.6f}\PY{l+s+s2}{, after }\PY{l+s+si}{\PYZpc{}i}\PY{l+s+s2}{ iterations.}\PY{l+s+s2}{\PYZdq{}} \PY{o}{\PYZpc{}} \PY{p}{(}\PY{n}{alpha}\PY{p}{,} \PY{n}{i}\PY{p}{)}\PY{p}{)}
\end{Verbatim}

    \begin{Verbatim}[commandchars=\\\{\}]
Alpha: 50.000000,	y(1500): 7530325144666815141508910187232100352.00000
Alpha: 25.000000,	y(1500): 6589240219362345509818200499882557440.00000
Alpha: 12.500000,	y(1500): 5656332808290053030214139594678468608.00000
Alpha:  6.250000,	y(1500): 4306259327736429587597257237950627840.00000
Alpha:  3.125000,	y(1500): 3059055851144424326418836468663320576.00000
Alpha:  1.562500,	y(1500): 11339092365329934405455388606044569600.00000
Alpha:  0.781250,	y(1500): 3321451425698408253588194370510651392.00000
Alpha:  0.390625,	y(1500): 122.20415
Alpha:  0.195312,	y(1500): 106.82599
Alpha:  0.097656,	y(1500): 76.17045
Alpha:  0.048828,	y(1500): 53.66049
Alpha:  0.024414,	y(1500): 37.61426
Alpha:  0.012207,	y(1500): 26.12319
Alpha:  0.006104,	y(1500): 17.78232
Alpha:  0.003052,	y(1500): 11.53726
Alpha:  0.001526,	y(1500): 6.45148
Alpha:  0.002289,	y(1500): 9.34556
Alpha:  0.002670,	y(1500): 10.49869
Alpha:  0.002480,	y(1500): 9.93883
Alpha:  0.002575,	y(1500): 10.22259
Alpha:  0.002527,	y(1500): 10.08171
Alpha:  0.002503,	y(1500): 10.01053
Alpha:  0.002491,	y(1500): 9.97475
Alpha:  0.002497,	y(1500): 9.99265
Alpha:  0.002500,	y(1500): 10.00159
Alpha:  0.002499,	y(1500): 9.99712
Alpha:  0.002500,	y(1500): 9.99936
Alpha:  0.002500,	y(1500): 10.00048

Found an alpha of 0.002500, after 27 iterations.

    \end{Verbatim}

    \hypertarget{plot-yx-with-the-previously-found-alpha-of-0.0025}{%
\paragraph{\texorpdfstring{Plot \(y(x)\) with the previously found
\(\alpha\) of
0.0025}{Plot y(x) with the previously found \textbackslash{}alpha of 0.0025}}\label{plot-yx-with-the-previously-found-alpha-of-0.0025}}

    \begin{Verbatim}[commandchars=\\\{\}]
{\color{incolor}In [{\color{incolor}16}]:} \PY{n}{y\PYZus{}est}\PY{p}{,} \PY{n}{\PYZus{}} \PY{o}{=} \PY{n}{runge\PYZus{}kutta4}\PY{p}{(}\PY{n}{dy\PYZus{}dx}\PY{p}{,} \PY{n}{d2y\PYZus{}dx2}\PY{p}{,} \PY{p}{[}\PY{l+m+mi}{20}\PY{p}{,} \PY{n}{alpha}\PY{p}{]}\PY{p}{,} \PY{p}{[}\PY{l+m+mi}{0}\PY{p}{,} \PY{l+m+mi}{1500}\PY{p}{]}\PY{p}{,} \PY{l+m+mi}{1}\PY{p}{)} \PY{c+c1}{\PYZsh{} Compute y(x)}
         \PY{n}{x\PYZus{}vals} \PY{o}{=} \PY{n}{np}\PY{o}{.}\PY{n}{arange}\PY{p}{(}\PY{l+m+mi}{0}\PY{p}{,} \PY{l+m+mi}{1500} \PY{o}{+} \PY{l+m+mi}{1}\PY{p}{,} \PY{l+m+mi}{1}\PY{p}{)} \PY{c+c1}{\PYZsh{} Generate the x\PYZhy{}values between 0 and 1500}
         \PY{n}{create\PYZus{}plot}\PY{p}{(}\PY{p}{[}\PY{n}{x\PYZus{}vals}\PY{p}{]}\PY{p}{,} \PY{p}{[}\PY{p}{(}\PY{n}{y\PYZus{}est}\PY{p}{,} \PY{p}{)}\PY{p}{]}\PY{p}{,} \PY{p}{[}\PY{l+s+s2}{\PYZdq{}}\PY{l+s+s2}{\PYZdl{}x\PYZdl{}}\PY{l+s+s2}{\PYZdq{}}\PY{p}{]}\PY{p}{,} \PY{p}{[}\PY{l+s+s2}{\PYZdq{}}\PY{l+s+s2}{\PYZdl{}y(x)\PYZdl{}}\PY{l+s+s2}{\PYZdq{}}\PY{p}{]}\PY{p}{,}
                     \PY{p}{[}\PY{p}{(}\PY{l+s+s2}{\PYZdq{}}\PY{l+s+s2}{\PYZdl{}y(x)\PYZdl{}, alpha=}\PY{l+s+si}{\PYZpc{}f}\PY{l+s+s2}{\PYZdq{}} \PY{o}{\PYZpc{}} \PY{n}{alpha}\PY{p}{,} \PY{p}{)}\PY{p}{]}\PY{p}{,} \PY{l+m+mi}{1}\PY{p}{,} \PY{n}{size}\PY{o}{=}\PY{p}{(}\PY{l+m+mi}{16}\PY{p}{,} \PY{l+m+mi}{10}\PY{p}{)}\PY{p}{)}
         \PY{n+nb}{print} \PY{p}{(}\PY{l+s+s2}{\PYZdq{}}\PY{l+s+s2}{With an alpha of }\PY{l+s+si}{\PYZpc{}.6f}\PY{l+s+s2}{, y(1500) is equal to }\PY{l+s+si}{\PYZpc{}.4f}\PY{l+s+s2}{\PYZdq{}} \PY{o}{\PYZpc{}} \PY{p}{(}\PY{n}{alpha}\PY{p}{,} \PY{n}{y\PYZus{}est}\PY{p}{[}\PY{o}{\PYZhy{}}\PY{l+m+mi}{1}\PY{p}{]}\PY{p}{)}\PY{p}{)}
\end{Verbatim}

    \begin{center}
    \adjustimage{max size={0.9\linewidth}{0.9\paperheight}}{output_31_0.png}
    \end{center}
    { \hspace*{\fill} \\}
    
    \begin{Verbatim}[commandchars=\\\{\}]
With an alpha of 0.002500, y(1500) is equal to 10.0001

    \end{Verbatim}

    \hypertarget{part-d}{%
\subsubsection{Part D}\label{part-d}}

In order to identify the maximum water depth of \(y(x)\) for the
\(\alpha\) previously found in \textbf{Part C} to a precision of
\(\pm5\) meters, I will use the maximum possible step size of 10, as
this means in the worst-case scenario the estimation will be at most
five meters from the real valus (if it was exactly between two points).
I'll then look at the results of the RK approximation of \(y(x)\) and
find the maximum value; specifically the maximum value in the RK
results, and the distance will correspond to the index of that value in
the x array.

    \begin{Verbatim}[commandchars=\\\{\}]
{\color{incolor}In [{\color{incolor}17}]:} \PY{n}{y\PYZus{}est}\PY{p}{,} \PY{n}{\PYZus{}} \PY{o}{=} \PY{n}{runge\PYZus{}kutta4}\PY{p}{(}\PY{n}{dy\PYZus{}dx}\PY{p}{,} \PY{n}{d2y\PYZus{}dx2}\PY{p}{,} \PY{p}{[}\PY{l+m+mi}{20}\PY{p}{,} \PY{n}{alpha}\PY{p}{]}\PY{p}{,} \PY{p}{[}\PY{l+m+mi}{0}\PY{p}{,} \PY{l+m+mi}{1500}\PY{p}{]}\PY{p}{,} \PY{n}{h}\PY{o}{=}\PY{l+m+mi}{10}\PY{p}{)} \PY{c+c1}{\PYZsh{} y(x)}
         \PY{n}{dist} \PY{o}{=} \PY{n}{np}\PY{o}{.}\PY{n}{arange}\PY{p}{(}\PY{l+m+mi}{0}\PY{p}{,} \PY{l+m+mi}{1500} \PY{o}{+} \PY{l+m+mi}{10}\PY{p}{,} \PY{l+m+mi}{10}\PY{p}{)} \PY{c+c1}{\PYZsh{} x}
         \PY{n}{max\PYZus{}index} \PY{o}{=} \PY{n}{y\PYZus{}est}\PY{o}{.}\PY{n}{index}\PY{p}{(}\PY{n+nb}{max}\PY{p}{(}\PY{n}{y\PYZus{}est}\PY{p}{)}\PY{p}{)} \PY{c+c1}{\PYZsh{} Find the index of the maximum water depth}
         \PY{n+nb}{print} \PY{p}{(}\PY{l+s+s2}{\PYZdq{}}\PY{l+s+s2}{The maximum water depth is }\PY{l+s+si}{\PYZpc{}.6f}\PY{l+s+s2}{ meters deep, at }\PY{l+s+si}{\PYZpc{}.3f}\PY{l+s+s2}{ meters.}\PY{l+s+s2}{\PYZdq{}} \PY{o}{\PYZpc{}}\PYZbs{}
                \PY{p}{(}\PY{n+nb}{max}\PY{p}{(}\PY{n}{y\PYZus{}est}\PY{p}{)}\PY{p}{,} \PY{n}{dist}\PY{p}{[}\PY{n}{max\PYZus{}index}\PY{p}{]}\PY{p}{)}\PY{p}{)}
\end{Verbatim}

    \begin{Verbatim}[commandchars=\\\{\}]
The maximum water depth is 20.310106 meters deep, at 250.000 meters.

    \end{Verbatim}


    % Add a bibliography block to the postdoc
    
    
    
    \end{document}
