
% Default to the notebook output style

    


% Inherit from the specified cell style.




    
\documentclass[11pt]{article}
\author{Collin Heist}
    
    
    \usepackage[T1]{fontenc}
    % Nicer default font (+ math font) than Computer Modern for most use cases
    \usepackage{mathpazo}

    % Basic figure setup, for now with no caption control since it's done
    % automatically by Pandoc (which extracts ![](path) syntax from Markdown).
    \usepackage{graphicx}
    % We will generate all images so they have a width \maxwidth. This means
    % that they will get their normal width if they fit onto the page, but
    % are scaled down if they would overflow the margins.
    \makeatletter
    \def\maxwidth{\ifdim\Gin@nat@width>\linewidth\linewidth
    \else\Gin@nat@width\fi}
    \makeatother
    \let\Oldincludegraphics\includegraphics
    % Set max figure width to be 80% of text width, for now hardcoded.
    \renewcommand{\includegraphics}[1]{\Oldincludegraphics[width=.8\maxwidth]{#1}}
    % Ensure that by default, figures have no caption (until we provide a
    % proper Figure object with a Caption API and a way to capture that
    % in the conversion process - todo).
    \usepackage{caption}
    \DeclareCaptionLabelFormat{nolabel}{}
    \captionsetup{labelformat=nolabel}

    \usepackage{adjustbox} % Used to constrain images to a maximum size 
    \usepackage{xcolor} % Allow colors to be defined
    \usepackage{enumerate} % Needed for markdown enumerations to work
    \usepackage{geometry} % Used to adjust the document margins
    \usepackage{amsmath} % Equations
    \usepackage{amssymb} % Equations
    \usepackage{textcomp} % defines textquotesingle
    % Hack from http://tex.stackexchange.com/a/47451/13684:
    \AtBeginDocument{%
        \def\PYZsq{\textquotesingle}% Upright quotes in Pygmentized code
    }
    \usepackage{upquote} % Upright quotes for verbatim code
    \usepackage{eurosym} % defines \euro
    \usepackage[mathletters]{ucs} % Extended unicode (utf-8) support
    \usepackage[utf8x]{inputenc} % Allow utf-8 characters in the tex document
    \usepackage{fancyvrb} % verbatim replacement that allows latex
    \usepackage{grffile} % extends the file name processing of package graphics 
                         % to support a larger range 
    % The hyperref package gives us a pdf with properly built
    % internal navigation ('pdf bookmarks' for the table of contents,
    % internal cross-reference links, web links for URLs, etc.)
    \usepackage{hyperref}
    \usepackage{longtable} % longtable support required by pandoc >1.10
    \usepackage{booktabs}  % table support for pandoc > 1.12.2
    \usepackage[inline]{enumitem} % IRkernel/repr support (it uses the enumerate* environment)
    \usepackage[normalem]{ulem} % ulem is needed to support strikethroughs (\sout)
                                % normalem makes italics be italics, not underlines
    \usepackage{mathrsfs}
    

    
    
    % Colors for the hyperref package
    \definecolor{urlcolor}{rgb}{0,.145,.698}
    \definecolor{linkcolor}{rgb}{.71,0.21,0.01}
    \definecolor{citecolor}{rgb}{.12,.54,.11}

    % ANSI colors
    \definecolor{ansi-black}{HTML}{3E424D}
    \definecolor{ansi-black-intense}{HTML}{282C36}
    \definecolor{ansi-red}{HTML}{E75C58}
    \definecolor{ansi-red-intense}{HTML}{B22B31}
    \definecolor{ansi-green}{HTML}{00A250}
    \definecolor{ansi-green-intense}{HTML}{007427}
    \definecolor{ansi-yellow}{HTML}{DDB62B}
    \definecolor{ansi-yellow-intense}{HTML}{B27D12}
    \definecolor{ansi-blue}{HTML}{208FFB}
    \definecolor{ansi-blue-intense}{HTML}{0065CA}
    \definecolor{ansi-magenta}{HTML}{D160C4}
    \definecolor{ansi-magenta-intense}{HTML}{A03196}
    \definecolor{ansi-cyan}{HTML}{60C6C8}
    \definecolor{ansi-cyan-intense}{HTML}{258F8F}
    \definecolor{ansi-white}{HTML}{C5C1B4}
    \definecolor{ansi-white-intense}{HTML}{A1A6B2}
    \definecolor{ansi-default-inverse-fg}{HTML}{FFFFFF}
    \definecolor{ansi-default-inverse-bg}{HTML}{000000}

    % commands and environments needed by pandoc snippets
    % extracted from the output of `pandoc -s`
    \providecommand{\tightlist}{%
      \setlength{\itemsep}{0pt}\setlength{\parskip}{0pt}}
    \DefineVerbatimEnvironment{Highlighting}{Verbatim}{commandchars=\\\{\}}
    % Add ',fontsize=\small' for more characters per line
    \newenvironment{Shaded}{}{}
    \newcommand{\KeywordTok}[1]{\textcolor[rgb]{0.00,0.44,0.13}{\textbf{{#1}}}}
    \newcommand{\DataTypeTok}[1]{\textcolor[rgb]{0.56,0.13,0.00}{{#1}}}
    \newcommand{\DecValTok}[1]{\textcolor[rgb]{0.25,0.63,0.44}{{#1}}}
    \newcommand{\BaseNTok}[1]{\textcolor[rgb]{0.25,0.63,0.44}{{#1}}}
    \newcommand{\FloatTok}[1]{\textcolor[rgb]{0.25,0.63,0.44}{{#1}}}
    \newcommand{\CharTok}[1]{\textcolor[rgb]{0.25,0.44,0.63}{{#1}}}
    \newcommand{\StringTok}[1]{\textcolor[rgb]{0.25,0.44,0.63}{{#1}}}
    \newcommand{\CommentTok}[1]{\textcolor[rgb]{0.38,0.63,0.69}{\textit{{#1}}}}
    \newcommand{\OtherTok}[1]{\textcolor[rgb]{0.00,0.44,0.13}{{#1}}}
    \newcommand{\AlertTok}[1]{\textcolor[rgb]{1.00,0.00,0.00}{\textbf{{#1}}}}
    \newcommand{\FunctionTok}[1]{\textcolor[rgb]{0.02,0.16,0.49}{{#1}}}
    \newcommand{\RegionMarkerTok}[1]{{#1}}
    \newcommand{\ErrorTok}[1]{\textcolor[rgb]{1.00,0.00,0.00}{\textbf{{#1}}}}
    \newcommand{\NormalTok}[1]{{#1}}
    
    % Additional commands for more recent versions of Pandoc
    \newcommand{\ConstantTok}[1]{\textcolor[rgb]{0.53,0.00,0.00}{{#1}}}
    \newcommand{\SpecialCharTok}[1]{\textcolor[rgb]{0.25,0.44,0.63}{{#1}}}
    \newcommand{\VerbatimStringTok}[1]{\textcolor[rgb]{0.25,0.44,0.63}{{#1}}}
    \newcommand{\SpecialStringTok}[1]{\textcolor[rgb]{0.73,0.40,0.53}{{#1}}}
    \newcommand{\ImportTok}[1]{{#1}}
    \newcommand{\DocumentationTok}[1]{\textcolor[rgb]{0.73,0.13,0.13}{\textit{{#1}}}}
    \newcommand{\AnnotationTok}[1]{\textcolor[rgb]{0.38,0.63,0.69}{\textbf{\textit{{#1}}}}}
    \newcommand{\CommentVarTok}[1]{\textcolor[rgb]{0.38,0.63,0.69}{\textbf{\textit{{#1}}}}}
    \newcommand{\VariableTok}[1]{\textcolor[rgb]{0.10,0.09,0.49}{{#1}}}
    \newcommand{\ControlFlowTok}[1]{\textcolor[rgb]{0.00,0.44,0.13}{\textbf{{#1}}}}
    \newcommand{\OperatorTok}[1]{\textcolor[rgb]{0.40,0.40,0.40}{{#1}}}
    \newcommand{\BuiltInTok}[1]{{#1}}
    \newcommand{\ExtensionTok}[1]{{#1}}
    \newcommand{\PreprocessorTok}[1]{\textcolor[rgb]{0.74,0.48,0.00}{{#1}}}
    \newcommand{\AttributeTok}[1]{\textcolor[rgb]{0.49,0.56,0.16}{{#1}}}
    \newcommand{\InformationTok}[1]{\textcolor[rgb]{0.38,0.63,0.69}{\textbf{\textit{{#1}}}}}
    \newcommand{\WarningTok}[1]{\textcolor[rgb]{0.38,0.63,0.69}{\textbf{\textit{{#1}}}}}
    
    
    % Define a nice break command that doesn't care if a line doesn't already
    % exist.
    \def\br{\hspace*{\fill} \\* }
    % Math Jax compatibility definitions
    \def\gt{>}
    \def\lt{<}
    \let\Oldtex\TeX
    \let\Oldlatex\LaTeX
    \renewcommand{\TeX}{\textrm{\Oldtex}}
    \renewcommand{\LaTeX}{\textrm{\Oldlatex}}
    % Document parameters
    % Document title
    \title{ECE 450 - Exam \#2}
    
    
    
    
    

    % Pygments definitions
    
\makeatletter
\def\PY@reset{\let\PY@it=\relax \let\PY@bf=\relax%
    \let\PY@ul=\relax \let\PY@tc=\relax%
    \let\PY@bc=\relax \let\PY@ff=\relax}
\def\PY@tok#1{\csname PY@tok@#1\endcsname}
\def\PY@toks#1+{\ifx\relax#1\empty\else%
    \PY@tok{#1}\expandafter\PY@toks\fi}
\def\PY@do#1{\PY@bc{\PY@tc{\PY@ul{%
    \PY@it{\PY@bf{\PY@ff{#1}}}}}}}
\def\PY#1#2{\PY@reset\PY@toks#1+\relax+\PY@do{#2}}

\expandafter\def\csname PY@tok@w\endcsname{\def\PY@tc##1{\textcolor[rgb]{0.73,0.73,0.73}{##1}}}
\expandafter\def\csname PY@tok@c\endcsname{\let\PY@it=\textit\def\PY@tc##1{\textcolor[rgb]{0.25,0.50,0.50}{##1}}}
\expandafter\def\csname PY@tok@cp\endcsname{\def\PY@tc##1{\textcolor[rgb]{0.74,0.48,0.00}{##1}}}
\expandafter\def\csname PY@tok@k\endcsname{\let\PY@bf=\textbf\def\PY@tc##1{\textcolor[rgb]{0.00,0.50,0.00}{##1}}}
\expandafter\def\csname PY@tok@kp\endcsname{\def\PY@tc##1{\textcolor[rgb]{0.00,0.50,0.00}{##1}}}
\expandafter\def\csname PY@tok@kt\endcsname{\def\PY@tc##1{\textcolor[rgb]{0.69,0.00,0.25}{##1}}}
\expandafter\def\csname PY@tok@o\endcsname{\def\PY@tc##1{\textcolor[rgb]{0.40,0.40,0.40}{##1}}}
\expandafter\def\csname PY@tok@ow\endcsname{\let\PY@bf=\textbf\def\PY@tc##1{\textcolor[rgb]{0.67,0.13,1.00}{##1}}}
\expandafter\def\csname PY@tok@nb\endcsname{\def\PY@tc##1{\textcolor[rgb]{0.00,0.50,0.00}{##1}}}
\expandafter\def\csname PY@tok@nf\endcsname{\def\PY@tc##1{\textcolor[rgb]{0.00,0.00,1.00}{##1}}}
\expandafter\def\csname PY@tok@nc\endcsname{\let\PY@bf=\textbf\def\PY@tc##1{\textcolor[rgb]{0.00,0.00,1.00}{##1}}}
\expandafter\def\csname PY@tok@nn\endcsname{\let\PY@bf=\textbf\def\PY@tc##1{\textcolor[rgb]{0.00,0.00,1.00}{##1}}}
\expandafter\def\csname PY@tok@ne\endcsname{\let\PY@bf=\textbf\def\PY@tc##1{\textcolor[rgb]{0.82,0.25,0.23}{##1}}}
\expandafter\def\csname PY@tok@nv\endcsname{\def\PY@tc##1{\textcolor[rgb]{0.10,0.09,0.49}{##1}}}
\expandafter\def\csname PY@tok@no\endcsname{\def\PY@tc##1{\textcolor[rgb]{0.53,0.00,0.00}{##1}}}
\expandafter\def\csname PY@tok@nl\endcsname{\def\PY@tc##1{\textcolor[rgb]{0.63,0.63,0.00}{##1}}}
\expandafter\def\csname PY@tok@ni\endcsname{\let\PY@bf=\textbf\def\PY@tc##1{\textcolor[rgb]{0.60,0.60,0.60}{##1}}}
\expandafter\def\csname PY@tok@na\endcsname{\def\PY@tc##1{\textcolor[rgb]{0.49,0.56,0.16}{##1}}}
\expandafter\def\csname PY@tok@nt\endcsname{\let\PY@bf=\textbf\def\PY@tc##1{\textcolor[rgb]{0.00,0.50,0.00}{##1}}}
\expandafter\def\csname PY@tok@nd\endcsname{\def\PY@tc##1{\textcolor[rgb]{0.67,0.13,1.00}{##1}}}
\expandafter\def\csname PY@tok@s\endcsname{\def\PY@tc##1{\textcolor[rgb]{0.73,0.13,0.13}{##1}}}
\expandafter\def\csname PY@tok@sd\endcsname{\let\PY@it=\textit\def\PY@tc##1{\textcolor[rgb]{0.73,0.13,0.13}{##1}}}
\expandafter\def\csname PY@tok@si\endcsname{\let\PY@bf=\textbf\def\PY@tc##1{\textcolor[rgb]{0.73,0.40,0.53}{##1}}}
\expandafter\def\csname PY@tok@se\endcsname{\let\PY@bf=\textbf\def\PY@tc##1{\textcolor[rgb]{0.73,0.40,0.13}{##1}}}
\expandafter\def\csname PY@tok@sr\endcsname{\def\PY@tc##1{\textcolor[rgb]{0.73,0.40,0.53}{##1}}}
\expandafter\def\csname PY@tok@ss\endcsname{\def\PY@tc##1{\textcolor[rgb]{0.10,0.09,0.49}{##1}}}
\expandafter\def\csname PY@tok@sx\endcsname{\def\PY@tc##1{\textcolor[rgb]{0.00,0.50,0.00}{##1}}}
\expandafter\def\csname PY@tok@m\endcsname{\def\PY@tc##1{\textcolor[rgb]{0.40,0.40,0.40}{##1}}}
\expandafter\def\csname PY@tok@gh\endcsname{\let\PY@bf=\textbf\def\PY@tc##1{\textcolor[rgb]{0.00,0.00,0.50}{##1}}}
\expandafter\def\csname PY@tok@gu\endcsname{\let\PY@bf=\textbf\def\PY@tc##1{\textcolor[rgb]{0.50,0.00,0.50}{##1}}}
\expandafter\def\csname PY@tok@gd\endcsname{\def\PY@tc##1{\textcolor[rgb]{0.63,0.00,0.00}{##1}}}
\expandafter\def\csname PY@tok@gi\endcsname{\def\PY@tc##1{\textcolor[rgb]{0.00,0.63,0.00}{##1}}}
\expandafter\def\csname PY@tok@gr\endcsname{\def\PY@tc##1{\textcolor[rgb]{1.00,0.00,0.00}{##1}}}
\expandafter\def\csname PY@tok@ge\endcsname{\let\PY@it=\textit}
\expandafter\def\csname PY@tok@gs\endcsname{\let\PY@bf=\textbf}
\expandafter\def\csname PY@tok@gp\endcsname{\let\PY@bf=\textbf\def\PY@tc##1{\textcolor[rgb]{0.00,0.00,0.50}{##1}}}
\expandafter\def\csname PY@tok@go\endcsname{\def\PY@tc##1{\textcolor[rgb]{0.53,0.53,0.53}{##1}}}
\expandafter\def\csname PY@tok@gt\endcsname{\def\PY@tc##1{\textcolor[rgb]{0.00,0.27,0.87}{##1}}}
\expandafter\def\csname PY@tok@err\endcsname{\def\PY@bc##1{\setlength{\fboxsep}{0pt}\fcolorbox[rgb]{1.00,0.00,0.00}{1,1,1}{\strut ##1}}}
\expandafter\def\csname PY@tok@kc\endcsname{\let\PY@bf=\textbf\def\PY@tc##1{\textcolor[rgb]{0.00,0.50,0.00}{##1}}}
\expandafter\def\csname PY@tok@kd\endcsname{\let\PY@bf=\textbf\def\PY@tc##1{\textcolor[rgb]{0.00,0.50,0.00}{##1}}}
\expandafter\def\csname PY@tok@kn\endcsname{\let\PY@bf=\textbf\def\PY@tc##1{\textcolor[rgb]{0.00,0.50,0.00}{##1}}}
\expandafter\def\csname PY@tok@kr\endcsname{\let\PY@bf=\textbf\def\PY@tc##1{\textcolor[rgb]{0.00,0.50,0.00}{##1}}}
\expandafter\def\csname PY@tok@bp\endcsname{\def\PY@tc##1{\textcolor[rgb]{0.00,0.50,0.00}{##1}}}
\expandafter\def\csname PY@tok@fm\endcsname{\def\PY@tc##1{\textcolor[rgb]{0.00,0.00,1.00}{##1}}}
\expandafter\def\csname PY@tok@vc\endcsname{\def\PY@tc##1{\textcolor[rgb]{0.10,0.09,0.49}{##1}}}
\expandafter\def\csname PY@tok@vg\endcsname{\def\PY@tc##1{\textcolor[rgb]{0.10,0.09,0.49}{##1}}}
\expandafter\def\csname PY@tok@vi\endcsname{\def\PY@tc##1{\textcolor[rgb]{0.10,0.09,0.49}{##1}}}
\expandafter\def\csname PY@tok@vm\endcsname{\def\PY@tc##1{\textcolor[rgb]{0.10,0.09,0.49}{##1}}}
\expandafter\def\csname PY@tok@sa\endcsname{\def\PY@tc##1{\textcolor[rgb]{0.73,0.13,0.13}{##1}}}
\expandafter\def\csname PY@tok@sb\endcsname{\def\PY@tc##1{\textcolor[rgb]{0.73,0.13,0.13}{##1}}}
\expandafter\def\csname PY@tok@sc\endcsname{\def\PY@tc##1{\textcolor[rgb]{0.73,0.13,0.13}{##1}}}
\expandafter\def\csname PY@tok@dl\endcsname{\def\PY@tc##1{\textcolor[rgb]{0.73,0.13,0.13}{##1}}}
\expandafter\def\csname PY@tok@s2\endcsname{\def\PY@tc##1{\textcolor[rgb]{0.73,0.13,0.13}{##1}}}
\expandafter\def\csname PY@tok@sh\endcsname{\def\PY@tc##1{\textcolor[rgb]{0.73,0.13,0.13}{##1}}}
\expandafter\def\csname PY@tok@s1\endcsname{\def\PY@tc##1{\textcolor[rgb]{0.73,0.13,0.13}{##1}}}
\expandafter\def\csname PY@tok@mb\endcsname{\def\PY@tc##1{\textcolor[rgb]{0.40,0.40,0.40}{##1}}}
\expandafter\def\csname PY@tok@mf\endcsname{\def\PY@tc##1{\textcolor[rgb]{0.40,0.40,0.40}{##1}}}
\expandafter\def\csname PY@tok@mh\endcsname{\def\PY@tc##1{\textcolor[rgb]{0.40,0.40,0.40}{##1}}}
\expandafter\def\csname PY@tok@mi\endcsname{\def\PY@tc##1{\textcolor[rgb]{0.40,0.40,0.40}{##1}}}
\expandafter\def\csname PY@tok@il\endcsname{\def\PY@tc##1{\textcolor[rgb]{0.40,0.40,0.40}{##1}}}
\expandafter\def\csname PY@tok@mo\endcsname{\def\PY@tc##1{\textcolor[rgb]{0.40,0.40,0.40}{##1}}}
\expandafter\def\csname PY@tok@ch\endcsname{\let\PY@it=\textit\def\PY@tc##1{\textcolor[rgb]{0.25,0.50,0.50}{##1}}}
\expandafter\def\csname PY@tok@cm\endcsname{\let\PY@it=\textit\def\PY@tc##1{\textcolor[rgb]{0.25,0.50,0.50}{##1}}}
\expandafter\def\csname PY@tok@cpf\endcsname{\let\PY@it=\textit\def\PY@tc##1{\textcolor[rgb]{0.25,0.50,0.50}{##1}}}
\expandafter\def\csname PY@tok@c1\endcsname{\let\PY@it=\textit\def\PY@tc##1{\textcolor[rgb]{0.25,0.50,0.50}{##1}}}
\expandafter\def\csname PY@tok@cs\endcsname{\let\PY@it=\textit\def\PY@tc##1{\textcolor[rgb]{0.25,0.50,0.50}{##1}}}

\def\PYZbs{\char`\\}
\def\PYZus{\char`\_}
\def\PYZob{\char`\{}
\def\PYZcb{\char`\}}
\def\PYZca{\char`\^}
\def\PYZam{\char`\&}
\def\PYZlt{\char`\<}
\def\PYZgt{\char`\>}
\def\PYZsh{\char`\#}
\def\PYZpc{\char`\%}
\def\PYZdl{\char`\$}
\def\PYZhy{\char`\-}
\def\PYZsq{\char`\'}
\def\PYZdq{\char`\"}
\def\PYZti{\char`\~}
% for compatibility with earlier versions
\def\PYZat{@}
\def\PYZlb{[}
\def\PYZrb{]}
\makeatother


    % Exact colors from NB
    \definecolor{incolor}{rgb}{0.0, 0.0, 0.5}
    \definecolor{outcolor}{rgb}{0.545, 0.0, 0.0}



    
    % Prevent overflowing lines due to hard-to-break entities
    \sloppy 
    % Setup hyperref package
    \hypersetup{
      breaklinks=true,  % so long urls are correctly broken across lines
      colorlinks=true,
      urlcolor=urlcolor,
      linkcolor=linkcolor,
      citecolor=citecolor,
      }
    % Slightly bigger margins than the latex defaults
    
    \geometry{verbose,tmargin=1in,bmargin=1in,lmargin=1in,rmargin=1in}
    
    

    \begin{document}
    
    
    \maketitle
    
    

    

\section{Solution}
\subsection{Problem A}
\subsubsection{Part i}

 \begin{center}
    \adjustimage{max size={1.15\linewidth}{0.9\paperheight}}{output_14_1.png}
    \end{center}
    { \hspace*{\fill} \\}
    
\subsubsection{Part ii}
For starters, lets' take a look at the step response of this system to
the system with no compensation network.

    \begin{center}
    \adjustimage{max size={1.15\linewidth}{0.9\paperheight}}{output_16_1.png}
    \end{center}
    { \hspace*{\fill} \\}
    
    First, I'll test a compensation network with just an increased gain. The
steady state error of the unity response is determined by:

\[e_{ss}^{step}=\frac{1}{1+lim_{s\to 0}H_0(s)}\]

\[e_{ss}^{step}=\frac{1}{1+lim_{s\to 0}(\frac{5\cdot 10^6}{s^2+6\cdot 10^3s+5\cdot 10^6})}\]

\[e_{ss}^{step}=\frac{1}{1+1}=0.5\]

In order to get this below one percent (0.01), I'll choose a gain
according to the following:

\[e_{ss}^{step}=\frac{1}{1+lim_{s\to 0}(K\cdot \frac{5\cdot 10^6}{s^2+6\cdot 10^3s+5\cdot 10^6})}=0.01\]

\[e_{ss}^{step}=\frac{1}{1+K}=0.01\]

\[1=0.01+0.01\cdot K \rightarrow 0.99=0.01\cdot K\]

\[K=99\]

Therefore a gain of \(100\) would produce the desired reduction in error
(+ some wiggle room).

    \begin{center}
    \adjustimage{max size={1.15\linewidth}{0.9\paperheight}}{output_18_0.png}
    \end{center}
    { \hspace*{\fill} \\}
    
    This seems to satisfy the problem requirements for the steady-state
error, but let's look at the values at 5 milliseconds, just to be sure.

    
    \begin{verbatim}
        Time     Value  Kind          Name  Error
20500  0.005  0.009901  Step  Step - Error  Error
    \end{verbatim}

    
    As we can see, at 5 milliseconds, the step error for this new transfer
function is below 1\%. Now, the overshoot needs to be dealt with.

Let's take a look at the phase margin for this new system, to identify
how much phase offset we need to add. From my code, the crossover is at 20603.534 (rad/s), and the phase-margin is 16.266 degrees.
    
    This phase margin of only \(16.266\) degrees, is clearly the reason we
have such an overshoot when it comes to the step response. To deal with
this, let's add a phase lead compensation. I chose a desired phase of
\(100^o\), since we want \textbf{no overshoot whatsoever}. My code reveals that the zero and pole frequencies are $5228.13$ (rad/s) and $1744826.86$ (rad/s) respectively.
    After computing the calculated zero and pole of the compensation
network, the final transfer function is:

\[G_c(s)=100\cdot \frac{1744826.86}{5228.13}\cdot \frac{s+5228.13}{s+1744826.86}\]

Now, I'll take a look at the final step response of this system - with
the compensation network added.

    \begin{center}
    \adjustimage{max size={1.15\linewidth}{0.9\paperheight}}{output_26_0.png}
    \end{center}
    { \hspace*{\fill} \\}
    
    This appears to do the job. There is no overshoot, and the response to
the unit step function definitely reaches equilibrium by 5 milliseconds.
However, just to verify, here is the error at this time:

\begin{Verbatim}[commandchars=\\\{\}]
{\color{outcolor}Out[{\color{outcolor}16}]:}         Time     Value  Kind          Name  Error
         25000  0.005  0.009901  Step  Step - Error  Error
\end{Verbatim}
    
    
\subsection{Problem B}
\subsubsection{Part i}
There is clearly a low pass filter with two poles. The reason we know
this is because the phase settles out to -180 degrees, and does not
appear to increase or have any zeros of the plotted range.

My first estimation is that the dominant pole exists at \(10^0\), or
\(1\). I am making this estimation because the magnitude response begins
to decrease at this point, and the phase reaches \(-45^o\) at this
frequency. The next pole clearly exists at \(10^1\), or \(10\). Once
again, we can see the magnitude transition from a \(-20 \frac{dB}{dec}\)
slope to \(-40 \frac{dB}{dec}\) (suggesting a pole) - \textbf{and} and
frequency is at \(-135^o\) at this point.

That results in the following approximated transfer function:

\[H(s)=K\cdot \frac{1}{(s+1)(s+10)}\]

However, we see that at near zero frequencies (\(w\approx 0\)), the
magnitude of the response is 20dB. In order to achieve this gain, K
needs to be approximately 100 - we know this because each gain of 10
corresponds to 20dB, and the starting gain of this equation is -20dB.

Thus, the following is the final transfer function:

\[H(s)=\frac{100}{(s+1)(s+10)}\]


    \begin{center}
    \adjustimage{max size={1.15\linewidth}{0.9\paperheight}}{output_30_1.png}
    \end{center}
    { \hspace*{\fill} \\}
    
    Which seems to match the given Bode plot.

\subsubsection{Part ii}

Let's start by looking at the position and velocity response.
    \begin{center}
    \adjustimage{max size={1.15\linewidth}{0.9\paperheight}}{output_32_0.png}
    \end{center}
    { \hspace*{\fill} \\}
    
    I can see that the position error seems to stabilize to about 10\%, but
the velocity error is unbounded. In order to address this, I need to
institute a compensation network that reduces the steady-state velocity
constant to a number less than 0.1.

To accomplish this, I will add a phase-lead offset, with the addition of
another pole at \(\omega =0\), in order to change the velocity error to
a bounded value. The general form of this will be:

\[G_c(s)=\frac{\omega _p}{\omega _z}\cdot \frac{s + \omega _z}{s\cdot(s+\omega _p)}\]

After calling my helper-function, the zero and pole frequencies are 6.19 (rad/s) and 25.71 (rad/s) accordingly.

    \begin{center}
    \adjustimage{max size={1.15\linewidth}{0.9\paperheight}}{output_35_0.png}
    \end{center}
    { \hspace*{\fill} \\}
    
    This seems to have done the trick. Just to verify, let's look at the
error of the response function at \(t=5\) seconds.

\begin{Verbatim}[commandchars=\\\{\}]
{\color{outcolor}Out[{\color{outcolor}21}]:}        Time     Value      Kind              Name  Error
         17000   5.0 -0.050009  Position  Position - Error  Error
         35000   5.0  0.098880  Velocity  Velocity - Error  Error
\end{Verbatim}
            
    As requested, the error of both the position and velocty have been
reduced to below 10\% after 5 seconds.





\newpage
\section{Code Appendix}

\hypertarget{package-imports}{%
\subsection{Package Imports}\label{package-imports}}

    \begin{Verbatim}[commandchars=\\\{\}]
{\color{incolor}In [{\color{incolor}1}]:} \PY{k+kn}{import} \PY{n+nn}{numpy} \PY{k}{as} \PY{n+nn}{np}
        \PY{k+kn}{import} \PY{n+nn}{seaborn} \PY{k}{as} \PY{n+nn}{sns}
        \PY{k+kn}{import} \PY{n+nn}{pandas} \PY{k}{as} \PY{n+nn}{pd}
        \PY{k+kn}{import} \PY{n+nn}{matplotlib}\PY{n+nn}{.}\PY{n+nn}{pyplot} \PY{k}{as} \PY{n+nn}{plt}
        \PY{k+kn}{from} \PY{n+nn}{scipy} \PY{k}{import} \PY{n}{signal} \PY{k}{as} \PY{n}{sig}
        \PY{k+kn}{from} \PY{n+nn}{control} \PY{k}{import} \PY{n}{margin}\PY{p}{,} \PY{n}{tf}
\end{Verbatim}

    \hypertarget{generic-functions-for-the-step-ramp-and-parabolic-inputs}{%
\subsection{Generic functions for the step, ramp, and parabolic
inputs}\label{generic-functions-for-the-step-ramp-and-parabolic-inputs}}

    \begin{Verbatim}[commandchars=\\\{\}]
{\color{incolor}In [{\color{incolor}2}]:} \PY{n}{step} \PY{o}{=} \PY{k}{lambda} \PY{n}{times}\PY{p}{:}     \PY{p}{[}\PY{l+m+mi}{0} \PY{k}{if} \PY{n}{t} \PY{o}{\PYZlt{}} \PY{l+m+mi}{0} \PY{k}{else} \PY{l+m+mi}{1} \PY{k}{for} \PY{n}{t} \PY{o+ow}{in} \PY{n}{times}\PY{p}{]}
        \PY{n}{ramp} \PY{o}{=} \PY{k}{lambda} \PY{n}{times}\PY{p}{:}     \PY{p}{[}\PY{l+m+mi}{0} \PY{k}{if} \PY{n}{t} \PY{o}{\PYZlt{}} \PY{l+m+mi}{0} \PY{k}{else} \PY{n}{t} \PY{k}{for} \PY{n}{t} \PY{o+ow}{in} \PY{n}{times}\PY{p}{]}
        \PY{n}{parabola} \PY{o}{=} \PY{k}{lambda} \PY{n}{times}\PY{p}{:} \PY{p}{[}\PY{l+m+mi}{0} \PY{k}{if} \PY{n}{t} \PY{o}{\PYZlt{}} \PY{l+m+mi}{0} \PY{k}{else} \PY{n}{t} \PY{o}{*} \PY{n}{t} \PY{k}{for} \PY{n}{t} \PY{o+ow}{in} \PY{n}{times}\PY{p}{]}
\end{Verbatim}

    \hypertarget{generic-function-to-convolve-any-number-of-equations}{%
\subsection{Generic function to convolve any number of
equations}\label{generic-function-to-convolve-any-number-of-equations}}

    \begin{Verbatim}[commandchars=\\\{\}]
{\color{incolor}In [{\color{incolor}3}]:} \PY{k}{def} \PY{n+nf}{convolve\PYZus{}all}\PY{p}{(}\PY{n}{values}\PY{p}{)}\PY{p}{:}
            \PY{n}{temp\PYZus{}conv} \PY{o}{=} \PY{n}{values}\PY{p}{[}\PY{l+m+mi}{0}\PY{p}{]}
            \PY{k}{if} \PY{n+nb}{len}\PY{p}{(}\PY{n}{values}\PY{p}{)} \PY{o}{\PYZgt{}} \PY{l+m+mi}{1}\PY{p}{:}
                \PY{k}{for} \PY{n}{next\PYZus{}val} \PY{o+ow}{in} \PY{n}{values}\PY{p}{[}\PY{l+m+mi}{1}\PY{p}{:}\PY{p}{]}\PY{p}{:}
                    \PY{n}{temp\PYZus{}conv} \PY{o}{=} \PY{n}{np}\PY{o}{.}\PY{n}{convolve}\PY{p}{(}\PY{n}{temp\PYZus{}conv}\PY{p}{,} \PY{n}{next\PYZus{}val}\PY{p}{)}
            
            \PY{k}{return} \PY{n}{temp\PYZus{}conv}
\end{Verbatim}

    \hypertarget{generic-function-to-generate-the-magnitude-and-phase-of-hjomega-values}{%
\subsection{\texorpdfstring{Generic function to generate the magnitude
and phase of \(H(j\omega)\)
values}{Generic function to generate the magnitude and phase of H(j\textbackslash{}omega) values}}\label{generic-function-to-generate-the-magnitude-and-phase-of-hjomega-values}}

    \begin{Verbatim}[commandchars=\\\{\}]
{\color{incolor}In [{\color{incolor}4}]:} \PY{k}{def} \PY{n+nf}{magnitude\PYZus{}phase\PYZus{}response}\PY{p}{(}\PY{n}{num}\PY{p}{,} \PY{n}{den}\PY{p}{,} \PY{n}{omega\PYZus{}range}\PY{p}{,} \PY{n}{omega\PYZus{}step}\PY{o}{=}\PY{l+m+mi}{10}\PY{p}{,} \PY{n}{gain\PYZus{}num}\PY{o}{=}\PY{k+kc}{None}\PY{p}{,} \PY{n}{gain\PYZus{}den}\PY{o}{=}\PY{k+kc}{None}\PY{p}{)}\PY{p}{:}
            \PY{k}{if} \PY{n+nb}{isinstance}\PY{p}{(}\PY{n}{gain\PYZus{}num}\PY{p}{,} \PY{p}{(}\PY{n}{np}\PY{o}{.}\PY{n}{ndarray}\PY{p}{,} \PY{n+nb}{list}\PY{p}{)}\PY{p}{)} \PY{o+ow}{and} \PY{n+nb}{isinstance}\PY{p}{(}\PY{n}{gain\PYZus{}den}\PY{p}{,} \PY{p}{(}\PY{n}{np}\PY{o}{.}\PY{n}{ndarray}\PY{p}{,} \PY{n+nb}{list}\PY{p}{)}\PY{p}{)}\PY{p}{:}
                \PY{n}{num} \PY{o}{=} \PY{n}{convolve\PYZus{}all}\PY{p}{(}\PY{p}{[}\PY{n}{num}\PY{p}{,} \PY{n}{gain\PYZus{}num}\PY{p}{]}\PY{p}{)}
                \PY{n}{den} \PY{o}{=} \PY{n}{convolve\PYZus{}all}\PY{p}{(}\PY{p}{[}\PY{n}{den}\PY{p}{,} \PY{n}{gain\PYZus{}den}\PY{p}{]}\PY{p}{)}
                
            \PY{n}{system} \PY{o}{=} \PY{n}{sig}\PY{o}{.}\PY{n}{lti}\PY{p}{(}\PY{n}{num}\PY{p}{,} \PY{n}{den}\PY{p}{)}
            \PY{n}{w}\PY{p}{,} \PY{n}{h\PYZus{}mag}\PY{p}{,} \PY{n}{h\PYZus{}phase} \PY{o}{=} \PY{n}{sig}\PY{o}{.}\PY{n}{bode}\PY{p}{(}\PY{n}{system}\PY{p}{,} \PY{n}{np}\PY{o}{.}\PY{n}{arange}\PY{p}{(}\PY{n}{omega\PYZus{}range}\PY{p}{[}\PY{l+m+mi}{0}\PY{p}{]}\PY{p}{,} \PY{n}{omega\PYZus{}range}\PY{p}{[}\PY{l+m+mi}{1}\PY{p}{]}\PY{p}{,} \PY{n}{omega\PYZus{}step}\PY{p}{)}\PY{p}{)}
            \PY{n}{\PYZus{}}\PY{p}{,} \PY{n}{phase\PYZus{}margin}\PY{p}{,} \PY{n}{\PYZus{}}\PY{p}{,} \PY{n}{crossover\PYZus{}w} \PY{o}{=} \PY{n}{margin}\PY{p}{(}\PY{n}{h\PYZus{}mag}\PY{p}{,} \PY{n}{h\PYZus{}phase}\PY{p}{,} \PY{n}{w}\PY{p}{)}
            
            \PY{n}{df\PYZus{}list} \PY{o}{=} \PY{p}{[}\PY{p}{]}
            \PY{n}{df} \PY{o}{=} \PY{n}{pd}\PY{o}{.}\PY{n}{DataFrame}\PY{p}{(}\PY{n+nb}{list}\PY{p}{(}\PY{n+nb}{zip}\PY{p}{(}\PY{n}{w}\PY{p}{,} \PY{n}{h\PYZus{}mag}\PY{p}{)}\PY{p}{)}\PY{p}{,} \PY{n}{columns}\PY{o}{=}\PY{p}{[}\PY{l+s+s2}{\PYZdq{}}\PY{l+s+s2}{\PYZdl{}}\PY{l+s+s2}{\PYZbs{}}\PY{l+s+s2}{omega\PYZdl{}}\PY{l+s+s2}{\PYZdq{}}\PY{p}{,} \PY{l+s+s2}{\PYZdq{}}\PY{l+s+s2}{Value}\PY{l+s+s2}{\PYZdq{}}\PY{p}{]}\PY{p}{)}
            \PY{n}{df}\PY{p}{[}\PY{l+s+s2}{\PYZdq{}}\PY{l+s+s2}{Kind}\PY{l+s+s2}{\PYZdq{}}\PY{p}{]} \PY{o}{=} \PY{l+s+s2}{\PYZdq{}}\PY{l+s+s2}{Magnitude Response}\PY{l+s+s2}{\PYZdq{}}
            \PY{n}{df\PYZus{}list}\PY{o}{.}\PY{n}{append}\PY{p}{(}\PY{n}{df}\PY{p}{)}
            
            \PY{n}{df} \PY{o}{=} \PY{n}{pd}\PY{o}{.}\PY{n}{DataFrame}\PY{p}{(}\PY{n+nb}{list}\PY{p}{(}\PY{n+nb}{zip}\PY{p}{(}\PY{n}{w}\PY{p}{,} \PY{n}{h\PYZus{}phase}\PY{p}{)}\PY{p}{)}\PY{p}{,} \PY{n}{columns}\PY{o}{=}\PY{p}{[}\PY{l+s+s2}{\PYZdq{}}\PY{l+s+s2}{\PYZdl{}}\PY{l+s+s2}{\PYZbs{}}\PY{l+s+s2}{omega\PYZdl{}}\PY{l+s+s2}{\PYZdq{}}\PY{p}{,} \PY{l+s+s2}{\PYZdq{}}\PY{l+s+s2}{Value}\PY{l+s+s2}{\PYZdq{}}\PY{p}{]}\PY{p}{)}
            \PY{n}{df}\PY{p}{[}\PY{l+s+s2}{\PYZdq{}}\PY{l+s+s2}{Kind}\PY{l+s+s2}{\PYZdq{}}\PY{p}{]} \PY{o}{=} \PY{l+s+s2}{\PYZdq{}}\PY{l+s+s2}{Phase Response}\PY{l+s+s2}{\PYZdq{}}
            \PY{n}{df\PYZus{}list}\PY{o}{.}\PY{n}{append}\PY{p}{(}\PY{n}{df}\PY{p}{)}
            
            \PY{k}{return} \PY{n}{pd}\PY{o}{.}\PY{n}{concat}\PY{p}{(}\PY{n}{df\PYZus{}list}\PY{p}{,} \PY{n}{ignore\PYZus{}index}\PY{o}{=}\PY{k+kc}{True}\PY{p}{,} \PY{n}{axis}\PY{o}{=}\PY{l+m+mi}{0}\PY{p}{)}\PY{p}{,} \PY{n}{phase\PYZus{}margin}\PY{p}{,} \PY{n}{crossover\PYZus{}w}
\end{Verbatim}

    \hypertarget{generic-function-to-obtain-response-of-a-system-to-inputs}{%
\subsection{Generic function to obtain response of a system to
inputs}\label{generic-function-to-obtain-response-of-a-system-to-inputs}}

    \begin{Verbatim}[commandchars=\\\{\}]
{\color{incolor}In [{\color{incolor}5}]:} \PY{k}{def} \PY{n+nf}{response\PYZus{}to\PYZus{}inputs}\PY{p}{(}\PY{n}{num}\PY{p}{,} \PY{n}{den}\PY{p}{,} \PY{n}{input\PYZus{}funcs}\PY{p}{,} \PY{n}{input\PYZus{}names}\PY{p}{,} \PY{n}{time}\PY{p}{,} \PY{n}{gain\PYZus{}num}\PY{o}{=}\PY{k+kc}{None}\PY{p}{,} \PY{n}{gain\PYZus{}den}\PY{o}{=}\PY{k+kc}{None}\PY{p}{)}\PY{p}{:}
            \PY{n}{df\PYZus{}list} \PY{o}{=} \PY{p}{[}\PY{p}{]}
            
            \PY{c+c1}{\PYZsh{} If a gain equation was given, adjust the system num / dun}
            \PY{k}{if} \PY{n+nb}{isinstance}\PY{p}{(}\PY{n}{gain\PYZus{}num}\PY{p}{,} \PY{p}{(}\PY{n}{np}\PY{o}{.}\PY{n}{ndarray}\PY{p}{,} \PY{n+nb}{list}\PY{p}{)}\PY{p}{)} \PY{o+ow}{and} \PY{n+nb}{isinstance}\PY{p}{(}\PY{n}{gain\PYZus{}den}\PY{p}{,} \PY{p}{(}\PY{n}{np}\PY{o}{.}\PY{n}{ndarray}\PY{p}{,} \PY{n+nb}{list}\PY{p}{)}\PY{p}{)}\PY{p}{:}
                \PY{n}{num} \PY{o}{=} \PY{n}{convolve\PYZus{}all}\PY{p}{(}\PY{p}{[}\PY{n}{num}\PY{p}{,} \PY{n}{gain\PYZus{}num}\PY{p}{]}\PY{p}{)}
                \PY{n}{den} \PY{o}{=} \PY{n}{convolve\PYZus{}all}\PY{p}{(}\PY{p}{[}\PY{n}{den}\PY{p}{,} \PY{n}{gain\PYZus{}den}\PY{p}{]}\PY{p}{)}
            
            \PY{n}{num} \PY{o}{=} \PY{n}{np}\PY{o}{.}\PY{n}{pad}\PY{p}{(}\PY{n}{num}\PY{p}{,} \PY{p}{(}\PY{n+nb}{len}\PY{p}{(}\PY{n}{den}\PY{p}{)} \PY{o}{\PYZhy{}} \PY{n+nb}{len}\PY{p}{(}\PY{n}{num}\PY{p}{)}\PY{p}{,} \PY{l+m+mi}{0}\PY{p}{)}\PY{p}{,} \PY{l+s+s2}{\PYZdq{}}\PY{l+s+s2}{constant}\PY{l+s+s2}{\PYZdq{}}\PY{p}{)} \PY{c+c1}{\PYZsh{} Make arrays same length}
            \PY{n}{den} \PY{o}{=} \PY{n}{np}\PY{o}{.}\PY{n}{add}\PY{p}{(}\PY{n}{den}\PY{p}{,} \PY{n}{num}\PY{p}{)}
            \PY{k}{for} \PY{n}{in\PYZus{}name}\PY{p}{,} \PY{n}{in\PYZus{}f} \PY{o+ow}{in} \PY{n+nb}{zip}\PY{p}{(}\PY{n}{input\PYZus{}names}\PY{p}{,} \PY{n}{input\PYZus{}funcs}\PY{p}{)}\PY{p}{:}
                \PY{n}{df} \PY{o}{=} \PY{n}{pd}\PY{o}{.}\PY{n}{DataFrame}\PY{p}{(}\PY{n+nb}{list}\PY{p}{(}\PY{n+nb}{zip}\PY{p}{(}\PY{n}{time}\PY{p}{,} \PY{n}{in\PYZus{}f}\PY{p}{(}\PY{n}{time}\PY{p}{)}\PY{p}{)}\PY{p}{)}\PY{p}{,} \PY{n}{columns}\PY{o}{=}\PY{p}{[}\PY{l+s+s2}{\PYZdq{}}\PY{l+s+s2}{Time}\PY{l+s+s2}{\PYZdq{}}\PY{p}{,} \PY{l+s+s2}{\PYZdq{}}\PY{l+s+s2}{Value}\PY{l+s+s2}{\PYZdq{}}\PY{p}{]}\PY{p}{)}
                \PY{n}{df}\PY{p}{[}\PY{l+s+s2}{\PYZdq{}}\PY{l+s+s2}{Kind}\PY{l+s+s2}{\PYZdq{}}\PY{p}{]} \PY{o}{=} \PY{n}{df}\PY{p}{[}\PY{l+s+s2}{\PYZdq{}}\PY{l+s+s2}{Name}\PY{l+s+s2}{\PYZdq{}}\PY{p}{]} \PY{o}{=} \PY{n}{in\PYZus{}name}
                \PY{n}{df}\PY{p}{[}\PY{l+s+s2}{\PYZdq{}}\PY{l+s+s2}{Error}\PY{l+s+s2}{\PYZdq{}}\PY{p}{]} \PY{o}{=} \PY{l+s+s2}{\PYZdq{}}\PY{l+s+s2}{Response}\PY{l+s+s2}{\PYZdq{}}
                \PY{n}{df\PYZus{}list}\PY{o}{.}\PY{n}{append}\PY{p}{(}\PY{n}{df}\PY{p}{)}
                
                \PY{n}{\PYZus{}}\PY{p}{,} \PY{n}{response}\PY{p}{,} \PY{n}{\PYZus{}} \PY{o}{=} \PY{n}{sig}\PY{o}{.}\PY{n}{lsim}\PY{p}{(}\PY{p}{(}\PY{n}{num}\PY{p}{,} \PY{n}{den}\PY{p}{)}\PY{p}{,} \PY{n}{in\PYZus{}f}\PY{p}{(}\PY{n}{time}\PY{p}{)}\PY{p}{,} \PY{n}{time}\PY{p}{)}
                \PY{n}{df} \PY{o}{=} \PY{n}{pd}\PY{o}{.}\PY{n}{DataFrame}\PY{p}{(}\PY{n+nb}{list}\PY{p}{(}\PY{n+nb}{zip}\PY{p}{(}\PY{n}{time}\PY{p}{,} \PY{n}{response}\PY{p}{)}\PY{p}{)}\PY{p}{,} \PY{n}{columns}\PY{o}{=}\PY{p}{[}\PY{l+s+s2}{\PYZdq{}}\PY{l+s+s2}{Time}\PY{l+s+s2}{\PYZdq{}}\PY{p}{,} \PY{l+s+s2}{\PYZdq{}}\PY{l+s+s2}{Value}\PY{l+s+s2}{\PYZdq{}}\PY{p}{]}\PY{p}{)}
                \PY{n}{df}\PY{p}{[}\PY{l+s+s2}{\PYZdq{}}\PY{l+s+s2}{Kind}\PY{l+s+s2}{\PYZdq{}}\PY{p}{]} \PY{o}{=} \PY{n}{in\PYZus{}name}
                \PY{n}{df}\PY{p}{[}\PY{l+s+s2}{\PYZdq{}}\PY{l+s+s2}{Name}\PY{l+s+s2}{\PYZdq{}}\PY{p}{]} \PY{o}{=} \PY{n}{in\PYZus{}name} \PY{o}{+} \PY{l+s+s2}{\PYZdq{}}\PY{l+s+s2}{ \PYZhy{} Response}\PY{l+s+s2}{\PYZdq{}}
                \PY{n}{df}\PY{p}{[}\PY{l+s+s2}{\PYZdq{}}\PY{l+s+s2}{Error}\PY{l+s+s2}{\PYZdq{}}\PY{p}{]} \PY{o}{=} \PY{l+s+s2}{\PYZdq{}}\PY{l+s+s2}{Response}\PY{l+s+s2}{\PYZdq{}}
                \PY{n}{df\PYZus{}list}\PY{o}{.}\PY{n}{append}\PY{p}{(}\PY{n}{df}\PY{p}{)}
                
                \PY{n}{response\PYZus{}err} \PY{o}{=}  \PY{n}{np}\PY{o}{.}\PY{n}{subtract}\PY{p}{(}\PY{n}{in\PYZus{}f}\PY{p}{(}\PY{n}{t}\PY{p}{)}\PY{p}{,} \PY{n}{response}\PY{p}{)}
                \PY{n}{df} \PY{o}{=} \PY{n}{pd}\PY{o}{.}\PY{n}{DataFrame}\PY{p}{(}\PY{n+nb}{list}\PY{p}{(}\PY{n+nb}{zip}\PY{p}{(}\PY{n}{time}\PY{p}{,} \PY{n}{response\PYZus{}err}\PY{p}{)}\PY{p}{)}\PY{p}{,} \PY{n}{columns}\PY{o}{=}\PY{p}{[}\PY{l+s+s2}{\PYZdq{}}\PY{l+s+s2}{Time}\PY{l+s+s2}{\PYZdq{}}\PY{p}{,} \PY{l+s+s2}{\PYZdq{}}\PY{l+s+s2}{Value}\PY{l+s+s2}{\PYZdq{}}\PY{p}{]}\PY{p}{)}
                \PY{n}{df}\PY{p}{[}\PY{l+s+s2}{\PYZdq{}}\PY{l+s+s2}{Kind}\PY{l+s+s2}{\PYZdq{}}\PY{p}{]} \PY{o}{=} \PY{n}{in\PYZus{}name}
                \PY{n}{df}\PY{p}{[}\PY{l+s+s2}{\PYZdq{}}\PY{l+s+s2}{Name}\PY{l+s+s2}{\PYZdq{}}\PY{p}{]} \PY{o}{=} \PY{n}{in\PYZus{}name} \PY{o}{+} \PY{l+s+s2}{\PYZdq{}}\PY{l+s+s2}{ \PYZhy{} Error}\PY{l+s+s2}{\PYZdq{}}
                \PY{n}{df}\PY{p}{[}\PY{l+s+s2}{\PYZdq{}}\PY{l+s+s2}{Error}\PY{l+s+s2}{\PYZdq{}}\PY{p}{]} \PY{o}{=} \PY{l+s+s2}{\PYZdq{}}\PY{l+s+s2}{Error}\PY{l+s+s2}{\PYZdq{}}
                \PY{n}{df\PYZus{}list}\PY{o}{.}\PY{n}{append}\PY{p}{(}\PY{n}{df}\PY{p}{)}
            
            \PY{k}{return} \PY{n}{pd}\PY{o}{.}\PY{n}{concat}\PY{p}{(}\PY{n}{df\PYZus{}list}\PY{p}{,} \PY{n}{ignore\PYZus{}index}\PY{o}{=}\PY{k+kc}{True}\PY{p}{,} \PY{n}{axis}\PY{o}{=}\PY{l+m+mi}{0}\PY{p}{)}
\end{Verbatim}

    \hypertarget{generic-function-to-plot-the-responses-of-a-system}{%
\subsection{Generic function to plot the responses of a
system}\label{generic-function-to-plot-the-responses-of-a-system}}

    \begin{Verbatim}[commandchars=\\\{\}]
{\color{incolor}In [{\color{incolor}6}]:} \PY{k}{def} \PY{n+nf}{create\PYZus{}plots}\PY{p}{(}\PY{n}{df}\PY{p}{,} \PY{n}{error}\PY{o}{=}\PY{k+kc}{False}\PY{p}{)}\PY{p}{:}
            \PY{k}{if} \PY{n}{error}\PY{p}{:}
                \PY{n}{sns}\PY{o}{.}\PY{n}{set}\PY{p}{(}\PY{n}{style}\PY{o}{=}\PY{l+s+s2}{\PYZdq{}}\PY{l+s+s2}{whitegrid}\PY{l+s+s2}{\PYZdq{}}\PY{p}{,} \PY{n}{font\PYZus{}scale}\PY{o}{=}\PY{l+m+mf}{2.75}\PY{p}{)}
                \PY{n}{g} \PY{o}{=} \PY{n}{sns}\PY{o}{.}\PY{n}{FacetGrid}\PY{p}{(}\PY{n}{df}\PY{p}{,} \PY{n}{hue}\PY{o}{=}\PY{l+s+s2}{\PYZdq{}}\PY{l+s+s2}{Name}\PY{l+s+s2}{\PYZdq{}}\PY{p}{,} \PY{n}{row}\PY{o}{=}\PY{l+s+s2}{\PYZdq{}}\PY{l+s+s2}{Kind}\PY{l+s+s2}{\PYZdq{}}\PY{p}{,} \PY{n}{col}\PY{o}{=}\PY{l+s+s2}{\PYZdq{}}\PY{l+s+s2}{Error}\PY{l+s+s2}{\PYZdq{}}\PY{p}{,} \PY{n}{height}\PY{o}{=}\PY{l+m+mf}{7.5}\PY{p}{,} \PY{n}{aspect}\PY{o}{=}\PY{l+m+mf}{1.2}\PY{p}{,} \PY{n}{sharey}\PY{o}{=}\PY{k+kc}{False}\PY{p}{)}
                \PY{n}{g}\PY{o}{.}\PY{n}{map}\PY{p}{(}\PY{n}{sns}\PY{o}{.}\PY{n}{lineplot}\PY{p}{,} \PY{l+s+s2}{\PYZdq{}}\PY{l+s+s2}{Time}\PY{l+s+s2}{\PYZdq{}}\PY{p}{,} \PY{l+s+s2}{\PYZdq{}}\PY{l+s+s2}{Value}\PY{l+s+s2}{\PYZdq{}}\PY{p}{,} \PY{o}{*}\PY{o}{*}\PY{n+nb}{dict}\PY{p}{(}\PY{n}{linewidth}\PY{o}{=}\PY{l+m+mf}{2.5}\PY{p}{)}\PY{p}{)}\PY{o}{.}\PY{n}{add\PYZus{}legend}\PY{p}{(}\PY{p}{)}\PY{o}{.}\PY{n}{despine}\PY{p}{(}\PY{n}{bottom}\PY{o}{=}\PY{k+kc}{True}\PY{p}{,} \PY{n}{left}\PY{o}{=}\PY{k+kc}{True}\PY{p}{)}
            \PY{k}{else}\PY{p}{:}
                \PY{n}{sns}\PY{o}{.}\PY{n}{set}\PY{p}{(}\PY{n}{style}\PY{o}{=}\PY{l+s+s2}{\PYZdq{}}\PY{l+s+s2}{whitegrid}\PY{l+s+s2}{\PYZdq{}}\PY{p}{,} \PY{n}{font\PYZus{}scale}\PY{o}{=}\PY{l+m+mf}{1.5}\PY{p}{)}
                \PY{n}{g} \PY{o}{=} \PY{n}{sns}\PY{o}{.}\PY{n}{FacetGrid}\PY{p}{(}\PY{n}{df}\PY{p}{,} \PY{n}{hue}\PY{o}{=}\PY{l+s+s2}{\PYZdq{}}\PY{l+s+s2}{Kind}\PY{l+s+s2}{\PYZdq{}}\PY{p}{,} \PY{n}{row}\PY{o}{=}\PY{l+s+s2}{\PYZdq{}}\PY{l+s+s2}{Kind}\PY{l+s+s2}{\PYZdq{}}\PY{p}{,} \PY{n}{height}\PY{o}{=}\PY{l+m+mf}{5.5}\PY{p}{,} \PY{n}{aspect}\PY{o}{=}\PY{l+m+mf}{1.75}\PY{p}{,}
                                  \PY{n}{sharey}\PY{o}{=}\PY{k+kc}{False}\PY{p}{,} \PY{n}{gridspec\PYZus{}kws}\PY{o}{=}\PY{p}{\PYZob{}}\PY{l+s+s2}{\PYZdq{}}\PY{l+s+s2}{hspace}\PY{l+s+s2}{\PYZdq{}}\PY{p}{:}\PY{l+m+mf}{0.3}\PY{p}{\PYZcb{}}\PY{p}{)}
                \PY{n}{g}\PY{o}{.}\PY{n}{map}\PY{p}{(}\PY{n}{sns}\PY{o}{.}\PY{n}{lineplot}\PY{p}{,} \PY{l+s+s2}{\PYZdq{}}\PY{l+s+s2}{\PYZdl{}}\PY{l+s+s2}{\PYZbs{}}\PY{l+s+s2}{omega\PYZdl{}}\PY{l+s+s2}{\PYZdq{}}\PY{p}{,} \PY{l+s+s2}{\PYZdq{}}\PY{l+s+s2}{Value}\PY{l+s+s2}{\PYZdq{}}\PY{p}{,} \PY{o}{*}\PY{o}{*}\PY{n+nb}{dict}\PY{p}{(}\PY{n}{linewidth}\PY{o}{=}\PY{l+m+mf}{2.5}\PY{p}{)}\PY{p}{)}\PY{o}{.}\PY{n}{add\PYZus{}legend}\PY{p}{(}\PY{p}{)}\PY{o}{.}\PY{n}{despine}\PY{p}{(}\PY{n}{bottom}\PY{o}{=}\PY{k+kc}{True}\PY{p}{,} \PY{n}{left}\PY{o}{=}\PY{k+kc}{True}\PY{p}{)}\PY{o}{.}\PY{n}{set}\PY{p}{(}\PY{n}{xscale}\PY{o}{=}\PY{l+s+s2}{\PYZdq{}}\PY{l+s+s2}{log}\PY{l+s+s2}{\PYZdq{}}\PY{p}{)}
            
            \PY{k}{return} \PY{n}{g}
\end{Verbatim}

    \begin{Verbatim}[commandchars=\\\{\}]
{\color{incolor}In [{\color{incolor}7}]:} \PY{k}{def} \PY{n+nf}{compute\PYZus{}phase\PYZus{}lead}\PY{p}{(}\PY{n}{df}\PY{p}{,} \PY{n}{desired\PYZus{}pm}\PY{p}{)}\PY{p}{:}
            \PY{n}{alpha} \PY{o}{=} \PY{p}{(}\PY{l+m+mi}{1} \PY{o}{+} \PY{n}{np}\PY{o}{.}\PY{n}{sin}\PY{p}{(}\PY{n}{desired\PYZus{}pm} \PY{o}{*} \PY{n}{np}\PY{o}{.}\PY{n}{pi} \PY{o}{/} \PY{l+m+mi}{180}\PY{p}{)}\PY{p}{)} \PY{o}{/} \PY{p}{(}\PY{l+m+mi}{1} \PY{o}{\PYZhy{}} \PY{n}{np}\PY{o}{.}\PY{n}{sin}\PY{p}{(}\PY{n}{desired\PYZus{}pm} \PY{o}{*} \PY{n}{np}\PY{o}{.}\PY{n}{pi} \PY{o}{/} \PY{l+m+mi}{180}\PY{p}{)}\PY{p}{)}
            \PY{n}{alpha\PYZus{}db} \PY{o}{=} \PY{o}{\PYZhy{}}\PY{l+m+mi}{10} \PY{o}{*} \PY{n}{np}\PY{o}{.}\PY{n}{log10}\PY{p}{(}\PY{n}{alpha}\PY{p}{)}
            \PY{n}{omega\PYZus{}m} \PY{o}{=} \PY{n}{df}\PY{p}{[}\PY{l+s+s2}{\PYZdq{}}\PY{l+s+s2}{\PYZdl{}}\PY{l+s+s2}{\PYZbs{}}\PY{l+s+s2}{omega\PYZdl{}}\PY{l+s+s2}{\PYZdq{}}\PY{p}{]}\PY{o}{.}\PY{n}{iloc}\PY{p}{[}\PY{p}{(}\PY{n}{df}\PY{p}{[}\PY{l+s+s2}{\PYZdq{}}\PY{l+s+s2}{Value}\PY{l+s+s2}{\PYZdq{}}\PY{p}{]} \PY{o}{\PYZhy{}} \PY{n}{alpha\PYZus{}db}\PY{p}{)}\PY{o}{.}\PY{n}{abs}\PY{p}{(}\PY{p}{)}\PY{o}{.}\PY{n}{argsort}\PY{p}{(}\PY{p}{)}\PY{p}{[}\PY{p}{:}\PY{l+m+mi}{1}\PY{p}{]}\PY{o}{.}\PY{n}{values}\PY{p}{[}\PY{l+m+mi}{0}\PY{p}{]}\PY{p}{]}
            \PY{n}{omega\PYZus{}p} \PY{o}{=} \PY{n}{np}\PY{o}{.}\PY{n}{sqrt}\PY{p}{(}\PY{n}{alpha}\PY{p}{)} \PY{o}{*} \PY{n}{omega\PYZus{}m}
            \PY{n}{omega\PYZus{}z} \PY{o}{=} \PY{n}{omega\PYZus{}m} \PY{o}{/} \PY{n}{np}\PY{o}{.}\PY{n}{sqrt}\PY{p}{(}\PY{n}{alpha}\PY{p}{)}
            
            \PY{k}{return} \PY{n}{omega\PYZus{}z}\PY{p}{,} \PY{n}{omega\PYZus{}p}
\end{Verbatim}

    \hypertarget{problem-a}{%
\subsection{Problem A}\label{problem-a}}

\hypertarget{part-i}{%
\subsubsection{Part i}\label{part-i}}

A system is described by:

\[H_0(s)=\frac{5\cdot 10^6}{s^2+6\cdot 10^3s+5\cdot 10^6}\]

    \begin{Verbatim}[commandchars=\\\{\}]
{\color{incolor}In [{\color{incolor}8}]:} \PY{n}{num} \PY{o}{=} \PY{p}{[}\PY{l+m+mi}{5} \PY{o}{*} \PY{l+m+mi}{10} \PY{o}{*}\PY{o}{*} \PY{l+m+mi}{6}\PY{p}{]}
        \PY{n}{den} \PY{o}{=} \PY{n}{convolve\PYZus{}all}\PY{p}{(}\PY{p}{[}\PY{p}{[}\PY{l+m+mi}{1}\PY{p}{,} \PY{l+m+mi}{6} \PY{o}{*} \PY{l+m+mi}{10} \PY{o}{*}\PY{o}{*} \PY{l+m+mi}{3}\PY{p}{,} \PY{l+m+mi}{5} \PY{o}{*} \PY{l+m+mi}{10} \PY{o}{*}\PY{o}{*} \PY{l+m+mi}{6}\PY{p}{]}\PY{p}{]}\PY{p}{)}
        
        \PY{n}{df}\PY{p}{,} \PY{n}{m}\PY{p}{,} \PY{n}{w} \PY{o}{=} \PY{n}{magnitude\PYZus{}phase\PYZus{}response}\PY{p}{(}\PY{n}{num}\PY{p}{,} \PY{n}{den}\PY{p}{,} \PY{p}{[}\PY{l+m+mf}{0.1}\PY{p}{,} \PY{l+m+mi}{10} \PY{o}{*}\PY{o}{*} \PY{l+m+mi}{6}\PY{p}{]}\PY{p}{,} \PY{l+m+mi}{10}\PY{p}{)}
        \PY{n}{g} \PY{o}{=} \PY{n}{create\PYZus{}plots}\PY{p}{(}\PY{n}{df}\PY{p}{)}
        \PY{k}{for} \PY{n}{ax} \PY{o+ow}{in} \PY{n}{g}\PY{o}{.}\PY{n}{axes}\PY{o}{.}\PY{n}{flatten}\PY{p}{(}\PY{p}{)}\PY{p}{:}
            \PY{n}{ax}\PY{o}{.}\PY{n}{tick\PYZus{}params}\PY{p}{(}\PY{n}{labelbottom}\PY{o}{=}\PY{k+kc}{True}\PY{p}{)}
\end{Verbatim}

    \begin{Verbatim}[commandchars=\\\{\}]
/usr/local/lib/python3.7/site-packages/matplotlib/figure.py:2359: UserWarning: This figure includes Axes that are not compatible with tight\_layout, so results might be incorrect.
  warnings.warn("This figure includes Axes that are not compatible "

    \end{Verbatim}

    \begin{center}
    \adjustimage{max size={0.9\linewidth}{0.9\paperheight}}{output_14_1.png}
    \end{center}
    { \hspace*{\fill} \\}
    
    \hypertarget{part-ii}{%
\subsubsection{Part ii}\label{part-ii}}

For starters, lets' take a look at the step response of this system to
the system with no compensation network.

    \begin{Verbatim}[commandchars=\\\{\}]
{\color{incolor}In [{\color{incolor}9}]:} \PY{n}{t} \PY{o}{=} \PY{n}{np}\PY{o}{.}\PY{n}{arange}\PY{p}{(}\PY{l+m+mi}{0}\PY{p}{,} \PY{l+m+mf}{0.1}\PY{p}{,} \PY{l+m+mf}{0.00001}\PY{p}{)}
        \PY{n}{df} \PY{o}{=} \PY{n}{response\PYZus{}to\PYZus{}inputs}\PY{p}{(}\PY{n}{num}\PY{p}{,} \PY{n}{den}\PY{p}{,} \PY{p}{[}\PY{n}{step}\PY{p}{]}\PY{p}{,} \PY{p}{[}\PY{l+s+s2}{\PYZdq{}}\PY{l+s+s2}{Step}\PY{l+s+s2}{\PYZdq{}}\PY{p}{]}\PY{p}{,} \PY{n}{t}\PY{p}{)}
        \PY{n}{create\PYZus{}plots}\PY{p}{(}\PY{n}{df}\PY{p}{,} \PY{k+kc}{True}\PY{p}{)}\PY{p}{;}
\end{Verbatim}

    \begin{Verbatim}[commandchars=\\\{\}]
/usr/local/lib/python3.7/site-packages/scipy/signal/filter\_design.py:1551: BadCoefficients: Badly conditioned filter coefficients (numerator): the results may be meaningless
  "results may be meaningless", BadCoefficients)

    \end{Verbatim}

    \begin{center}
    \adjustimage{max size={0.9\linewidth}{0.9\paperheight}}{output_16_1.png}
    \end{center}
    { \hspace*{\fill} \\}
    
    First, I'll test a compensation network with just an increased gain. The
steady state error of the unity response is determined by:

\[e_{ss}^{step}=\frac{1}{1+lim_{s\to 0}H_0(s)}\]

\[e_{ss}^{step}=\frac{1}{1+lim_{s\to 0}(\frac{5\cdot 10^6}{s^2+6\cdot 10^3s+5\cdot 10^6})}\]

\[e_{ss}^{step}=\frac{1}{1+1}=0.5\]

In order to get this below one percent (0.01), I'll choose a gain
according to the following:

\[e_{ss}^{step}=\frac{1}{1+lim_{s\to 0}(K\cdot \frac{5\cdot 10^6}{s^2+6\cdot 10^3s+5\cdot 10^6})}=0.01\]

\[e_{ss}^{step}=\frac{1}{1+K}=0.01\]

\[1=0.01+0.01\cdot K \rightarrow 0.99=0.01\cdot K\]

\[K=99\]

Therefore a gain of \(100\) would produce the desired reduction in error
(+ some wiggle room).

    \begin{Verbatim}[commandchars=\\\{\}]
{\color{incolor}In [{\color{incolor}10}]:} \PY{n}{comp\PYZus{}network\PYZus{}num} \PY{o}{=} \PY{p}{[}\PY{l+m+mi}{100}\PY{p}{]}
         \PY{n}{comp\PYZus{}network\PYZus{}den} \PY{o}{=} \PY{p}{[}\PY{l+m+mi}{1}\PY{p}{]}
          
         \PY{n}{t} \PY{o}{=} \PY{n}{np}\PY{o}{.}\PY{n}{arange}\PY{p}{(}\PY{l+m+mi}{0}\PY{p}{,} \PY{l+m+mf}{0.1}\PY{p}{,} \PY{l+m+mf}{0.00001}\PY{p}{)}
         \PY{n}{df} \PY{o}{=} \PY{n}{response\PYZus{}to\PYZus{}inputs}\PY{p}{(}\PY{n}{num}\PY{p}{,} \PY{n}{den}\PY{p}{,} \PY{p}{[}\PY{n}{step}\PY{p}{]}\PY{p}{,} \PY{p}{[}\PY{l+s+s2}{\PYZdq{}}\PY{l+s+s2}{Step}\PY{l+s+s2}{\PYZdq{}}\PY{p}{]}\PY{p}{,} \PY{n}{t}\PY{p}{,} \PY{n}{comp\PYZus{}network\PYZus{}num}\PY{p}{,} \PY{n}{comp\PYZus{}network\PYZus{}den}\PY{p}{)}
         \PY{n}{create\PYZus{}plots}\PY{p}{(}\PY{n}{df}\PY{p}{,} \PY{k+kc}{True}\PY{p}{)}\PY{p}{;}
\end{Verbatim}

    \begin{center}
    \adjustimage{max size={0.9\linewidth}{0.9\paperheight}}{output_18_0.png}
    \end{center}
    { \hspace*{\fill} \\}
    
    This seems to satisfy the problem requirements for the steady-state
error, but let's look at the values at 5 milliseconds, just to be sure.

    \begin{Verbatim}[commandchars=\\\{\}]
{\color{incolor}In [{\color{incolor}11}]:} \PY{n}{display}\PY{p}{(}\PY{n}{df}\PY{p}{[}\PY{p}{(}\PY{n}{df}\PY{p}{[}\PY{l+s+s2}{\PYZdq{}}\PY{l+s+s2}{Time}\PY{l+s+s2}{\PYZdq{}}\PY{p}{]} \PY{o}{==} \PY{l+m+mi}{5} \PY{o}{/} \PY{l+m+mi}{1000}\PY{p}{)} \PY{o}{\PYZam{}} \PY{p}{(}\PY{n}{df}\PY{p}{[}\PY{l+s+s2}{\PYZdq{}}\PY{l+s+s2}{Error}\PY{l+s+s2}{\PYZdq{}}\PY{p}{]} \PY{o}{==} \PY{l+s+s2}{\PYZdq{}}\PY{l+s+s2}{Error}\PY{l+s+s2}{\PYZdq{}}\PY{p}{)}\PY{p}{]}\PY{p}{)}
\end{Verbatim}

    
    \begin{verbatim}
        Time     Value  Kind          Name  Error
20500  0.005  0.009901  Step  Step - Error  Error
    \end{verbatim}

    
    As we can see, at 5 milliseconds, the step error for this new transfer
function is below 1\%. Now, the overshoot needs to be dealt with.

Let's take a look at the phase margin for this new system, to identify
how much phase offset we need to add.

    \begin{Verbatim}[commandchars=\\\{\}]
{\color{incolor}In [{\color{incolor}13}]:} \PY{n}{df}\PY{p}{,} \PY{n}{phase\PYZus{}margin}\PY{p}{,} \PY{n}{crossover\PYZus{}w} \PY{o}{=} \PY{n}{magnitude\PYZus{}phase\PYZus{}response}\PY{p}{(}\PY{n}{num}\PY{p}{,} \PY{n}{den}\PY{p}{,} \PY{p}{[}\PY{l+m+mf}{0.1}\PY{p}{,} \PY{l+m+mi}{10} \PY{o}{*}\PY{o}{*} \PY{l+m+mi}{6}\PY{p}{]}\PY{p}{,} \PY{l+m+mi}{10}\PY{p}{,}
                                                                  \PY{n}{comp\PYZus{}network\PYZus{}num}\PY{p}{,} \PY{n}{comp\PYZus{}network\PYZus{}den}\PY{p}{)}
         \PY{n+nb}{print} \PY{p}{(}\PY{l+s+s2}{\PYZdq{}}\PY{l+s+s2}{Crossover at }\PY{l+s+si}{\PYZob{}:.3f\PYZcb{}}\PY{l+s+s2}{ (rad/s) has a phase\PYZhy{}margin of }\PY{l+s+si}{\PYZob{}:.3f\PYZcb{}}\PY{l+s+s2}{ degrees}\PY{l+s+s2}{\PYZdq{}}\PY{o}{.}\PY{n}{format}\PY{p}{(}\PY{n}{crossover\PYZus{}w}\PY{p}{,} \PY{n}{phase\PYZus{}margin}\PY{p}{)}\PY{p}{)}
\end{Verbatim}

    \begin{Verbatim}[commandchars=\\\{\}]
Crossover at 20803.534 (rad/s) has a phase-margin of 16.266 degrees

    \end{Verbatim}

    \begin{center}
    \adjustimage{max size={0.9\linewidth}{0.9\paperheight}}{output_22_1.png}
    \end{center}
    { \hspace*{\fill} \\}
    
    This phase margin, of only \(16.266\) degrees, is clearly the reason we
have such an overshoot when it comes to the step response. To deal with
this, let's add a phase lead compensation. I chose a desired phase of
\(100^o\), since we want \textbf{no overshoot whatsoever}.

    \begin{Verbatim}[commandchars=\\\{\}]
{\color{incolor}In [{\color{incolor}14}]:} \PY{n}{z}\PY{p}{,} \PY{n}{p} \PY{o}{=} \PY{n}{compute\PYZus{}phase\PYZus{}lead}\PY{p}{(}\PY{n}{df}\PY{p}{,} \PY{l+m+mi}{100} \PY{o}{\PYZhy{}} \PY{n}{phase\PYZus{}margin}\PY{p}{)}
         \PY{n+nb}{print} \PY{p}{(}\PY{l+s+s2}{\PYZdq{}}\PY{l+s+s2}{Zero: }\PY{l+s+si}{\PYZob{}:.2f\PYZcb{}}\PY{l+s+s2}{ (rad/s)}\PY{l+s+se}{\PYZbs{}n}\PY{l+s+s2}{Pole: }\PY{l+s+si}{\PYZob{}:.2f\PYZcb{}}\PY{l+s+s2}{ (rad/s)}\PY{l+s+s2}{\PYZdq{}}\PY{o}{.}\PY{n}{format}\PY{p}{(}\PY{n}{z}\PY{p}{,} \PY{n}{p}\PY{p}{)}\PY{p}{)}
\end{Verbatim}

    \begin{Verbatim}[commandchars=\\\{\}]
Zero: 5228.13 (rad/s)
Pole: 1744826.86 (rad/s)

    \end{Verbatim}

    After computing the calculated zero and pole of the compensation
network, the final transfer function is:

\[G_c(s)=100\cdot \frac{1744826.86}{5228.13}\cdot \frac{s+5228.13}{s+1744826.86}\]

Now, I'll take a look at the final step response of this system - with
the compensation network added.

    \begin{Verbatim}[commandchars=\\\{\}]
{\color{incolor}In [{\color{incolor}15}]:} \PY{n}{comp\PYZus{}network\PYZus{}num} \PY{o}{=} \PY{n}{np}\PY{o}{.}\PY{n}{multiply}\PY{p}{(}\PY{l+m+mi}{100} \PY{o}{*} \PY{n}{p} \PY{o}{/} \PY{n}{z}\PY{p}{,} \PY{p}{[}\PY{l+m+mi}{1}\PY{p}{,} \PY{n}{z}\PY{p}{]}\PY{p}{)}
         \PY{n}{comp\PYZus{}network\PYZus{}den} \PY{o}{=} \PY{p}{[}\PY{l+m+mi}{1}\PY{p}{,} \PY{n}{p}\PY{p}{]}
          
         \PY{n}{t} \PY{o}{=} \PY{n}{np}\PY{o}{.}\PY{n}{arange}\PY{p}{(}\PY{l+m+mi}{0}\PY{p}{,} \PY{l+m+mf}{0.01}\PY{p}{,} \PY{l+m+mf}{0.000001}\PY{p}{)}
         \PY{n}{df} \PY{o}{=} \PY{n}{response\PYZus{}to\PYZus{}inputs}\PY{p}{(}\PY{n}{num}\PY{p}{,} \PY{n}{den}\PY{p}{,} \PY{p}{[}\PY{n}{step}\PY{p}{]}\PY{p}{,} \PY{p}{[}\PY{l+s+s2}{\PYZdq{}}\PY{l+s+s2}{Step}\PY{l+s+s2}{\PYZdq{}}\PY{p}{]}\PY{p}{,} \PY{n}{t}\PY{p}{,} \PY{n}{comp\PYZus{}network\PYZus{}num}\PY{p}{,} \PY{n}{comp\PYZus{}network\PYZus{}den}\PY{p}{)}
         \PY{n}{create\PYZus{}plots}\PY{p}{(}\PY{n}{df}\PY{p}{,} \PY{k+kc}{True}\PY{p}{)}\PY{p}{;}
\end{Verbatim}

    \begin{center}
    \adjustimage{max size={0.9\linewidth}{0.9\paperheight}}{output_26_0.png}
    \end{center}
    { \hspace*{\fill} \\}
    
    This appears to do the job. There is no overshoot, and the response to
the unit step function definitely reaches equilibrium by 5 milliseconds.
However, just to verify, here is the error at this time:

    \begin{Verbatim}[commandchars=\\\{\}]
{\color{incolor}In [{\color{incolor}16}]:} \PY{n}{df}\PY{p}{[}\PY{p}{(}\PY{n}{df}\PY{p}{[}\PY{l+s+s2}{\PYZdq{}}\PY{l+s+s2}{Error}\PY{l+s+s2}{\PYZdq{}}\PY{p}{]} \PY{o}{==} \PY{l+s+s2}{\PYZdq{}}\PY{l+s+s2}{Error}\PY{l+s+s2}{\PYZdq{}}\PY{p}{)} \PY{o}{\PYZam{}} \PY{p}{(}\PY{n}{df}\PY{p}{[}\PY{l+s+s2}{\PYZdq{}}\PY{l+s+s2}{Time}\PY{l+s+s2}{\PYZdq{}}\PY{p}{]} \PY{o}{==} \PY{l+m+mi}{5} \PY{o}{/} \PY{l+m+mi}{1000}\PY{p}{)}\PY{p}{]}
\end{Verbatim}

\begin{Verbatim}[commandchars=\\\{\}]
{\color{outcolor}Out[{\color{outcolor}16}]:}         Time     Value  Kind          Name  Error
         25000  0.005  0.009901  Step  Step - Error  Error
\end{Verbatim}
            
    \hypertarget{problem-b}{%
\subsection{Problem B}\label{problem-b}}

\hypertarget{part-i}{%
\subsubsection{Part i}\label{part-i}}

There is clearly a low pass filter with two poles. The reason we know
this is because the phase settles out to -180 degrees, and does not
appear to increase or have any zeros of the plotted range.

My first estimation is that the dominant pole exists at \(10^0\), or
\(1\). I am making this estimation because the magnitude response begins
to decrease at this point, and the phase reaches \(-45^o\) at this
frequency. The next pole clearly exists at \(10^1\), or \(10\). Once
again, we can see the magnitude transition from a \(-20 \frac{dB}{dec}\)
slope to \(-40 \frac{dB}{dec}\) (suggesting a pole) - \textbf{and} and
frequency is at \(-135^o\) at this point.

That results in the following approximated transfer function:

\[H(s)=K\cdot \frac{1}{(s+1)(s+10)}\]

However, we see that at near zero frequencies (\(w\approx 0\)), the
magnitude of the response is 20dB. In order to achieve this gain, K
needs to be approximately 100 - we know this because each gain of 10
corresponds to 20dB, and the starting gain of this equation is -20dB.

Thus, the following is the final transfer function:

\[H(s)=\frac{100}{(s+1)(s+10)}\]

    \begin{Verbatim}[commandchars=\\\{\}]
{\color{incolor}In [{\color{incolor}17}]:} \PY{n}{num} \PY{o}{=} \PY{p}{[}\PY{l+m+mi}{100}\PY{p}{]}
         \PY{n}{den} \PY{o}{=} \PY{n}{convolve\PYZus{}all}\PY{p}{(}\PY{p}{[}\PY{p}{[}\PY{l+m+mi}{1}\PY{p}{,} \PY{l+m+mi}{1}\PY{p}{]}\PY{p}{,} \PY{p}{[}\PY{l+m+mi}{1}\PY{p}{,} \PY{l+m+mi}{10}\PY{p}{]}\PY{p}{]}\PY{p}{)}
         
         \PY{n}{df}\PY{p}{,} \PY{n}{m}\PY{p}{,} \PY{n}{w} \PY{o}{=} \PY{n}{magnitude\PYZus{}phase\PYZus{}response}\PY{p}{(}\PY{n}{num}\PY{p}{,} \PY{n}{den}\PY{p}{,} \PY{p}{[}\PY{l+m+mi}{10} \PY{o}{*}\PY{o}{*} \PY{o}{\PYZhy{}}\PY{l+m+mi}{1}\PY{p}{,} \PY{l+m+mi}{10} \PY{o}{*}\PY{o}{*} \PY{l+m+mi}{2}\PY{p}{]}\PY{p}{,} \PY{o}{.}\PY{l+m+mi}{01}\PY{p}{)}
         \PY{n+nb}{print} \PY{p}{(}\PY{l+s+s2}{\PYZdq{}}\PY{l+s+s2}{Crossover at }\PY{l+s+si}{\PYZob{}:.3f\PYZcb{}}\PY{l+s+s2}{ (rad/s) has a phase\PYZhy{}margin of }\PY{l+s+si}{\PYZob{}:.3f\PYZcb{}}\PY{l+s+s2}{ degrees}\PY{l+s+s2}{\PYZdq{}}\PY{o}{.}\PY{n}{format}\PY{p}{(}\PY{n}{w}\PY{p}{,} \PY{n}{m}\PY{p}{)}\PY{p}{)}
         \PY{n}{g} \PY{o}{=} \PY{n}{create\PYZus{}plots}\PY{p}{(}\PY{n}{df}\PY{p}{)}
         \PY{k}{for} \PY{n}{ax} \PY{o+ow}{in} \PY{n}{g}\PY{o}{.}\PY{n}{axes}\PY{o}{.}\PY{n}{flatten}\PY{p}{(}\PY{p}{)}\PY{p}{:}
             \PY{n}{ax}\PY{o}{.}\PY{n}{tick\PYZus{}params}\PY{p}{(}\PY{n}{labelbottom}\PY{o}{=}\PY{k+kc}{True}\PY{p}{)}
\end{Verbatim}

    \begin{Verbatim}[commandchars=\\\{\}]
Crossover at 7.173 (rad/s) has a phase-margin of 62.286 degrees

    \end{Verbatim}

    \begin{center}
    \adjustimage{max size={0.9\linewidth}{0.9\paperheight}}{output_30_1.png}
    \end{center}
    { \hspace*{\fill} \\}
    
    Which seems to match the given Bode plot.

\hypertarget{part-ii}{%
\subsubsection{Part ii}\label{part-ii}}

Let's start by looking at the position and velocity response.

    \begin{Verbatim}[commandchars=\\\{\}]
{\color{incolor}In [{\color{incolor}18}]:} \PY{n}{t} \PY{o}{=} \PY{n}{np}\PY{o}{.}\PY{n}{arange}\PY{p}{(}\PY{l+m+mi}{0}\PY{p}{,} \PY{l+m+mi}{6}\PY{p}{,} \PY{l+m+mf}{0.001}\PY{p}{)}
         \PY{n}{df2} \PY{o}{=} \PY{n}{response\PYZus{}to\PYZus{}inputs}\PY{p}{(}\PY{n}{num}\PY{p}{,} \PY{n}{den}\PY{p}{,} \PY{p}{[}\PY{n}{step}\PY{p}{,} \PY{n}{ramp}\PY{p}{]}\PY{p}{,} \PY{p}{[}\PY{l+s+s2}{\PYZdq{}}\PY{l+s+s2}{Position}\PY{l+s+s2}{\PYZdq{}}\PY{p}{,} \PY{l+s+s2}{\PYZdq{}}\PY{l+s+s2}{Velocity}\PY{l+s+s2}{\PYZdq{}}\PY{p}{]}\PY{p}{,} \PY{n}{t}\PY{p}{)}
         \PY{n}{create\PYZus{}plots}\PY{p}{(}\PY{n}{df2}\PY{p}{,} \PY{k+kc}{True}\PY{p}{)}\PY{p}{;}
\end{Verbatim}

    \begin{center}
    \adjustimage{max size={0.9\linewidth}{0.9\paperheight}}{output_32_0.png}
    \end{center}
    { \hspace*{\fill} \\}
    
    I can see that the position error seems to stabilize to about 10\%, but
the velocity error is unbounded. In order to address this, I need to
institute a compensation network that reduces the steady-state velocity
constant to a number less than 0.1.

To accomplish this, I will add a phase-lead offset, with the addition of
another pole at \(\omega =0\), in order to change the velocity error to
a bounded value. The general form of this will be:

\[G_c(s)=\frac{\omega _p}{\omega _z}\cdot \frac{s + \omega _z}{s\cdot(s+\omega _p)}\]

    \begin{Verbatim}[commandchars=\\\{\}]
{\color{incolor}In [{\color{incolor}19}]:} \PY{n}{z}\PY{p}{,} \PY{n}{p} \PY{o}{=} \PY{n}{compute\PYZus{}phase\PYZus{}lead}\PY{p}{(}\PY{n}{df}\PY{p}{,} \PY{l+m+mi}{100}\PY{o}{\PYZhy{}}\PY{n}{m}\PY{p}{)}
         \PY{n+nb}{print} \PY{p}{(}\PY{l+s+s2}{\PYZdq{}}\PY{l+s+s2}{The new frequency of the zero is }\PY{l+s+si}{\PYZob{}:.2f\PYZcb{}}\PY{l+s+s2}{ (rad/s)}\PY{l+s+se}{\PYZbs{}}
         \PY{l+s+s2}{       }\PY{l+s+se}{\PYZbs{}n}\PY{l+s+s2}{The new frequency of the pole is }\PY{l+s+si}{\PYZob{}:.2f\PYZcb{}}\PY{l+s+s2}{ (rad/s)}\PY{l+s+s2}{\PYZdq{}}\PY{o}{.}\PY{n}{format}\PY{p}{(}\PY{n}{z}\PY{p}{,} \PY{n}{p}\PY{p}{)}\PY{p}{)}
\end{Verbatim}

    \begin{Verbatim}[commandchars=\\\{\}]
The new frequency of the zero is 6.19 (rad/s)       
The new frequency of the pole is 25.71 (rad/s)

    \end{Verbatim}

    \begin{Verbatim}[commandchars=\\\{\}]
{\color{incolor}In [{\color{incolor}20}]:} \PY{n}{comp\PYZus{}num} \PY{o}{=} \PY{n}{np}\PY{o}{.}\PY{n}{multiply}\PY{p}{(}\PY{n}{p} \PY{o}{/} \PY{n}{z}\PY{p}{,} \PY{p}{[}\PY{l+m+mi}{1}\PY{p}{,} \PY{n}{z}\PY{p}{]}\PY{p}{)} \PY{c+c1}{\PYZsh{} Was 7.5}
         \PY{n}{comp\PYZus{}den} \PY{o}{=} \PY{n}{convolve\PYZus{}all}\PY{p}{(}\PY{p}{[}\PY{p}{[}\PY{l+m+mi}{1}\PY{p}{,} \PY{l+m+mi}{0}\PY{p}{]}\PY{p}{,} \PY{p}{[}\PY{l+m+mi}{1}\PY{p}{,} \PY{n}{p}\PY{p}{]}\PY{p}{]}\PY{p}{)}
         
         \PY{n}{t} \PY{o}{=} \PY{n}{np}\PY{o}{.}\PY{n}{arange}\PY{p}{(}\PY{l+m+mi}{0}\PY{p}{,} \PY{l+m+mi}{6}\PY{p}{,} \PY{l+m+mf}{0.001}\PY{p}{)}
         \PY{n}{df} \PY{o}{=} \PY{n}{response\PYZus{}to\PYZus{}inputs}\PY{p}{(}\PY{n}{num}\PY{p}{,} \PY{n}{den}\PY{p}{,} \PY{p}{[}\PY{n}{step}\PY{p}{,} \PY{n}{ramp}\PY{p}{]}\PY{p}{,} \PY{p}{[}\PY{l+s+s2}{\PYZdq{}}\PY{l+s+s2}{Position}\PY{l+s+s2}{\PYZdq{}}\PY{p}{,} \PY{l+s+s2}{\PYZdq{}}\PY{l+s+s2}{Velocity}\PY{l+s+s2}{\PYZdq{}}\PY{p}{]}\PY{p}{,} \PY{n}{t}\PY{p}{,} \PY{n}{comp\PYZus{}num}\PY{p}{,} \PY{n}{comp\PYZus{}den}\PY{p}{)}
         \PY{n}{create\PYZus{}plots}\PY{p}{(}\PY{n}{df}\PY{p}{,} \PY{k+kc}{True}\PY{p}{)}\PY{p}{;}
\end{Verbatim}

    \begin{center}
    \adjustimage{max size={0.9\linewidth}{0.9\paperheight}}{output_35_0.png}
    \end{center}
    { \hspace*{\fill} \\}
    
    This seems to have done the trick. Just to verify, let's look at the
error of the response function at \(t=5\) seconds.

    \begin{Verbatim}[commandchars=\\\{\}]
{\color{incolor}In [{\color{incolor}21}]:} \PY{n}{df}\PY{p}{[}\PY{p}{(}\PY{n}{df}\PY{p}{[}\PY{l+s+s2}{\PYZdq{}}\PY{l+s+s2}{Time}\PY{l+s+s2}{\PYZdq{}}\PY{p}{]} \PY{o}{==} \PY{l+m+mi}{5}\PY{p}{)} \PY{o}{\PYZam{}} \PY{p}{(}\PY{n}{df}\PY{p}{[}\PY{l+s+s2}{\PYZdq{}}\PY{l+s+s2}{Error}\PY{l+s+s2}{\PYZdq{}}\PY{p}{]} \PY{o}{==} \PY{l+s+s2}{\PYZdq{}}\PY{l+s+s2}{Error}\PY{l+s+s2}{\PYZdq{}}\PY{p}{)}\PY{p}{]}
\end{Verbatim}

\begin{Verbatim}[commandchars=\\\{\}]
{\color{outcolor}Out[{\color{outcolor}21}]:}        Time     Value      Kind              Name  Error
         17000   5.0 -0.050009  Position  Position - Error  Error
         35000   5.0  0.098880  Velocity  Velocity - Error  Error
\end{Verbatim}
            
    As requested, the error of both the position and velocty have been
reduced to below 10\% after 5 seconds.


    % Add a bibliography block to the postdoc
    
    
    
    \end{document}
