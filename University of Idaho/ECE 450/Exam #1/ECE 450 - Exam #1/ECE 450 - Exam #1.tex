\documentclass[11pt]{article}
\author{Collin Heist}
    \usepackage[T1]{fontenc}
    % Nicer default font (+ math font) than Computer Modern for most use cases
    \usepackage{mathpazo}
    \usepackage{amsmath}

    % Basic figure setup, for now with no caption control since it's done
    % automatically by Pandoc (which extracts ![](path) syntax from Markdown).
    \usepackage{graphicx}
    % We will generate all images so they have a width \maxwidth. This means
    % that they will get their normal width if they fit onto the page, but
    % are scaled down if they would overflow the margins.
    \makeatletter
    \def\maxwidth{\ifdim\Gin@nat@width>\linewidth\linewidth
    \else\Gin@nat@width\fi}
    \makeatother
    \let\Oldincludegraphics\includegraphics
    % Set max figure width to be 80% of text width, for now hardcoded.
    \renewcommand{\includegraphics}[1]{\Oldincludegraphics[width=.8\maxwidth]{#1}}
    % Ensure that by default, figures have no caption (until we provide a
    % proper Figure object with a Caption API and a way to capture that
    % in the conversion process - todo).
    \usepackage{caption}
    \DeclareCaptionLabelFormat{nolabel}{}
    \captionsetup{labelformat=nolabel}

    \usepackage{adjustbox} % Used to constrain images to a maximum size 
    \usepackage{xcolor} % Allow colors to be defined
    \usepackage{enumerate} % Needed for markdown enumerations to work
    \usepackage{geometry} % Used to adjust the document margins
    \usepackage{amsmath} % Equations
    \usepackage{amssymb} % Equations
    \usepackage{textcomp} % defines textquotesingle
    % Hack from http://tex.stackexchange.com/a/47451/13684:
    \AtBeginDocument{%
        \def\PYZsq{\textquotesingle}% Upright quotes in Pygmentized code
    }
    \usepackage{upquote} % Upright quotes for verbatim code
    \usepackage{eurosym} % defines \euro
    \usepackage[mathletters]{ucs} % Extended unicode (utf-8) support
    \usepackage[utf8x]{inputenc} % Allow utf-8 characters in the tex document
    \usepackage{fancyvrb} % verbatim replacement that allows latex
    \usepackage{grffile} % extends the file name processing of package graphics 
                         % to support a larger range 
    % The hyperref package gives us a pdf with properly built
    % internal navigation ('pdf bookmarks' for the table of contents,
    % internal cross-reference links, web links for URLs, etc.)
    \usepackage{hyperref}
    \usepackage{longtable} % longtable support required by pandoc >1.10
    \usepackage{booktabs}  % table support for pandoc > 1.12.2
    \usepackage[inline]{enumitem} % IRkernel/repr support (it uses the enumerate* environment)
    \usepackage[normalem]{ulem} % ulem is needed to support strikethroughs (\sout)
                                % normalem makes italics be italics, not underlines
    \usepackage{mathrsfs}
    

    
    
    % Colors for the hyperref package
    \definecolor{urlcolor}{rgb}{0,.145,.698}
    \definecolor{linkcolor}{rgb}{.71,0.21,0.01}
    \definecolor{citecolor}{rgb}{.12,.54,.11}

    % ANSI colors
    \definecolor{ansi-black}{HTML}{3E424D}
    \definecolor{ansi-black-intense}{HTML}{282C36}
    \definecolor{ansi-red}{HTML}{E75C58}
    \definecolor{ansi-red-intense}{HTML}{B22B31}
    \definecolor{ansi-green}{HTML}{00A250}
    \definecolor{ansi-green-intense}{HTML}{007427}
    \definecolor{ansi-yellow}{HTML}{DDB62B}
    \definecolor{ansi-yellow-intense}{HTML}{B27D12}
    \definecolor{ansi-blue}{HTML}{208FFB}
    \definecolor{ansi-blue-intense}{HTML}{0065CA}
    \definecolor{ansi-magenta}{HTML}{D160C4}
    \definecolor{ansi-magenta-intense}{HTML}{A03196}
    \definecolor{ansi-cyan}{HTML}{60C6C8}
    \definecolor{ansi-cyan-intense}{HTML}{258F8F}
    \definecolor{ansi-white}{HTML}{C5C1B4}
    \definecolor{ansi-white-intense}{HTML}{A1A6B2}
    \definecolor{ansi-default-inverse-fg}{HTML}{FFFFFF}
    \definecolor{ansi-default-inverse-bg}{HTML}{000000}

    % commands and environments needed by pandoc snippets
    % extracted from the output of `pandoc -s`
    \providecommand{\tightlist}{%
      \setlength{\itemsep}{0pt}\setlength{\parskip}{0pt}}
    \DefineVerbatimEnvironment{Highlighting}{Verbatim}{commandchars=\\\{\}}
    % Add ',fontsize=\small' for more characters per line
    \newenvironment{Shaded}{}{}
    \newcommand{\KeywordTok}[1]{\textcolor[rgb]{0.00,0.44,0.13}{\textbf{{#1}}}}
    \newcommand{\DataTypeTok}[1]{\textcolor[rgb]{0.56,0.13,0.00}{{#1}}}
    \newcommand{\DecValTok}[1]{\textcolor[rgb]{0.25,0.63,0.44}{{#1}}}
    \newcommand{\BaseNTok}[1]{\textcolor[rgb]{0.25,0.63,0.44}{{#1}}}
    \newcommand{\FloatTok}[1]{\textcolor[rgb]{0.25,0.63,0.44}{{#1}}}
    \newcommand{\CharTok}[1]{\textcolor[rgb]{0.25,0.44,0.63}{{#1}}}
    \newcommand{\StringTok}[1]{\textcolor[rgb]{0.25,0.44,0.63}{{#1}}}
    \newcommand{\CommentTok}[1]{\textcolor[rgb]{0.38,0.63,0.69}{\textit{{#1}}}}
    \newcommand{\OtherTok}[1]{\textcolor[rgb]{0.00,0.44,0.13}{{#1}}}
    \newcommand{\AlertTok}[1]{\textcolor[rgb]{1.00,0.00,0.00}{\textbf{{#1}}}}
    \newcommand{\FunctionTok}[1]{\textcolor[rgb]{0.02,0.16,0.49}{{#1}}}
    \newcommand{\RegionMarkerTok}[1]{{#1}}
    \newcommand{\ErrorTok}[1]{\textcolor[rgb]{1.00,0.00,0.00}{\textbf{{#1}}}}
    \newcommand{\NormalTok}[1]{{#1}}
    
    % Additional commands for more recent versions of Pandoc
    \newcommand{\ConstantTok}[1]{\textcolor[rgb]{0.53,0.00,0.00}{{#1}}}
    \newcommand{\SpecialCharTok}[1]{\textcolor[rgb]{0.25,0.44,0.63}{{#1}}}
    \newcommand{\VerbatimStringTok}[1]{\textcolor[rgb]{0.25,0.44,0.63}{{#1}}}
    \newcommand{\SpecialStringTok}[1]{\textcolor[rgb]{0.73,0.40,0.53}{{#1}}}
    \newcommand{\ImportTok}[1]{{#1}}
    \newcommand{\DocumentationTok}[1]{\textcolor[rgb]{0.73,0.13,0.13}{\textit{{#1}}}}
    \newcommand{\AnnotationTok}[1]{\textcolor[rgb]{0.38,0.63,0.69}{\textbf{\textit{{#1}}}}}
    \newcommand{\CommentVarTok}[1]{\textcolor[rgb]{0.38,0.63,0.69}{\textbf{\textit{{#1}}}}}
    \newcommand{\VariableTok}[1]{\textcolor[rgb]{0.10,0.09,0.49}{{#1}}}
    \newcommand{\ControlFlowTok}[1]{\textcolor[rgb]{0.00,0.44,0.13}{\textbf{{#1}}}}
    \newcommand{\OperatorTok}[1]{\textcolor[rgb]{0.40,0.40,0.40}{{#1}}}
    \newcommand{\BuiltInTok}[1]{{#1}}
    \newcommand{\ExtensionTok}[1]{{#1}}
    \newcommand{\PreprocessorTok}[1]{\textcolor[rgb]{0.74,0.48,0.00}{{#1}}}
    \newcommand{\AttributeTok}[1]{\textcolor[rgb]{0.49,0.56,0.16}{{#1}}}
    \newcommand{\InformationTok}[1]{\textcolor[rgb]{0.38,0.63,0.69}{\textbf{\textit{{#1}}}}}
    \newcommand{\WarningTok}[1]{\textcolor[rgb]{0.38,0.63,0.69}{\textbf{\textit{{#1}}}}}
    
    
    % Define a nice break command that doesn't care if a line doesn't already
    % exist.
    \def\br{\hspace*{\fill} \\* }
    % Math Jax compatibility definitions
    \def\gt{>}
    \def\lt{<}
    \let\Oldtex\TeX
    \let\Oldlatex\LaTeX
    \renewcommand{\TeX}{\textrm{\Oldtex}}
    \renewcommand{\LaTeX}{\textrm{\Oldlatex}}
    % Document parameters
    % Document title
    \title{ECE 450 - Exam \#1}
    
    
    
    
    

    % Pygments definitions
    
\makeatletter
\def\PY@reset{\let\PY@it=\relax \let\PY@bf=\relax%
    \let\PY@ul=\relax \let\PY@tc=\relax%
    \let\PY@bc=\relax \let\PY@ff=\relax}
\def\PY@tok#1{\csname PY@tok@#1\endcsname}
\def\PY@toks#1+{\ifx\relax#1\empty\else%
    \PY@tok{#1}\expandafter\PY@toks\fi}
\def\PY@do#1{\PY@bc{\PY@tc{\PY@ul{%
    \PY@it{\PY@bf{\PY@ff{#1}}}}}}}
\def\PY#1#2{\PY@reset\PY@toks#1+\relax+\PY@do{#2}}

\expandafter\def\csname PY@tok@w\endcsname{\def\PY@tc##1{\textcolor[rgb]{0.73,0.73,0.73}{##1}}}
\expandafter\def\csname PY@tok@c\endcsname{\let\PY@it=\textit\def\PY@tc##1{\textcolor[rgb]{0.25,0.50,0.50}{##1}}}
\expandafter\def\csname PY@tok@cp\endcsname{\def\PY@tc##1{\textcolor[rgb]{0.74,0.48,0.00}{##1}}}
\expandafter\def\csname PY@tok@k\endcsname{\let\PY@bf=\textbf\def\PY@tc##1{\textcolor[rgb]{0.00,0.50,0.00}{##1}}}
\expandafter\def\csname PY@tok@kp\endcsname{\def\PY@tc##1{\textcolor[rgb]{0.00,0.50,0.00}{##1}}}
\expandafter\def\csname PY@tok@kt\endcsname{\def\PY@tc##1{\textcolor[rgb]{0.69,0.00,0.25}{##1}}}
\expandafter\def\csname PY@tok@o\endcsname{\def\PY@tc##1{\textcolor[rgb]{0.40,0.40,0.40}{##1}}}
\expandafter\def\csname PY@tok@ow\endcsname{\let\PY@bf=\textbf\def\PY@tc##1{\textcolor[rgb]{0.67,0.13,1.00}{##1}}}
\expandafter\def\csname PY@tok@nb\endcsname{\def\PY@tc##1{\textcolor[rgb]{0.00,0.50,0.00}{##1}}}
\expandafter\def\csname PY@tok@nf\endcsname{\def\PY@tc##1{\textcolor[rgb]{0.00,0.00,1.00}{##1}}}
\expandafter\def\csname PY@tok@nc\endcsname{\let\PY@bf=\textbf\def\PY@tc##1{\textcolor[rgb]{0.00,0.00,1.00}{##1}}}
\expandafter\def\csname PY@tok@nn\endcsname{\let\PY@bf=\textbf\def\PY@tc##1{\textcolor[rgb]{0.00,0.00,1.00}{##1}}}
\expandafter\def\csname PY@tok@ne\endcsname{\let\PY@bf=\textbf\def\PY@tc##1{\textcolor[rgb]{0.82,0.25,0.23}{##1}}}
\expandafter\def\csname PY@tok@nv\endcsname{\def\PY@tc##1{\textcolor[rgb]{0.10,0.09,0.49}{##1}}}
\expandafter\def\csname PY@tok@no\endcsname{\def\PY@tc##1{\textcolor[rgb]{0.53,0.00,0.00}{##1}}}
\expandafter\def\csname PY@tok@nl\endcsname{\def\PY@tc##1{\textcolor[rgb]{0.63,0.63,0.00}{##1}}}
\expandafter\def\csname PY@tok@ni\endcsname{\let\PY@bf=\textbf\def\PY@tc##1{\textcolor[rgb]{0.60,0.60,0.60}{##1}}}
\expandafter\def\csname PY@tok@na\endcsname{\def\PY@tc##1{\textcolor[rgb]{0.49,0.56,0.16}{##1}}}
\expandafter\def\csname PY@tok@nt\endcsname{\let\PY@bf=\textbf\def\PY@tc##1{\textcolor[rgb]{0.00,0.50,0.00}{##1}}}
\expandafter\def\csname PY@tok@nd\endcsname{\def\PY@tc##1{\textcolor[rgb]{0.67,0.13,1.00}{##1}}}
\expandafter\def\csname PY@tok@s\endcsname{\def\PY@tc##1{\textcolor[rgb]{0.73,0.13,0.13}{##1}}}
\expandafter\def\csname PY@tok@sd\endcsname{\let\PY@it=\textit\def\PY@tc##1{\textcolor[rgb]{0.73,0.13,0.13}{##1}}}
\expandafter\def\csname PY@tok@si\endcsname{\let\PY@bf=\textbf\def\PY@tc##1{\textcolor[rgb]{0.73,0.40,0.53}{##1}}}
\expandafter\def\csname PY@tok@se\endcsname{\let\PY@bf=\textbf\def\PY@tc##1{\textcolor[rgb]{0.73,0.40,0.13}{##1}}}
\expandafter\def\csname PY@tok@sr\endcsname{\def\PY@tc##1{\textcolor[rgb]{0.73,0.40,0.53}{##1}}}
\expandafter\def\csname PY@tok@ss\endcsname{\def\PY@tc##1{\textcolor[rgb]{0.10,0.09,0.49}{##1}}}
\expandafter\def\csname PY@tok@sx\endcsname{\def\PY@tc##1{\textcolor[rgb]{0.00,0.50,0.00}{##1}}}
\expandafter\def\csname PY@tok@m\endcsname{\def\PY@tc##1{\textcolor[rgb]{0.40,0.40,0.40}{##1}}}
\expandafter\def\csname PY@tok@gh\endcsname{\let\PY@bf=\textbf\def\PY@tc##1{\textcolor[rgb]{0.00,0.00,0.50}{##1}}}
\expandafter\def\csname PY@tok@gu\endcsname{\let\PY@bf=\textbf\def\PY@tc##1{\textcolor[rgb]{0.50,0.00,0.50}{##1}}}
\expandafter\def\csname PY@tok@gd\endcsname{\def\PY@tc##1{\textcolor[rgb]{0.63,0.00,0.00}{##1}}}
\expandafter\def\csname PY@tok@gi\endcsname{\def\PY@tc##1{\textcolor[rgb]{0.00,0.63,0.00}{##1}}}
\expandafter\def\csname PY@tok@gr\endcsname{\def\PY@tc##1{\textcolor[rgb]{1.00,0.00,0.00}{##1}}}
\expandafter\def\csname PY@tok@ge\endcsname{\let\PY@it=\textit}
\expandafter\def\csname PY@tok@gs\endcsname{\let\PY@bf=\textbf}
\expandafter\def\csname PY@tok@gp\endcsname{\let\PY@bf=\textbf\def\PY@tc##1{\textcolor[rgb]{0.00,0.00,0.50}{##1}}}
\expandafter\def\csname PY@tok@go\endcsname{\def\PY@tc##1{\textcolor[rgb]{0.53,0.53,0.53}{##1}}}
\expandafter\def\csname PY@tok@gt\endcsname{\def\PY@tc##1{\textcolor[rgb]{0.00,0.27,0.87}{##1}}}
\expandafter\def\csname PY@tok@err\endcsname{\def\PY@bc##1{\setlength{\fboxsep}{0pt}\fcolorbox[rgb]{1.00,0.00,0.00}{1,1,1}{\strut ##1}}}
\expandafter\def\csname PY@tok@kc\endcsname{\let\PY@bf=\textbf\def\PY@tc##1{\textcolor[rgb]{0.00,0.50,0.00}{##1}}}
\expandafter\def\csname PY@tok@kd\endcsname{\let\PY@bf=\textbf\def\PY@tc##1{\textcolor[rgb]{0.00,0.50,0.00}{##1}}}
\expandafter\def\csname PY@tok@kn\endcsname{\let\PY@bf=\textbf\def\PY@tc##1{\textcolor[rgb]{0.00,0.50,0.00}{##1}}}
\expandafter\def\csname PY@tok@kr\endcsname{\let\PY@bf=\textbf\def\PY@tc##1{\textcolor[rgb]{0.00,0.50,0.00}{##1}}}
\expandafter\def\csname PY@tok@bp\endcsname{\def\PY@tc##1{\textcolor[rgb]{0.00,0.50,0.00}{##1}}}
\expandafter\def\csname PY@tok@fm\endcsname{\def\PY@tc##1{\textcolor[rgb]{0.00,0.00,1.00}{##1}}}
\expandafter\def\csname PY@tok@vc\endcsname{\def\PY@tc##1{\textcolor[rgb]{0.10,0.09,0.49}{##1}}}
\expandafter\def\csname PY@tok@vg\endcsname{\def\PY@tc##1{\textcolor[rgb]{0.10,0.09,0.49}{##1}}}
\expandafter\def\csname PY@tok@vi\endcsname{\def\PY@tc##1{\textcolor[rgb]{0.10,0.09,0.49}{##1}}}
\expandafter\def\csname PY@tok@vm\endcsname{\def\PY@tc##1{\textcolor[rgb]{0.10,0.09,0.49}{##1}}}
\expandafter\def\csname PY@tok@sa\endcsname{\def\PY@tc##1{\textcolor[rgb]{0.73,0.13,0.13}{##1}}}
\expandafter\def\csname PY@tok@sb\endcsname{\def\PY@tc##1{\textcolor[rgb]{0.73,0.13,0.13}{##1}}}
\expandafter\def\csname PY@tok@sc\endcsname{\def\PY@tc##1{\textcolor[rgb]{0.73,0.13,0.13}{##1}}}
\expandafter\def\csname PY@tok@dl\endcsname{\def\PY@tc##1{\textcolor[rgb]{0.73,0.13,0.13}{##1}}}
\expandafter\def\csname PY@tok@s2\endcsname{\def\PY@tc##1{\textcolor[rgb]{0.73,0.13,0.13}{##1}}}
\expandafter\def\csname PY@tok@sh\endcsname{\def\PY@tc##1{\textcolor[rgb]{0.73,0.13,0.13}{##1}}}
\expandafter\def\csname PY@tok@s1\endcsname{\def\PY@tc##1{\textcolor[rgb]{0.73,0.13,0.13}{##1}}}
\expandafter\def\csname PY@tok@mb\endcsname{\def\PY@tc##1{\textcolor[rgb]{0.40,0.40,0.40}{##1}}}
\expandafter\def\csname PY@tok@mf\endcsname{\def\PY@tc##1{\textcolor[rgb]{0.40,0.40,0.40}{##1}}}
\expandafter\def\csname PY@tok@mh\endcsname{\def\PY@tc##1{\textcolor[rgb]{0.40,0.40,0.40}{##1}}}
\expandafter\def\csname PY@tok@mi\endcsname{\def\PY@tc##1{\textcolor[rgb]{0.40,0.40,0.40}{##1}}}
\expandafter\def\csname PY@tok@il\endcsname{\def\PY@tc##1{\textcolor[rgb]{0.40,0.40,0.40}{##1}}}
\expandafter\def\csname PY@tok@mo\endcsname{\def\PY@tc##1{\textcolor[rgb]{0.40,0.40,0.40}{##1}}}
\expandafter\def\csname PY@tok@ch\endcsname{\let\PY@it=\textit\def\PY@tc##1{\textcolor[rgb]{0.25,0.50,0.50}{##1}}}
\expandafter\def\csname PY@tok@cm\endcsname{\let\PY@it=\textit\def\PY@tc##1{\textcolor[rgb]{0.25,0.50,0.50}{##1}}}
\expandafter\def\csname PY@tok@cpf\endcsname{\let\PY@it=\textit\def\PY@tc##1{\textcolor[rgb]{0.25,0.50,0.50}{##1}}}
\expandafter\def\csname PY@tok@c1\endcsname{\let\PY@it=\textit\def\PY@tc##1{\textcolor[rgb]{0.25,0.50,0.50}{##1}}}
\expandafter\def\csname PY@tok@cs\endcsname{\let\PY@it=\textit\def\PY@tc##1{\textcolor[rgb]{0.25,0.50,0.50}{##1}}}

\def\PYZbs{\char`\\}
\def\PYZus{\char`\_}
\def\PYZob{\char`\{}
\def\PYZcb{\char`\}}
\def\PYZca{\char`\^}
\def\PYZam{\char`\&}
\def\PYZlt{\char`\<}
\def\PYZgt{\char`\>}
\def\PYZsh{\char`\#}
\def\PYZpc{\char`\%}
\def\PYZdl{\char`\$}
\def\PYZhy{\char`\-}
\def\PYZsq{\char`\'}
\def\PYZdq{\char`\"}
\def\PYZti{\char`\~}
% for compatibility with earlier versions
\def\PYZat{@}
\def\PYZlb{[}
\def\PYZrb{]}
\makeatother


    % Exact colors from NB
    \definecolor{incolor}{rgb}{0.0, 0.0, 0.5}
    \definecolor{outcolor}{rgb}{0.545, 0.0, 0.0}



    
    % Prevent overflowing lines due to hard-to-break entities
    \sloppy 
    % Setup hyperref package
    \hypersetup{
      breaklinks=true,  % so long urls are correctly broken across lines
      colorlinks=true,
      urlcolor=urlcolor,
      linkcolor=linkcolor,
      citecolor=citecolor,
      }
    % Slightly bigger margins than the latex defaults
    
    \geometry{verbose,tmargin=1in,bmargin=1in,lmargin=1in,rmargin=1in}
    
    

    \begin{document}
    
    
    \maketitle
    
    
\section{Solution}
I'll be using the following state variable assignments:

\begin{align*}
x_1&=p(t)\\
x_2&=q(t)\\
x_3&=\frac{dq(t)}{dt} \\
x_4&=r(t)
\end{align*}

Using these state variables, the provided set of differential equations can be rewritten as the following:

\begin{align}
\dot{x}_1+20x_1&=f(t) \\
\dot{x}_3+10x_3+20\dot{x}_1+400x_2&=0 \\
\dot{x}_4+5x_4+20x_2&=g(t)
\end{align}

With these state-variable-ized equations, a solution for $\dot{x}_1, \dot{x}_3, \text{ and } \dot{x}_4$ can be found by solving for them inside \textbf{Equations 1}, \textbf{2}, and \textbf{3}.

\begin{align}
\dot{x}_1 &= -20x_1 + f(t) \\
\dot{x}_3 &= -20\dot{x}_1 -400x_2 -10x_3 \\
\dot{x}_4 &= -20x_2 -5x_4 + g(t)
\end{align}

The final form of these state space equations can then be found by substituting \textbf{Equation 4} into \textbf{Equation 5}, resulting in the following:

\begin{align*}
\dot{x}_1 &= -20x_1 + f(t) \\
\dot{x}_2 &= x_3 \\ 
\dot{x}_3 &= -20\cdot(-20x_1 + f(t)) -400x_2 -10x_3 \\
\dot{x}_4 &= -20x_2 -5x_4 + g(t)
\end{align*}

This can be simplified, and the aligned equations reveal the \textbf{A} and \textbf{B} matrices.

\begin{align}
\dot{x}_1 &= &-20x_1 & & & &+f(t) \\
\dot{x}_2 &= & & &x_3 & & \\ 
\dot{x}_3 &= &400x_1 &-400x_2 &-10x_3 & &-20f(t) \\
\dot{x}_4 &= & &-20x_2 & &-5x_4 + &g(t)
\end{align}

Thus, \textbf{Equations 7-10} can be used to find the state space formulation. The output is given as $r(t)$, which was chosen to be the state variable $x_4$, creating a very simple output equation.

\begin{gather}
\begin{bmatrix}
\dot{x}_1 \\ \dot{x}_2 \\ \dot{x}_3 \\ \dot{x}_4
\end{bmatrix} =
\begin{bmatrix}
-20 & 0 & 0 & 0 \\
0 & 0 & 1 & 0 \\
400 & -400 & -10 & 0 \\
0 & -20 & 0 & -5
\end{bmatrix} \cdot
\begin{bmatrix}
x_1 \\ x_2 \\ x_3 \\ x_4
\end{bmatrix} + 
\begin{bmatrix}
1 & 0 \\
0 & 0 \\
-20 & 0 \\
0 & 1 \\
\end{bmatrix} \cdot
\begin{bmatrix}
f(t) \\
g(t)
\end{bmatrix}
\end{gather}

\begin{gather}
y_{out}(t)=\begin{bmatrix}
0 & 0 & 0 & 1
\end{bmatrix} \cdot
\begin{bmatrix}
x_1 \\ x_2 \\ x_3 \\ x_4
\end{bmatrix} + 0
\end{gather}

From \textbf{Equation 11} and \textbf{12}, matrix \textbf{A}, \textbf{B}, \textbf{C}, and \textbf{D} can be stated as:

\begin{gather*}
A = \begin{bmatrix}
-20 & 0 & 0 & 0 \\
0 & 0 & 1 & 0 \\
400 & -400 & -10 & 0 \\
0 & -20 & 0 & -5
\end{bmatrix}, B=\begin{bmatrix}
1 & 0 \\
0 & 0 \\
-20 & 0 \\
0 & 1 \\
\end{bmatrix}, C=\begin{bmatrix}
0 & 0 & 0 & 1
\end{bmatrix}, D=0
\end{gather*}

\newpage
Below is a plot of the given input functions, state variables (defined at the beginning of~\textbf{Section~1}), and the output function -- all as a function of time.

\begin{center}
    \adjustimage{max size={1.1\linewidth}{0.9\paperheight}}{output-graph.png}
    \end{center}
    { \hspace*{\fill} \\}
    
My code for generating the above plot is listed below, in \textbf{Section 2}.
\newpage

\hypertarget{code-appendix}{%
\section{Code Appendix}\label{code-appendix}}

\hypertarget{package-imports}{%
\subsection{Package Imports}\label{package-imports}}

    \begin{Verbatim}[commandchars=\\\{\}]
{\color{incolor}In [{\color{incolor}1}]:} \PY{k+kn}{import} \PY{n+nn}{numpy} \PY{k}{as} \PY{n+nn}{np}
        \PY{k+kn}{import} \PY{n+nn}{seaborn} \PY{k}{as} \PY{n+nn}{sns}
        \PY{k+kn}{import} \PY{n+nn}{pandas} \PY{k}{as} \PY{n+nn}{pd}
        \PY{k+kn}{import} \PY{n+nn}{matplotlib}\PY{n+nn}{.}\PY{n+nn}{pyplot} \PY{k}{as} \PY{n+nn}{plt}
\end{Verbatim}

    \hypertarget{generic-function-for-a-state-variable-solver}{%
\subsection{Generic Function for a State Variable
Solver}\label{generic-function-for-a-state-variable-solver}}

    \begin{Verbatim}[commandchars=\\\{\}]
{\color{incolor}In [{\color{incolor}2}]:} \PY{k}{def} \PY{n+nf}{state\PYZus{}solver}\PY{p}{(}\PY{n}{A\PYZus{}matrix}\PY{p}{,} \PY{n}{B\PYZus{}matrix}\PY{p}{,} \PY{n}{force\PYZus{}f}\PY{p}{,} \PY{n}{initial\PYZus{}conditions}\PY{p}{,}
                         \PY{n}{time\PYZus{}range}\PY{o}{=}\PY{p}{[}\PY{l+m+mi}{0}\PY{p}{,} \PY{l+m+mi}{10}\PY{p}{]}\PY{p}{,} \PY{n}{dt}\PY{o}{=}\PY{l+m+mf}{0.01}\PY{p}{)}\PY{p}{:}
            \PY{n}{time\PYZus{}values} \PY{o}{=} \PY{n}{np}\PY{o}{.}\PY{n}{arange}\PY{p}{(}\PY{n}{time\PYZus{}range}\PY{p}{[}\PY{l+m+mi}{0}\PY{p}{]}\PY{p}{,} \PY{n}{time\PYZus{}range}\PY{p}{[}\PY{l+m+mi}{1}\PY{p}{]}\PY{p}{,} \PY{n}{dt}\PY{p}{)}
            \PY{n}{x\PYZus{}vals} \PY{o}{=} \PY{n}{np}\PY{o}{.}\PY{n}{array}\PY{p}{(}\PY{n}{initial\PYZus{}conditions}\PY{p}{)}
            \PY{n}{state\PYZus{}variables} \PY{o}{=} \PY{p}{[}\PY{p}{[}\PY{p}{]} \PY{k}{for} \PY{n}{\PYZus{}} \PY{o+ow}{in} \PY{n}{initial\PYZus{}conditions}\PY{p}{]}
            
            \PY{c+c1}{\PYZsh{} Loop through each instance in time, calculate the state variable at that time}
            \PY{k}{for} \PY{n}{time} \PY{o+ow}{in} \PY{n}{time\PYZus{}values}\PY{p}{:}
                \PY{c+c1}{\PYZsh{} For multiple input functions, do matrix operations on the 2nd term}
                \PY{k}{if} \PY{n+nb}{isinstance}\PY{p}{(}\PY{n}{force\PYZus{}f}\PY{p}{,} \PY{n+nb}{list}\PY{p}{)}\PY{p}{:}
                    \PY{n}{force\PYZus{}vals} \PY{o}{=} \PY{n}{np}\PY{o}{.}\PY{n}{array}\PY{p}{(}\PY{p}{[}\PY{n}{func}\PY{p}{(}\PY{n}{time}\PY{p}{)} \PY{k}{for} \PY{n}{func} \PY{o+ow}{in} \PY{n}{force\PYZus{}f}\PY{p}{]}\PY{p}{)}
                    \PY{n}{x\PYZus{}vals} \PY{o}{=} \PY{n}{x\PYZus{}vals} \PY{o}{+} \PY{n}{dt} \PY{o}{*} \PY{p}{(}\PY{n}{A\PYZus{}matrix} \PY{o}{@} \PY{n}{x\PYZus{}vals}\PY{p}{)} \PY{o}{+} \PY{n}{dt} \PY{o}{*} \PY{p}{(}\PY{n}{B\PYZus{}matrix} \PY{o}{@} \PY{n}{force\PYZus{}vals}\PY{p}{)}
                \PY{k}{else}\PY{p}{:}
                    \PY{n}{x\PYZus{}vals} \PY{o}{=} \PY{n}{x\PYZus{}vals} \PY{o}{+} \PY{n}{dt} \PY{o}{*} \PY{p}{(}\PY{n}{A\PYZus{}matrix} \PY{o}{@} \PY{n}{x\PYZus{}vals}\PY{p}{)} \PY{o}{+} \PY{n}{dt} \PY{o}{*} \PY{p}{(}\PY{n}{B\PYZus{}matrix} \PY{o}{*} \PY{n}{force\PYZus{}f}\PY{p}{(}\PY{n}{time}\PY{p}{)}\PY{p}{)}
                \PY{c+c1}{\PYZsh{} Add the value of the state variable at this time to the list}
                \PY{k}{for} \PY{n}{index}\PY{p}{,} \PY{n}{\PYZus{}} \PY{o+ow}{in} \PY{n+nb}{enumerate}\PY{p}{(}\PY{n}{state\PYZus{}variables}\PY{p}{)}\PY{p}{:}
                    \PY{n}{state\PYZus{}variables}\PY{p}{[}\PY{n}{index}\PY{p}{]}\PY{o}{.}\PY{n}{append}\PY{p}{(}\PY{n}{x\PYZus{}vals}\PY{p}{[}\PY{n}{index}\PY{p}{]}\PY{p}{)}
                    
            \PY{k}{return} \PY{n}{state\PYZus{}variables}\PY{p}{,} \PY{n}{time\PYZus{}values}
\end{Verbatim}

    \hypertarget{generic-function-for-creating-combined-dataframe-out-of-state-variables-input-function-and-output-function}{%
\subsection{\texorpdfstring{Generic Function for Creating a Combined
\texttt{DataFrame} out of State Variables, Input Function(s), and an Output
Function}{Generic Function for Creating Combined DataFrame out of State Variables, Input Function, and Output Function}}\label{generic-function-for-creating-combined-dataframe-out-of-state-variables-input-function-and-output-function}}

    \begin{Verbatim}[commandchars=\\\{\}]
{\color{incolor}In [{\color{incolor}3}]:} \PY{k}{def} \PY{n+nf}{create\PYZus{}df}\PY{p}{(}\PY{n}{time}\PY{p}{,} \PY{n}{state\PYZus{}variables}\PY{p}{,} \PY{n}{input\PYZus{}fs}\PY{p}{,} \PY{n}{output\PYZus{}f}\PY{p}{,} \PY{n}{state\PYZus{}names}\PY{p}{,} \PY{n}{input\PYZus{}names}\PY{p}{)}\PY{p}{:}
            \PY{n}{df\PYZus{}list} \PY{o}{=} \PY{p}{[}\PY{p}{]}
            \PY{c+c1}{\PYZsh{} If there is only one input, create an array out of it anyway (for looping)}
            \PY{n}{input\PYZus{}fs} \PY{o}{=} \PY{n}{input\PYZus{}fs} \PY{k}{if} \PY{n+nb}{isinstance}\PY{p}{(}\PY{n}{input\PYZus{}fs}\PY{p}{,} \PY{p}{(}\PY{n+nb}{list}\PY{p}{,} \PY{n}{np}\PY{o}{.}\PY{n}{ndarray}\PY{p}{)}\PY{p}{)} \PY{k}{else} \PY{p}{[}\PY{n}{input\PYZus{}fs}\PY{p}{]}
            \PY{c+c1}{\PYZsh{} For each input function, compute the time\PYZhy{}values and add to the DF list}
            \PY{k}{for} \PY{n}{input\PYZus{}f}\PY{p}{,} \PY{n}{name} \PY{o+ow}{in} \PY{n+nb}{zip}\PY{p}{(}\PY{n}{input\PYZus{}fs}\PY{p}{,} \PY{n}{input\PYZus{}names}\PY{p}{)}\PY{p}{:}
                \PY{n}{input\PYZus{}vals} \PY{o}{=} \PY{p}{[}\PY{n}{input\PYZus{}f}\PY{p}{(}\PY{n}{t}\PY{p}{)} \PY{k}{for} \PY{n}{t} \PY{o+ow}{in} \PY{n}{time}\PY{p}{]}
                \PY{n}{df} \PY{o}{=} \PY{n}{pd}\PY{o}{.}\PY{n}{DataFrame}\PY{p}{(}\PY{n+nb}{list}\PY{p}{(}\PY{n+nb}{zip}\PY{p}{(}\PY{n}{time}\PY{p}{,} \PY{n}{input\PYZus{}vals}\PY{p}{)}\PY{p}{)}\PY{p}{,} \PY{n}{columns}\PY{o}{=}\PY{p}{[}\PY{l+s+s2}{\PYZdq{}}\PY{l+s+s2}{Time}\PY{l+s+s2}{\PYZdq{}}\PY{p}{,} \PY{l+s+s2}{\PYZdq{}}\PY{l+s+s2}{Value}\PY{l+s+s2}{\PYZdq{}}\PY{p}{]}\PY{p}{)}
                \PY{n}{df}\PY{p}{[}\PY{l+s+s2}{\PYZdq{}}\PY{l+s+s2}{Name}\PY{l+s+s2}{\PYZdq{}}\PY{p}{]} \PY{o}{=} \PY{n}{name}             \PY{c+c1}{\PYZsh{} Name the solution column}
                \PY{n}{df}\PY{p}{[}\PY{l+s+s2}{\PYZdq{}}\PY{l+s+s2}{Type}\PY{l+s+s2}{\PYZdq{}}\PY{p}{]} \PY{o}{=} \PY{l+s+s2}{\PYZdq{}}\PY{l+s+s2}{Input Functions}\PY{l+s+s2}{\PYZdq{}} \PY{k}{if} \PY{n+nb}{len}\PY{p}{(}\PY{n}{input\PYZus{}fs}\PY{p}{)} \PY{o}{\PYZgt{}} \PY{l+m+mi}{1} \PY{k}{else} \PY{l+s+s2}{\PYZdq{}}\PY{l+s+s2}{Input Function}\PY{l+s+s2}{\PYZdq{}}
                \PY{n}{df\PYZus{}list}\PY{o}{.}\PY{n}{append}\PY{p}{(}\PY{n}{df}\PY{p}{)}
            
            \PY{c+c1}{\PYZsh{} Loop through every state variable, add its time and output to a list of DFs}
            \PY{k}{for} \PY{n}{state\PYZus{}var}\PY{p}{,} \PY{n}{name} \PY{o+ow}{in} \PY{n+nb}{zip}\PY{p}{(}\PY{n}{state\PYZus{}variables}\PY{p}{,} \PY{n}{state\PYZus{}names}\PY{p}{)}\PY{p}{:}
                \PY{n}{df} \PY{o}{=} \PY{n}{pd}\PY{o}{.}\PY{n}{DataFrame}\PY{p}{(}\PY{n+nb}{list}\PY{p}{(}\PY{n+nb}{zip}\PY{p}{(}\PY{n}{time}\PY{p}{,} \PY{n}{state\PYZus{}var}\PY{p}{)}\PY{p}{)}\PY{p}{,} \PY{n}{columns}\PY{o}{=}\PY{p}{[}\PY{l+s+s2}{\PYZdq{}}\PY{l+s+s2}{Time}\PY{l+s+s2}{\PYZdq{}}\PY{p}{,} \PY{l+s+s2}{\PYZdq{}}\PY{l+s+s2}{Value}\PY{l+s+s2}{\PYZdq{}}\PY{p}{]}\PY{p}{)}
                \PY{n}{df}\PY{p}{[}\PY{l+s+s2}{\PYZdq{}}\PY{l+s+s2}{Name}\PY{l+s+s2}{\PYZdq{}}\PY{p}{]} \PY{o}{=} \PY{n}{name}              \PY{c+c1}{\PYZsh{} Name the solution column}
                \PY{n}{df}\PY{p}{[}\PY{l+s+s2}{\PYZdq{}}\PY{l+s+s2}{Type}\PY{l+s+s2}{\PYZdq{}}\PY{p}{]} \PY{o}{=} \PY{l+s+s2}{\PYZdq{}}\PY{l+s+s2}{State Variables}\PY{l+s+s2}{\PYZdq{}} \PY{c+c1}{\PYZsh{} The \PYZsq{}type\PYZsq{} of this DF}
                \PY{n}{df\PYZus{}list}\PY{o}{.}\PY{n}{append}\PY{p}{(}\PY{n}{df}\PY{p}{)}
                    
            \PY{c+c1}{\PYZsh{} Add the output to the DF list}
            \PY{n}{output\PYZus{}vals} \PY{o}{=} \PY{n}{output\PYZus{}f}\PY{p}{(}\PY{n}{state\PYZus{}variables}\PY{p}{)}
            \PY{n}{df} \PY{o}{=} \PY{n}{pd}\PY{o}{.}\PY{n}{DataFrame}\PY{p}{(}\PY{n+nb}{list}\PY{p}{(}\PY{n+nb}{zip}\PY{p}{(}\PY{n}{time}\PY{p}{,} \PY{n}{output\PYZus{}vals}\PY{p}{)}\PY{p}{)}\PY{p}{,} \PY{n}{columns}\PY{o}{=}\PY{p}{[}\PY{l+s+s2}{\PYZdq{}}\PY{l+s+s2}{Time}\PY{l+s+s2}{\PYZdq{}}\PY{p}{,} \PY{l+s+s2}{\PYZdq{}}\PY{l+s+s2}{Value}\PY{l+s+s2}{\PYZdq{}}\PY{p}{]}\PY{p}{)}
            \PY{n}{df}\PY{p}{[}\PY{l+s+s2}{\PYZdq{}}\PY{l+s+s2}{Name}\PY{l+s+s2}{\PYZdq{}}\PY{p}{]} \PY{o}{=} \PY{l+s+s2}{\PYZdq{}}\PY{l+s+s2}{Output}\PY{l+s+s2}{\PYZdq{}}          \PY{c+c1}{\PYZsh{} Name of the instance}
            \PY{n}{df}\PY{p}{[}\PY{l+s+s2}{\PYZdq{}}\PY{l+s+s2}{Type}\PY{l+s+s2}{\PYZdq{}}\PY{p}{]} \PY{o}{=} \PY{l+s+s2}{\PYZdq{}}\PY{l+s+s2}{Output Function}\PY{l+s+s2}{\PYZdq{}} \PY{c+c1}{\PYZsh{} The \PYZsq{}type\PYZsq{} of this DF}
            \PY{n}{df\PYZus{}list}\PY{o}{.}\PY{n}{append}\PY{p}{(}\PY{n}{df}\PY{p}{)}
                
            \PY{k}{return} \PY{n}{pd}\PY{o}{.}\PY{n}{concat}\PY{p}{(}\PY{n}{df\PYZus{}list}\PY{p}{,} \PY{n}{ignore\PYZus{}index}\PY{o}{=}\PY{k+kc}{True}\PY{p}{,} \PY{n}{axis}\PY{o}{=}\PY{l+m+mi}{0}\PY{p}{)}
\end{Verbatim}

    \hypertarget{generic-function-to-create-the-three-necessary-plots}{%
\subsection{Generic Function to create the Three Necessary
Plots}\label{generic-function-to-create-the-three-necessary-plots}}

    \begin{Verbatim}[commandchars=\\\{\}]
{\color{incolor}In [{\color{incolor}4}]:} \PY{k}{def} \PY{n+nf}{create\PYZus{}plots}\PY{p}{(}\PY{n}{df}\PY{p}{)}\PY{p}{:}
            \PY{n}{sns}\PY{o}{.}\PY{n}{set}\PY{p}{(}\PY{n}{style}\PY{o}{=}\PY{l+s+s2}{\PYZdq{}}\PY{l+s+s2}{whitegrid}\PY{l+s+s2}{\PYZdq{}}\PY{p}{,} \PY{n}{font\PYZus{}scale}\PY{o}{=}\PY{l+m+mf}{2.25}\PY{p}{)}
            \PY{n}{g} \PY{o}{=} \PY{n}{sns}\PY{o}{.}\PY{n}{FacetGrid}\PY{p}{(}\PY{n}{df}\PY{p}{,} \PY{n}{hue}\PY{o}{=}\PY{l+s+s2}{\PYZdq{}}\PY{l+s+s2}{Name}\PY{l+s+s2}{\PYZdq{}}\PY{p}{,} \PY{n}{row}\PY{o}{=}\PY{l+s+s2}{\PYZdq{}}\PY{l+s+s2}{Type}\PY{l+s+s2}{\PYZdq{}}\PY{p}{,} \PY{n}{height}\PY{o}{=}\PY{l+m+mf}{5.5}\PY{p}{,} \PY{n}{aspect}\PY{o}{=}\PY{l+m+mi}{4}\PY{p}{,} \PY{n}{sharey}\PY{o}{=}\PY{k+kc}{False}\PY{p}{)}
            \PY{n}{g}\PY{o}{.}\PY{n}{map}\PY{p}{(}\PY{n}{sns}\PY{o}{.}\PY{n}{lineplot}\PY{p}{,} \PY{l+s+s2}{\PYZdq{}}\PY{l+s+s2}{Time}\PY{l+s+s2}{\PYZdq{}}\PY{p}{,} \PY{l+s+s2}{\PYZdq{}}\PY{l+s+s2}{Value}\PY{l+s+s2}{\PYZdq{}}\PY{p}{,} \PY{o}{*}\PY{o}{*}\PY{n+nb}{dict}\PY{p}{(}\PY{n}{linewidth}\PY{o}{=}\PY{l+m+mf}{2.5}\PY{p}{)}\PY{p}{)}\PY{o}{.}\PY{n}{add\PYZus{}legend}\PY{p}{(}\PY{p}{)}\PY{o}{.}\PY{n}{despine}\PY{p}{(}\PY{n}{bottom}\PY{o}{=}\PY{k+kc}{True}\PY{p}{,} \PY{n}{left}\PY{o}{=}\PY{k+kc}{True}\PY{p}{)}
\end{Verbatim}

    \hypertarget{my-solution}{%
\subsection{My Solution}\label{my-solution}}

    \begin{Verbatim}[commandchars=\\\{\}]
{\color{incolor}In [{\color{incolor}5}]:} \PY{n}{A\PYZus{}matrix} \PY{o}{=} \PY{n}{np}\PY{o}{.}\PY{n}{array}\PY{p}{(}\PY{p}{[}\PY{p}{[}\PY{o}{\PYZhy{}}\PY{l+m+mi}{20}\PY{p}{,}    \PY{l+m+mi}{0}\PY{p}{,}   \PY{l+m+mi}{0}\PY{p}{,}  \PY{l+m+mi}{0}\PY{p}{]}\PY{p}{,}
                             \PY{p}{[}  \PY{l+m+mi}{0}\PY{p}{,}    \PY{l+m+mi}{0}\PY{p}{,}   \PY{l+m+mi}{1}\PY{p}{,}  \PY{l+m+mi}{0}\PY{p}{]}\PY{p}{,}
                             \PY{p}{[}\PY{l+m+mi}{400}\PY{p}{,} \PY{o}{\PYZhy{}}\PY{l+m+mi}{400}\PY{p}{,} \PY{o}{\PYZhy{}}\PY{l+m+mi}{10}\PY{p}{,}  \PY{l+m+mi}{0}\PY{p}{]}\PY{p}{,}
                             \PY{p}{[}  \PY{l+m+mi}{0}\PY{p}{,}  \PY{o}{\PYZhy{}}\PY{l+m+mi}{20}\PY{p}{,}   \PY{l+m+mi}{0}\PY{p}{,} \PY{o}{\PYZhy{}}\PY{l+m+mi}{5}\PY{p}{]}\PY{p}{]}\PY{p}{)}
        \PY{n}{B\PYZus{}matrix} \PY{o}{=} \PY{n}{np}\PY{o}{.}\PY{n}{array}\PY{p}{(}\PY{p}{[}\PY{p}{[}  \PY{l+m+mi}{1}\PY{p}{,} \PY{l+m+mi}{0}\PY{p}{]}\PY{p}{,}
                             \PY{p}{[}  \PY{l+m+mi}{0}\PY{p}{,} \PY{l+m+mi}{0}\PY{p}{]}\PY{p}{,}
                             \PY{p}{[}\PY{o}{\PYZhy{}}\PY{l+m+mi}{20}\PY{p}{,} \PY{l+m+mi}{0}\PY{p}{]}\PY{p}{,}
                             \PY{p}{[}  \PY{l+m+mi}{0}\PY{p}{,} \PY{l+m+mi}{1}\PY{p}{]}\PY{p}{]}\PY{p}{)}
        \PY{n}{init\PYZus{}state} \PY{o}{=} \PY{p}{[}\PY{l+m+mi}{0}\PY{p}{,} \PY{l+m+mi}{0}\PY{p}{,} \PY{l+m+mi}{0}\PY{p}{,} \PY{l+m+mi}{0}\PY{p}{]}
        
        \PY{n}{step} \PY{o}{=} \PY{k}{lambda} \PY{n}{t}\PY{p}{:} \PY{l+m+mi}{0} \PY{k}{if} \PY{n}{t} \PY{o}{\PYZlt{}} \PY{l+m+mi}{0} \PY{k}{else} \PY{l+m+mi}{1}
        \PY{n}{f} \PY{o}{=} \PY{k}{lambda} \PY{n}{t}\PY{p}{:} \PY{l+m+mi}{5} \PY{o}{*} \PY{n}{np}\PY{o}{.}\PY{n}{sin}\PY{p}{(}\PY{p}{(}\PY{l+m+mi}{2} \PY{o}{*} \PY{n}{np}\PY{o}{.}\PY{n}{pi} \PY{o}{*} \PY{n}{t}\PY{p}{)}\PY{o}{/}\PY{l+m+mi}{30}\PY{p}{)} \PY{o}{*} \PY{p}{(}\PY{n}{step}\PY{p}{(}\PY{n}{t}\PY{p}{)} \PY{o}{\PYZhy{}} \PY{n}{step}\PY{p}{(}\PY{n}{t} \PY{o}{\PYZhy{}} \PY{l+m+mi}{1}\PY{p}{)}\PY{p}{)}
        \PY{n}{g} \PY{o}{=} \PY{k}{lambda} \PY{n}{t}\PY{p}{:} \PY{l+m+mi}{10} \PY{o}{*} \PY{n}{np}\PY{o}{.}\PY{n}{sin}\PY{p}{(}\PY{p}{(}\PY{l+m+mi}{2} \PY{o}{*} \PY{n}{np}\PY{o}{.}\PY{n}{pi} \PY{o}{*} \PY{p}{(}\PY{n}{t} \PY{o}{\PYZhy{}} \PY{l+m+mi}{1}\PY{p}{)}\PY{p}{)}\PY{o}{/}\PY{l+m+mi}{50}\PY{p}{)} \PY{o}{*} \PY{p}{(}\PY{n}{step}\PY{p}{(}\PY{n}{t} \PY{o}{\PYZhy{}} \PY{l+m+mi}{1}\PY{p}{)} \PY{o}{\PYZhy{}} \PY{n}{step}\PY{p}{(}\PY{n}{t} \PY{o}{\PYZhy{}} \PY{l+m+mi}{2}\PY{p}{)}\PY{p}{)}
        \PY{n}{input\PYZus{}funcs} \PY{o}{=} \PY{p}{[}\PY{n}{f}\PY{p}{,} \PY{n}{g}\PY{p}{]}
        \PY{n}{output\PYZus{}func} \PY{o}{=} \PY{k}{lambda} \PY{n}{st}\PY{p}{:} \PY{p}{[}\PY{n}{x4} \PY{k}{for} \PY{n}{x1}\PY{p}{,} \PY{n}{x2}\PY{p}{,} \PY{n}{x3}\PY{p}{,} \PY{n}{x4} \PY{o+ow}{in} \PY{n+nb}{zip}\PY{p}{(}\PY{n}{st}\PY{p}{[}\PY{l+m+mi}{0}\PY{p}{]}\PY{p}{,} \PY{n}{st}\PY{p}{[}\PY{l+m+mi}{1}\PY{p}{]}\PY{p}{,} \PY{n}{st}\PY{p}{[}\PY{l+m+mi}{2}\PY{p}{]}\PY{p}{,} \PY{n}{st}\PY{p}{[}\PY{l+m+mi}{3}\PY{p}{]}\PY{p}{)}\PY{p}{]}
        
        \PY{n}{state\PYZus{}vars}\PY{p}{,} \PY{n}{time} \PY{o}{=} \PY{n}{state\PYZus{}solver}\PY{p}{(}\PY{n}{A\PYZus{}matrix}\PY{p}{,} \PY{n}{B\PYZus{}matrix}\PY{p}{,} \PY{n}{input\PYZus{}funcs}\PY{p}{,} \PY{n}{init\PYZus{}state}\PY{p}{,} \PY{p}{[}\PY{l+m+mi}{0}\PY{p}{,} \PY{l+m+mf}{2.5}\PY{p}{]}\PY{p}{)}
\end{Verbatim}

    \hypertarget{plot-the-state-variables-over-time}{%
\paragraph{Plot the state variables over
time}\label{plot-the-state-variables-over-time}}

    \begin{Verbatim}[commandchars=\\\{\}]
{\color{incolor}In [{\color{incolor}6}]:} \PY{n}{state\PYZus{}names} \PY{o}{=} \PY{p}{[}\PY{l+s+s2}{\PYZdq{}}\PY{l+s+s2}{\PYZdl{}x\PYZus{}1(t)\PYZdl{}}\PY{l+s+s2}{\PYZdq{}}\PY{p}{,} \PY{l+s+s2}{\PYZdq{}}\PY{l+s+s2}{\PYZdl{}x\PYZus{}2(t)\PYZdl{}}\PY{l+s+s2}{\PYZdq{}}\PY{p}{,} \PY{l+s+s2}{\PYZdq{}}\PY{l+s+s2}{\PYZdl{}x\PYZus{}3(t)\PYZdl{}}\PY{l+s+s2}{\PYZdq{}}\PY{p}{,} \PY{l+s+s2}{\PYZdq{}}\PY{l+s+s2}{\PYZdl{}x\PYZus{}4(t)\PYZdl{}}\PY{l+s+s2}{\PYZdq{}}\PY{p}{]}
        \PY{n}{input\PYZus{}names} \PY{o}{=} \PY{p}{[}\PY{l+s+s2}{\PYZdq{}}\PY{l+s+s2}{\PYZdl{}f(t)\PYZdl{}}\PY{l+s+s2}{\PYZdq{}}\PY{p}{,} \PY{l+s+s2}{\PYZdq{}}\PY{l+s+s2}{\PYZdl{}g(t)\PYZdl{}}\PY{l+s+s2}{\PYZdq{}}\PY{p}{]}
        \PY{n}{df} \PY{o}{=} \PY{n}{create\PYZus{}df}\PY{p}{(}\PY{n}{time}\PY{p}{,} \PY{n}{state\PYZus{}vars}\PY{p}{,} \PY{n}{input\PYZus{}funcs}\PY{p}{,} \PY{n}{output\PYZus{}func}\PY{p}{,} \PY{n}{state\PYZus{}names}\PY{p}{,} \PY{n}{input\PYZus{}names}\PY{p}{)}
\end{Verbatim}

    \hypertarget{plot-the-results}{%
\paragraph{Plot the results}\label{plot-the-results}}

    \begin{Verbatim}[commandchars=\\\{\}]
{\color{incolor}In [{\color{incolor}7}]:} \PY{n}{create\PYZus{}plots}\PY{p}{(}\PY{n}{df}\PY{p}{)}
\end{Verbatim}

    \begin{center}
    \adjustimage{max size={1.1\linewidth}{0.9\paperheight}}{output_13_0.png}
    \end{center}
    { \hspace*{\fill} \\}
    

    % Add a bibliography block to the postdoc
    
    
    
    \end{document}
