\documentclass[a4paper, 12pt]{article}
\usepackage{listings} 
\usepackage{xcolor}
\usepackage{mdframed}
\usepackage{graphicx}
\usepackage{pgfplots}
\usepackage{float}
\usepackage{mathtools}

%% Custom FSM's
\usepackage{tikz}
\usetikzlibrary{automata, positioning, arrows}
\tikzset{thick, ->, >=stealth', node distance=6cm, every state/.style={thick, fill=gray!10}, initial text=$ $}

%% Macros for logic timing diagrams
\newcounter{wavenum}
\setlength{\unitlength}{1cm}
% advance clock one cycle, not to be called directly
\newcommand*{\clki}{
  \draw (t_cur) -- ++(0,.3) -- ++(.5,0) -- ++(0,-.6) -- ++(.5,0) -- ++(0,.3)
    node[time] (t_cur) {};
}
\newcommand*{\bitvector}[3]{
  \draw[fill=#3] (t_cur) -- ++( .1, .3) -- ++(#2-.2,0) -- ++(.1, -.3)
                         -- ++(-.1,-.3) -- ++(.2-#2,0) -- cycle;
  \path (t_cur) -- node[anchor=mid] {#1} ++(#2,0) node[time] (t_cur) {};
}
% \known{val}{length}
\newcommand*{\known}[2]{
    \bitvector{#1}{#2}{white}
}
% \unknown{length}
\newcommand*{\unknown}[2][XXX]{
    \bitvector{#1}{#2}{black!20}
}
% \bit{1 or 0}{length}
\newcommand*{\bit}[2]{
  \draw (t_cur) -- ++(0,.6*#1-.3) -- ++(#2,0) -- ++(0,.3-.6*#1)
    node[time] (t_cur) {};
}
% \unknownbit{length}
\newcommand*{\unknownbit}[1]{
  \draw[ultra thick,black!50] (t_cur) -- ++(#1,0) node[time] (t_cur) {};
}
% \nextwave{name}
\newcommand{\nextwave}[1]{
  \path (0,\value{wavenum}) node[left] {#1} node[time] (t_cur) {};
  \addtocounter{wavenum}{-1}
}
% \clk{name}{period}
\newcommand{\clk}[2]{
    \nextwave{#1}
    \FPeval{\res}{(\wavewidth+1)/#2}
    \FPeval{\reshalf}{#2/2}
    \foreach \t in {1,2,...,\res}{
        \bit{\reshalf}{1}
        \bit{\reshalf}{0}
    }
}

% \begin{wave}[clkname]{num_waves}{clock_cycles}{start_val}
\newenvironment{wave}[4][clk]{
  \begin{tikzpicture}[draw=black, yscale=.7,xscale=1]
    \tikzstyle{time}=[coordinate]
    \setlength{\unitlength}{1cm}
    \def\wavewidth{#3}
    \setcounter{wavenum}{0}
    \nextwave{#1}
    \foreach \t in {0,1,...,\wavewidth}{
    	 \pgfmathtruncatemacro{\val}{\t+#4} %Calculate time values
      \draw[dotted] (t_cur) + (0,.5) node[above] {\tiny{t=\val}} -- ++(0,.4-#2);
      \clki
    }
}{\end{tikzpicture}}

% Specific Line Breaks
% See https://tex.stackexchange.com/questions/26174/ for details
\usepackage[british]{babel} 

% Page Margins
\usepackage[margin=1.00in]{geometry}

% Large, array-sized, ceiling and floor operators
\DeclarePairedDelimiter\ceil{\lceil}{\rceil}
\DeclarePairedDelimiter\floor{\lfloor}{\rfloor}
\definecolor{code-gray}{gray}{0.93}

% Beginning of Document
\begin{document}
% Title
\title{ECE 440 - Homework \#5}
\author{Collin Heist}
\date{\today}
\maketitle

% Table of Content and Listings
\pagenumbering{arabic}

% Beginning of Report
\section{Timing Diagrams}

\begin{figure}[h]
\hspace{-2cm}
\begin{wave}[\textbf{Clock}]{9}{16}{0}
\nextwave{Reset} \bit{1}{1.3} \bit{0}{17-1.3}
\nextwave{Counter[3:0]} \unknown[x]{1} \known{\tiny{0b0000}}{1} \known{\tiny{0b0001}}{1} \known{\tiny{0b0010}}{1} \known{\tiny{0b0011}}{1} \known{\tiny{0b0100}}{1} \known{\tiny{0b0101}}{1} \known{\tiny{0b0110}}{1} \known{\tiny{0b0111}}{1} \known{\tiny{0b1000}}{1} \known{\tiny{0b1001}}{1} \known{\tiny{0b1010}}{1} \known{\tiny{0b1011}}{1} \known{\tiny{0b1100}}{1} \known{\tiny{0b1101}}{1} \known{\tiny{0b1110}}{1} \known{\tiny{0b1111}}{1}
\nextwave{Write Enable} \unknownbit{1} \bit{0}{9} \bit{1}{1} \bit{0}{1} \bit{1}{1} \bit{0}{1} \bit{1}{1} \bit{0}{1} \bit{1}{1}
\nextwave{Address} \unknown[x]{1} \known{\tiny{0b00}}{2} \known{\tiny{0b01}}{2} \known{\tiny{0b10}}{2} \known{\tiny{0b11}}{2} \known{\tiny{0b00}}{2} \known{\tiny{0b01}}{2} \known{\tiny{0b10}}{2} \known{\tiny{0b11}}{2}
\nextwave{SPO} \unknown[x]{1.3} \known{\tiny{0b11}}{2} \known{\tiny{0b10}}{2} \known{\tiny{0b01}}{2} \known{\tiny{0b00}}{2} \known{\tiny{0b11}}{2} \known{\tiny{0b10}}{2} \known{\tiny{0b01}}{2} \known{\tiny{0b00}}{1.7}
\nextwave{Data In} \unknown[x]{2} \known{\tiny{0b10}}{2} \known{\tiny{0b01}}{2} \known{\tiny{0b00}}{2} \known{\tiny{0b11}}{2} \known{\tiny{0b10}}{2} \known{\tiny{0b01}}{2} \known{\tiny{0b00}}{2} \known{\tiny{0b11}}{1} 
\nextwave{Data Out$_2$} \unknown[x]{2} \known{\tiny{0b11}}{2} \known{\tiny{0b10}}{2} \known{\tiny{0b01}}{2} \known{\tiny{0b00}}{2} \known{\tiny{0b11}}{2} \known{\tiny{0b10}}{2} \known{\tiny{0b01}}{2} \known{\tiny{0b00}}{1}
\nextwave{\textbf{Data Out$_{10}$}} \unknown[x]{2} \known{\tiny{3}}{2} \known{\tiny{2}}{2} \known{\tiny{1}}{2} \known{\tiny{0}}{2}\known{\tiny{3}}{2} \known{\tiny{2}}{2} \known{\tiny{1}}{2} \known{\tiny{0}}{1}
\end{wave}
\caption{First counter cycle.}
\end{figure}

During each full cycle of the 4-bit counter, the \textbf{Write-Enable} signal goes high four times (at time 9, 11, 13, and 15). During each of these periods, the currently selected address (in order 0, 1, 2 and 3) is overwritten with one less than its current value. The output is a sequential reading of the memory module (repeated twice), reading 3-2-1-0 at first.

The second full-sequence of the timer shows the pattern of this circuit completely. 

\begin{figure}[H]
\hspace{-2cm}
\begin{wave}[\textbf{Clock}]{9}{15}{17}
\nextwave{Reset} \bit{0}{16}
\nextwave{Counter[3:0]} \known{\tiny{0b0000}}{1} \known{\tiny{0b0001}}{1} \known{\tiny{0b0010}}{1} \known{\tiny{0b0011}}{1} \known{\tiny{0b0100}}{1} \known{\tiny{0b0101}}{1} \known{\tiny{0b0110}}{1} \known{\tiny{0b0111}}{1} \known{\tiny{0b1000}}{1} \known{\tiny{0b1001}}{1} \known{\tiny{0b1010}}{1} \known{\tiny{0b1011}}{1} \known{\tiny{0b1100}}{1} \known{\tiny{0b1101}}{1} \known{\tiny{0b1110}}{1} \known{\tiny{0b1111}}{1}
\nextwave{Write Enable} \bit{0}{9} \bit{1}{1} \bit{0}{1} \bit{1}{1} \bit{0}{1} \bit{1}{1} \bit{0}{1} \bit{1}{1}
\nextwave{Address} \known{\tiny{0b00}}{2} \known{\tiny{0b01}}{2} \known{\tiny{0b10}}{2} \known{\tiny{0b11}}{2} \known{\tiny{0b00}}{2} \known{\tiny{0b01}}{2} \known{\tiny{0b10}}{2} \known{\tiny{0b11}}{2}
\nextwave{SPO} \known{\tiny{0}}{0.3} \known{\tiny{0b10}}{2} \known{\tiny{0b01}}{2} \known{\tiny{0b00}}{2} \known{\tiny{0b11}}{2} \known{\tiny{0b10}}{2} \known{\tiny{0b01}}{2} \known{\tiny{0b00}}{2} \known{\tiny{0b11}}{1.7}
\nextwave{Data In} \known{\tiny{0b11}}{1} \known{\tiny{0b10}}{2} \known{\tiny{0b01}}{2} \known{\tiny{0b00}}{2} \known{\tiny{0b11}}{2} \known{\tiny{0b10}}{2} \known{\tiny{0b01}}{2} \known{\tiny{0b00}}{2} \known{\tiny{0b11}}{1} 
\nextwave{Data Out$_2$} \known{\tiny{0b00}}{1}  \known{\tiny{0b10}}{2} \known{\tiny{0b01}}{2} \known{\tiny{0b00}}{2} \known{\tiny{0b11}}{2} \known{\tiny{0b10}}{2} \known{\tiny{0b01}}{2} \known{\tiny{0b00}}{2} \known{\tiny{0b00}}{1}
\nextwave{\textbf{Data Out$_{10}$}} \known{\tiny{0}}{1} \known{\tiny{2}}{2} \known{\tiny{1}}{2} \known{\tiny{0}}{2} \known{\tiny{3}}{2}\known{\tiny{2}}{2} \known{\tiny{1}}{2} \known{\tiny{0}}{2} \known{\tiny{3}}{1}
\end{wave}
\caption{The second counter cycle.}
\end{figure}

As discussed previously, on each of the write-enable pulses the currently-selected address is written with its current value minus one. Now the output of the entire circuit is that newly-subtracted values; resulting in 2-1-0-3 (a rotating shift on the original contents in memory).

The following two cycles of the counter follow this same pattern, resulting in the same sequence of events, but a further rotated output that is: 1-0-3-2 and then 0-3-2-1. The timing diagrams for these cycles are shown below.

\begin{figure}[H]
\hspace{-2cm}
\begin{wave}[\textbf{Clock}]{9}{15}{33}
\nextwave{Reset} \bit{0}{16}
\nextwave{Counter[3:0]} \known{\tiny{0b0000}}{1} \known{\tiny{0b0001}}{1} \known{\tiny{0b0010}}{1} \known{\tiny{0b0011}}{1} \known{\tiny{0b0100}}{1} \known{\tiny{0b0101}}{1} \known{\tiny{0b0110}}{1} \known{\tiny{0b0111}}{1} \known{\tiny{0b1000}}{1} \known{\tiny{0b1001}}{1} \known{\tiny{0b1010}}{1} \known{\tiny{0b1011}}{1} \known{\tiny{0b1100}}{1} \known{\tiny{0b1101}}{1} \known{\tiny{0b1110}}{1} \known{\tiny{0b1111}}{1}
\nextwave{Write Enable} \bit{0}{9} \bit{1}{1} \bit{0}{1} \bit{1}{1} \bit{0}{1} \bit{1}{1} \bit{0}{1} \bit{1}{1}
\nextwave{Address} \known{\tiny{0b00}}{2} \known{\tiny{0b01}}{2} \known{\tiny{0b10}}{2} \known{\tiny{0b11}}{2} \known{\tiny{0b00}}{2} \known{\tiny{0b01}}{2} \known{\tiny{0b10}}{2} \known{\tiny{0b11}}{2}
\nextwave{SPO} \known{\tiny{3}}{0.3} \known{\tiny{0b10}}{2} \known{\tiny{0b01}}{2} \known{\tiny{0b00}}{2} \known{\tiny{0b11}}{2} \known{\tiny{0b10}}{2} \known{\tiny{0b01}}{2} \known{\tiny{0b00}}{2} \known{\tiny{0b11}}{1.7}
\nextwave{Data In} \known{\tiny{0b11}}{1} \known{\tiny{0b10}}{2} \known{\tiny{0b01}}{2} \known{\tiny{0b00}}{2} \known{\tiny{0b11}}{2} \known{\tiny{0b10}}{2} \known{\tiny{0b01}}{2} \known{\tiny{0b00}}{2} \known{\tiny{0b11}}{1} 
\nextwave{Data Out$_2$} \known{\tiny{0b11}}{1}  \known{\tiny{0b01}}{2} \known{\tiny{0b00}}{2} \known{\tiny{0b11}}{2} \known{\tiny{0b10}}{2} \known{\tiny{0b01}}{2} \known{\tiny{0b00}}{2} \known{\tiny{0b11}}{2} \known{\tiny{0b10}}{1}
\nextwave{\textbf{Data Out$_{10}$}} \known{\tiny{3}}{1} \known{\tiny{1}}{2} \known{\tiny{0}}{2} \known{\tiny{3}}{2} \known{\tiny{2}}{2}\known{\tiny{1}}{2} \known{\tiny{0}}{2} \known{\tiny{3}}{2} \known{\tiny{2}}{1}
\end{wave}
\caption{The third counter cycle.}
\end{figure}

\begin{figure}[H]
\hspace{-2cm}
\begin{wave}[\textbf{Clock}]{9}{15}{49}
\nextwave{Reset} \bit{0}{16}
\nextwave{Counter[3:0]} \known{\tiny{0b0000}}{1} \known{\tiny{0b0001}}{1} \known{\tiny{0b0010}}{1} \known{\tiny{0b0011}}{1} \known{\tiny{0b0100}}{1} \known{\tiny{0b0101}}{1} \known{\tiny{0b0110}}{1} \known{\tiny{0b0111}}{1} \known{\tiny{0b1000}}{1} \known{\tiny{0b1001}}{1} \known{\tiny{0b1010}}{1} \known{\tiny{0b1011}}{1} \known{\tiny{0b1100}}{1} \known{\tiny{0b1101}}{1} \known{\tiny{0b1110}}{1} \known{\tiny{0b1111}}{1}
\nextwave{Write Enable} \bit{0}{9} \bit{1}{1} \bit{0}{1} \bit{1}{1} \bit{0}{1} \bit{1}{1} \bit{0}{1} \bit{1}{1}
\nextwave{Address} \known{\tiny{0b00}}{2} \known{\tiny{0b01}}{2} \known{\tiny{0b10}}{2} \known{\tiny{0b11}}{2} \known{\tiny{0b00}}{2} \known{\tiny{0b01}}{2} \known{\tiny{0b10}}{2} \known{\tiny{0b11}}{2}
\nextwave{SPO} \known{\tiny{0}}{0.3} \known{\tiny{0b10}}{2} \known{\tiny{0b01}}{2} \known{\tiny{0b00}}{2} \known{\tiny{0b11}}{2} \known{\tiny{0b10}}{2} \known{\tiny{0b01}}{2} \known{\tiny{0b00}}{2} \known{\tiny{0b11}}{1.7}
\nextwave{Data In} \known{\tiny{0b11}}{1} \known{\tiny{0b10}}{2} \known{\tiny{0b01}}{2} \known{\tiny{0b00}}{2} \known{\tiny{0b11}}{2} \known{\tiny{0b10}}{2} \known{\tiny{0b01}}{2} \known{\tiny{0b00}}{2} \known{\tiny{0b11}}{1} 
\nextwave{Data Out$_2$} \known{\tiny{0b10}}{1}  \known{\tiny{0b00}}{2} \known{\tiny{0b11}}{2} \known{\tiny{0b10}}{2} \known{\tiny{0b01}}{2} \known{\tiny{0b00}}{2} \known{\tiny{0b11}}{2} \known{\tiny{0b10}}{2} \known{\tiny{0b01}}{1}
\nextwave{\textbf{Data Out$_{10}$}} \known{\tiny{2}}{1} \known{\tiny{0}}{2} \known{\tiny{3}}{2} \known{\tiny{2}}{2} \known{\tiny{1}}{2}\known{\tiny{0}}{2} \known{\tiny{3}}{2} \known{\tiny{2}}{2} \known{\tiny{1}}{1}
\end{wave}
\caption{The fourth counter cycle.}
\end{figure}

\section{Summary}
In summary, the output of the circuit is as follows:

\begin{center}
\begin{tabular}{cccc|lll}
\multicolumn{4}{c}{Outputs} & \multicolumn{3}{c}{Continued Sequence} \\
\hline
\textbf{3} & \textbf{2} & \textbf{1} & \textbf{0} & - & - & - \\
\textbf{3} & \textbf{2} & \textbf{1} & \textbf{0} & - & - & - \\
\hline
2 & 1 & 0 & \textbf{3} & \textbf{2} & \textbf{1} & \textbf{0} \\
2 & 1 & 0 & \textbf{3} & \textbf{2} & \textbf{1} & \textbf{0} \\
\hline
1 & 0 & \textbf{3} & \textbf{2} & \textbf{1} & \textbf{0} & - \\
1 & 0 & \textbf{3} & \textbf{2} & \textbf{1} & \textbf{0} & - \\
\hline
0 & \textbf{3} & \textbf{2} & \textbf{1} & \textbf{0} & - & -\\
0 & \textbf{3} & \textbf{2} & \textbf{1} & \textbf{0} & - & -\\
\vdots & \vdots & \vdots & \vdots & \vdots & \vdots & \vdots \\
\end{tabular}
\end{center}

After these four cycles, the sequence is complete and restarts from the beginning.

\end{document}